\mainlanguage[cz]
\environment env_dis

\startluacode
        sorters.setlanguage("cz")
\stopluacode


\usebtxdefinitions[apa]

\unprotect
\def\textampersand{a}
\protect

\setuplocalinterlinespace[line=2.8ex] %spacing 1

\setupbtxlabeltext
  [cz]
  [apa:number={},
   apa:edition={ed.}, % edición
   apa:Editor={ed.}, % Ed./Eds.
   apa:Editors={edd.},
   apa:Volume={Vol.},   % Volumen
   apa:Volumes={Vols.},
   p={s.}, %single page
   pp={s.} %prefix for pages
   ] 

\setupbtx
[apa:cite:authoryear]
[inbetween={\btxspace\space}]


 \setupbtx
 [apa:cite]
 [left=,
  right=,
separator:2={\btxsemicolon}, % :0 and :1 - between items of a list
separator:3={\btxsemicolon},
separator:4={\btxsemicolon},
separator:names:2={\btxcomma},
separator:names:3={\btxnobreakspace{a}\space},
separator:names:4={\btxnobreakspace{a}\space},   
etaldisplay=2,
etallimit=3
]

\usebtxdataset[ancient][ancient.bib]
\definebtxrendering[ancient][apa][dataset=ancient]
\usebtxdataset[epigraphic][epigraphic.bib]
\definebtxrendering[epigraphic][apa][dataset=epigraphic]
\usebtxdataset[secondary][secondary.bib]
\definebtxrendering[secondary][apa][dataset=secondary]

\starttext
\startchapter[title={Použité zkratky a bibliografie}, reference={bibliografie}, marking={Seznam použitých zkratek a bibliografie}]

\setuplayout[double]
\setuppagenumbering[alternative=doublesided]
%\unprotect
%\usemodule[database]

\section{Seznam použitých zkratek}
%\startchapter[title={Seznam použitých zkratek}, reference={zkratky}, marking={Seznam použitých zkratek}]
\setuplocalinterlinespace[line=1.5ex]
\startlines

 
{\bi AE}: {\em L’Année épigraphique}. ed. Cagnat, R. Paris 1888-1992. ed. Corbier, M. Paris 1992- .

{\bi CIL}: {\em Corpus Inscriptionum Latinarum, consilio et auctoritate Academiae litterarum regiae Borussicae editum}. 1863 - , Berlin. \goto{http://cil.bbaw.de/cil_en/dateien/datenbank_eng.php}[url(http://cil.bbaw.de/cil_en/dateien/datenbank_eng.php)] 

{\bi EDH}: {\em Epigraphic Database Heidelberg}, databáze řeckých a 
latinských nápisů, \goto{http://edh-www.adw.uni-heidelberg.de/home/}[url(http://edh-www.adw.uni-heidelberg.de/home/)]

{\bi GIS}: geografický informační systém

{\bi Gyuzelev 2002}: Gyuzelev, M. (2002). Ancient Funerary Monuments at Sozopol Archaeological Museum. {\em Izvestija na narodnija muzej Burgas} 4, 119–129.

{\bi Gyuzelev 2005}: Gyuzelev, M. (2005). Ancient Funerary Monuments Found at the Necropolis of Kalfata (Sozopol) in the Year of 2002. In Stoyanov, T. (Ed.). {\em Heros Hephaistos: Studia in honorem Liubae Ognenova-Marinova}, 131–138. Sofia: Faber Publisher.

{\bi Gyuzelev 2013}: Gyuzelev, M. (2013). Tituli sepulcrales in necropoli antiqua locis dictis Kalfata and Budzaka prope urbem Sozopolim reperti in effosionibus annorum MMIV et MMV. {\em Il Mar Nero} 7, 115–148.

{\bi HAT}: {\em Hellenization of Ancient Thrace}, databáze vytvořená pro účely této disertační práce, dostupná na adrese \goto{https://github.com/petrajanouchova/hat_project}[url(https://github.com/petrajanouchova/hat_project)] 

{\bi I Aeg Thrace}: Loukopoulou, L. D., Parissaki, M. G., Psoma, S., Zournatzi, A. (2005). {\em Inscriptiones antiquae partis Thraciae quae ad ora maris Aegaei site est: praefecture Xanthes, Rhodopes et Hebri}. Athens: Diffusion de Boccard.

{\bi IG}: {\em Inscriptiones Graecae}. 1903- , Berlin. 

{\bi IG Bulg}: Mihailov, G. (1956, 1958, 1961, 1964, 1966, 1970, 1997). {\em Inscriptiones graecae in Bulgaria repertae}, 5 vols. Sofia: Academiae Litterarum Bulgaricae.

{\bi IK Byzantion}: Lajtar, A. (2000). {\em Die Inschriften von Byzantion}. Bonn: Rudolf Habelt Verlag.

{\bi IK Sestos}: Krauss, J. (1980). {\em Die Inschriften von Sestos und der Thrakischen Chersonesos}. Bonn: Rudolf Habelt Verlag.

{\bi LSJ}: Liddell, H. G., Scott, R., Jones, H. S., McKenzie, R. (1996). {\em A Greek-English Lexicon}. Oxford: Clarendon Press.

{\bi Manov 2008}: Manov, M. (2008). {\em Selishnijat Zhivot v dolinata na Sredna Struma spored antichnite epigrafski pametnitsi ot IV/III v. pr. Ch. - III v. sl. Ch.} Sofia: BAN - NAIM.

{\bi QGIS}: program {\em QuantumGIS} nebo také {\em QGIS} sloužící k vytváření, zobrazování a analyzování geografických dat, \goto{http://www.qgis.org/}[url(http://www.qgis.org/)]

{\bi Perinthos-Herakleia}: Sayar, M. H. (1998). {\em Perinthos-Herakleia (Marmara Ereğlisi) und Umgebung. Geschichte, Testimonien, griechische und lateinische Inschriften}. Vienna: Österreichische Akademie der Wissenschaften.

{\bi PHI}: {\em Packard Humanities Institute}, databáze řeckých nápisů {\em Searchable Greek Inscriptions},  \goto{http://inscriptions.packhum.org/}[url(http://inscriptions.packhum.org/)]

{\bi R}: statistický program {\em R}, \goto{https://www.r-project.org/about.html}[url(https://www.r-project.org/about.html)]

{\bi SEG}: {\em Supplementum Epigraphicum Graecum}. Vols. 1-11, ed. Jacob E. Hondius, Leiden 1923-1954. Vols. 12-25, ed. Arthur G. Woodhead. Leiden 1955-1971. Vols. 26-41, eds. Henry W. Pleket a Ronald S. Stroud. Amsterdam 1979-1994. Vols. 42-44, eds. Henry W. Pleket, Ronald S. Stroud a Johan H. M. Strubbe. Amsterdam 1995-1997. Vols. 45-49, eds. Henry W. Pleket, Ronald S. Stroud, Angelos Chaniotis a Johan H. M. Strubbe. Amsterdam 1998-2002. Vol. 50, eds. Angelos Chaniotis, Ronald S. Stroud, Johan H. M. Strubbe. Amsterdam. 2003. Vols. 51-58, eds. Angelos Chaniotis, Ronald S. Stroud, Nicolas Papazarkadas, and R. A. Tybout. Leiden 2005-2012. Vols. 59- , eds. Angelos Chaniotis, Thomas Corsten, Nicolas Papazarkadas, and R. A. Tybout. Leiden. 2013- . \goto{http://referenceworks.brillonline.com/browse/supplementum-epigraphicum-graecum}[url(http://referenceworks.brillonline.com/browse/supplementum-epigraphicum-graecum)]

{\bi TRAP}: {\em The Tundzha Regional Archaeological Project}, archeologický projekt vedený S. A. Rossem a A. Sobotkovou, Macquarie University, Sydney, Australia, \goto{www.tundzha.org}[url(www.tundzha.org)]

{\bi Velkov 1991}: Velkov, V. (1991). Nadpisi ot Kabile. In Velkov, V. (Ed.), {\em Kabile} 2, 7–53. Sofia: Balgarska Akademia na naukite.

{\bi Velkov 2005}: Velkov, V. (2005). Inscriptions antiques de Messambria (1964-1984). In  Karayotov, I. (Ed.), {\em Nessebre III}, 159–193. Burgas: Spring Ltd.

\stoplines

\section{Seznam zkratek antických autorů a děl}

%\setuplocalinterlinespace[line=1.5ex]
\startlines


{\bi Aischinés  De falsa legatione}: Aischinés, {\em O falešném vyslanectví}

{\bi Arist.}: Aristotelés

{\bi Aristoph.  Ach.}: Aristofanés, {\em Acharňané}

{\bi Arr.  Anab.}: Arriános, {\em Anabasis}

{\bi Arr.  Peripl. Ponti Euxini}: Arriános, {\em Periplous Ponti Euxini}

{\bi Dem.  Contra Timocratem (23)}: Démosthenés, {\em Proti Tímokratovi}

{\bi Dem.  De Chersoneso (8)}: Démosthenés, {\em O záležitostech na Chersonésu}

{\bi Dem.  Philippica quarta (10)}: Démosthenés, {\em Čtvrtá řeč proti Filippovi}

{\bi Dem.  De Corona (18)}: Démosthenés, {\em O věnci}

{\bi Dio. Cass.}: Dión Cassius

{\bi D. S.}: Diodóros Sicilský

{\bi Eutrop.  Breviarium}: Eutropius, {\em Breviarium}

{\bi Hom.  Il.}: Homér, {\em Ílias}

{\bi Hdt.}: Hérodotos

{\bi Klaudios Ptolemaios,  Geogr.}: Klaudios Ptolemaios, {\em Geographica}

{\bi Plat.  Resp.}: Platón, {\em Ústava}

{\bi Plin.  H. N.}: Plinius Starší, {\em Historia Naturalis}

{\bi Plut.  Kim.}: Plútarchos, {\em Kimón}

{\bi Pomp. Mela}: Pomponius Mela

{\bi Steph. Byz.}: Stefanos Byzantský, {\em Ethica}

{\bi Strabo}: Strabón

{\bi Suetonius  Aug.,  Tib.}: Suetonius, {\em Augustus}, {\em Tiberius}

{\bi Tac.  Ann.}: Tacitus, {\em Annales}

{\bi T. Liv.}: Titus Livius

{\bi Thuc.}: Thúkýdidés

{\bi Xen.  Anab.}: Xenofón, {\em Anabasis}

\stoplines


\section{Primární antické literární prameny}
* Jméno antického autora je uvedeno v podobě uvedené v citované publikaci 
\crlf

\placelistofpublications % aka \placebtxrendering
[ancient] % rendering defined above
[method=dataset] % i.e. all entries


\section{Primární epigrafické prameny}
\placelistofpublications % aka \placebtxrendering
[epigraphic] % rendering defined above
[method=dataset] % i.e. all entries

\section{Sekundární prameny}
\placelistofpublications % aka \placebtxrendering
[secondary] % rendering defined above
[method=dataset] % i.e. all entries

\stopchapter
\stoptext


