
\environment ../env_dis
\startcomponent section-charakteristika-epigrafické-produkce-1.-st.-př.-n.-l.-až-1.-st.-n.-l.
\section[charakteristika-epigrafické-produkce-1.-st.-př.-n.-l.-až-1.-st.-n.-l.]{Charakteristika epigrafické produkce 1. st. př. n. l. až 1. st. n. l.}

Nápisy datované do 1. st. př. n. l. až 1. st. n. l. se vyznačují narůstající otevřeností a multikulturalitou epigraficky aktivních komunit. Tomu nasvědčuje i prolínání onomastických tradic, rozvoj místního náboženství, ale i kultů nethráckého původu. Epigrafická aktivita je celkově nižší oproti předcházejícím stoletím, avšak narůstá celkový počet osob vystupujících na nápisech. Thrácká aristokracie se nicméně z epigrafických dokumentů vytrácí a stejně tak i ustávají epigrafické aktivity v thráckém vnitrozemí.


\startframedbox
{\em Celkem}~ 56 nápisů

{\em Region měst na pobřeží}~ Abdéra 3, Anchialos 1, Bizóné 1, Byzantion 30, Dionýsopolis 1, Maróneia 8, Mesámbria 1, Odéssos 3, Perinthos (Hérakleia) 2, Sélymbria 2, Séstos 1, Topeiros 1 (celkem 54 nápisů)

{\em Region měst ve vnitrozemí}~ 0\footnote{Celkem dva nápisy nebyly nalezeny v rámci regionu známých měst, editoři korpusů udávají jejich polohu vzhledem k nejbližšímu modernímu sídlišti či muzeu, kde se v současnosti nacházejí.}

{\em Celkový počet individuálních lokalit}~ 19

{\em Archeologický kontext nálezu}~ sídelní 1, náboženský 1, sekundární 5, neznámý 49

{\em Materiál}~ kámen 54 (mramor 52, neznámý 2), neznámý 2

{\em Dochování nosiče}~ 100 \letterpercent{} 4, 75 \letterpercent{} 2, 50 \letterpercent{} 5, 25 \letterpercent{} 10, nemožno určit 35

{\em Objekt}~ stéla 48, architektonický prvek 4, socha 2, neznámý 2

{\em Dekorace}~ reliéf 39, bez dekorace 17; reliéfní dekorace figurální 32 nápisů (vyskytující se motiv: jezdec 3, sedící osoba 1, skupina lidí 1, zvíře 1, funerální scéna/symposion 7), architektonické prvky 8 nápisů (vyskytující se motiv: naiskos 3, sloup 2, báze sloupu či oltář 2, architektonický tvar/forma 1)

{\em Typologie nápisu}~ soukromé 45, veřejné 9, neurčitelné 2

{\em Soukromé nápisy}~ funerální 35, dedikační 9, neznámý 1

{\em Veřejné nápisy}~ seznamy 1, honorifikační dekrety 6, státní dekrety 1, neznámý 1

{\em Délka}~ aritm. průměr 4,14 řádku, medián 2, max. délka 48, min. délka 1

{\em Obsah}~ dórský dialekt 2, latinský text 1 nápis, písmo římského typu 1; hledané termíny (administrativní termíny 10 - celkem 16 výskytů, epigrafické formule 5 - 11 výskytů, honorifikační 6 - 8 výskytů, náboženské 11 - 18 výskytů, epiteton 4 - počet výskytů 5)

{\em Identita}~ řecká božstva 2, egyptská božstva 3, pojmenování míst a funkcí typických pro řecké náboženské prostředí, místní thrácká božstva, regionální epiteton 1, subregionální epiteton 3, kolektivní identita 14 termínů, celkem 14 výskytů - obyvatelé řeckých obcí z oblasti Thrákie 10, ale i mimo ni 2, kolektivní pojmenování Thráx 1, Rómaios 1; celkem 122 osob na nápisech, 31 nápisů s jednou osobou; max. 46 osob na nápis, aritm. průměr 2,18 osoby na nápis, medián 1; komunita multikulturního charakteru se zastoupením řeckého, římského a thráckého prvku, jména pouze řecká (33,9 \letterpercent{}), pouze thrácká (3,57 \letterpercent{}), pouze římská (8,92 \letterpercent{}), kombinace řeckého a thráckého (10,71 \letterpercent{}), kombinace řeckého a římského (7,14 \letterpercent{}), kombinace thráckého a římského (3,57 \letterpercent{}), kombinovaná řecká, thrácká a římská jména (5,35 \letterpercent{}), jména nejistého původu (12,49 \letterpercent{}), beze jména (14,28 \letterpercent{}); geografická jména z oblasti Thrákie 9, geografická jména mimo Thrákii 1;



\stopframedbox

Oproti nápisům datovaným do 2. až 1. st. př. n. l. je u skupiny nápisů datovaných do 1 st. př. n. l. až 1. st. n. l. pozorovatelný pokles o 45 \letterpercent{} celkového počtu nápisů. Produkční centra se nachází výhradně na pobřeží, jak je možné vidět na mapě 6.06 v Apendixu 2. Hlavním produkčním centrem je i nadále Byzantion, odkud pochází přes polovinu všech nápisů. Pozici menšího produkčního centra si i nadále udržuje Maróneia s osmi nápisy, což představuje 14 \letterpercent{} celkového počtu nápisů. Materiálem, z nějž jsou nápisy zhotovovány, je výhradně kámen, většina nápisů má tvar stély a slouží jako funerální nápis. Zhruba 16 \letterpercent{} nápisů představují nápisy veřejné, což je nepatrně větší zastoupení než v předcházejícím období.

\subsection[funerální-nápisy-10]{Funerální nápisy}

Dochovaných 35 funerálních nápisů pochází výhradně z kontextu řeckých komunit na pobřeží, nicméně onomastické záznamy nasvědčují na proměňující se zvyklosti a pravděpodobně i složení tamější populace. Tři čtvrtiny nápisů pocházejí z Byzantia, kde lze pozorovat narůstající politickou moc Říma, která se projevila i na charakteru nápisů.

O narůstající roli římského kulturního vlivu svědčí i nápis {\em I Aeg Thrace} 72 je psán výhradně latinsky. Navíc se v této době u funerálních nápisů začíná zvyk uvádět roky, jichž se zemřelý dožil, většinou zaokrouhlené na pět let, což je zvyk typický pro římské nápisy (MacMullen 1982, 238). Mezi jmény vyskytujících se na nápisech nicméně i nadále převládají řecká jména, kterých je přibližně čtyřikrát více než jmen římských a pětkrát více než jmen thráckých. Jako měst svého původu označuje Byzantion na nápise {\em IK Byzantion} 352 muž nesoucí čistě římská jména Gaios Ioulios, což svědčí o jeho dvojí loajalitě směrem k římské tradici, ale i politické příslušnosti k Byzantiu, tedy původně řecké kolonii. Z toho lze soudit, že jména v této době přestávají být jednoznačným ukazatelem původu, ale spíše se jedná o uvědomělou volbu identity, vědomým prohlášením přináležitosti k určité komunitě.

\subsection[dedikační-nápisy-10]{Dedikační nápisy}

Dedikační nápisy se pomalu začínají prosazovat i v rámci místních thráckých kultů, avšak řecká božstva mají stále převahu. Dedikačních nápisů se dochovalo celkem devět, z nichž tři jsou věnovány Apollónovi, který v jednom případě nesl lokální přízvisko {\em Eptaikenthos} a v jednom {\em Toronténos}, dále tři nápisy věnované {\em héróovi}, který dvakrát nesl přízvisko {\em Stomiános}, jednou {\em Perkón}. V jednom případě je nápis věnován neznámým božstvům ze Sélymbrie. Nápisy pocházejí převážně od osob nesoucí řecká jména, nicméně v případě nápisu {\em Perinthos-Herakleia} 51 věnovaného Apollónovi {\em Toronténovi} se setkáváme s dedikanty nesoucími thrácká jména. Římská jména se objevují pouze v jednom případě na nápise {\em I Aeg Thrace} 202 z Maróneie.

\subsection[veřejné-nápisy-10]{Veřejné nápisy}

Veřejných nápisů se dochovalo celkem devět, z nichž šest představují honorifikační dekrety udělené institucemi řeckých měst významným jedincům.\footnote{Jako např. thrácký král Kotys na nápise {\em I Aeg Thrace} 207 z Maróneie,} Politickou autoritu představují jednotlivé instituce řeckých {\em poleis}, osoba thráckého krále a jednotliví stratégové, případně sama autorita Říma.\footnote{Příkladem je nápis {\em IG Bulg} 5 5011 z Dionýsopole; {\em IK Sestos} 1 ze Séstu.}

Osobní jména na veřejných nápisech dokazují stále ještě řecký charakter epigraficky aktivní populace: dochovalo se celkem 93 řeckých jmen, 11 thráckých a devět římských.\footnote{Nápis {\em IG Bulg} 1,2 46 představuje seznam kněžích nejmenovaného kultu z Odéssu a dochovalo se na něm celkem 50 jmen, z čehož 46 bylo řeckého původu a dvě byla identifikována jako jména thrácká a dvě římská. Celkem 46 kněžích byli výhradně muži a původ jejich jmen byl takřka výhradně řecký, což může nasvědčovat i jisté konzervativnosti nejmenovaného kultu.} Formule použité v honorifikačních nápisech dokumentují jisté přetrvání zvyklostí a procedur spojených s vystavením nápisů, podobně jako v předcházejících stoletích, avšak pokles výskytu tradičních formulí spojených s udílením poct značí proměnu vnitřního uspořádání politických autorit, která se odráží i v použitém jazyce veřejných nápisů.\footnote{Nápis {\em IG Bulg} 1,2 320 z Mesámbrie popisuje proceduru korunovace, což byla jedna z poct udílených v rámci řecké {\em polis}.}

\subsection[shrnutí-14]{Shrnutí}

Na přelomu 1. st. př. n. l. a 1. st. n. l. se setkáváme s projevy celospolečenských změn, které se odráží i na charakteru epigrafické produkce z Thrákie. Dochází k poklesu celkové produkce, jejímu přesunutí do městských center na pobřeží, v čele s Byzantiem. Řím začíná hrát důležitou roli jak v měnícím se charakteru a uspořádání společnosti, ale i pozvolna se vyvíjejících zvyklostech, spojených např. s proměnou tradičního epigrafického jazyka či přijímáním nových osobních jmen.

\stopcomponent