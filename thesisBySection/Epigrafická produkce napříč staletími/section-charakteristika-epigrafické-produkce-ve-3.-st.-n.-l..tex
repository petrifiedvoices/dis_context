
\environment ../env_dis
\startcomponent section-charakteristika-epigrafické-produkce-ve-3.-st.-n.-l.
\section[charakteristika-epigrafické-produkce-ve-3.-st.-n.-l.]{Charakteristika epigrafické produkce ve 3. st. n. l.}

Nápisy datované do 3. st. n. l. pocházejí ze tří čtvrtin z vnitrozemských městských center, jako je Augusta Traiana, Filippopolis a Serdica. Zvýšený výskyt geografických termínů a kolektivních pojmenování z oblastí mimo Thrákii naznačuje otevírání společnosti a migraci lidí z dalších částí římské říše. I nadále převládají dedikační nápisy, často věnované božstvům lokálního charakteru. Veřejné nápisy představují téměř polovinu všech nápisů, poprvé se v nich objevuje i místní samospráva na úrovni vesnic, nikoliv pouze na úrovni měst.

\startDelimitedTable
 {\em Celkem}~ 390 nápisů

{\em Region měst na pobřeží}~ Abdéra 5, Anchialos 3, Bizóné 1, Byzantion 20, Caron Limen 2, Dionýsopolis 7, Ferai 1, Maróneia 21, Maximiánúpolis 1, Mesámbriá 1, Odéssos 7, Perinthos (Hérakleia) 16, Sélymbria 2, Strýmé 1, Topeiros 9 (celkem 97 nápisů)

{\em Region měst ve vnitrozemí}~ Augusta Traiana 58, Discoduraterae 7, Filippopolis 49, Hadriánopolis 1, Hérakleia Sintská 2, Marcianopolis 4, Neiné 3, Nicopolis ad Istrum 25, Pautália 7, Plótinúpolis 4, Serdica 91, Traianúpolis 1, údolí středního toku řeky Strýmónu (Particopolis a okolí) 29 (celkem 281 nápisů)\postponenotes\footnote{Celkem 12 nápisů nebylo nalezeno v rámci regionu známých měst, editoři korpusů udávají jejich polohu vzhledem k nejbližšímu modernímu sídlišti. Devět nápisů pochází z pobřeží Egejského moře a tři z bulharského vnitrozemí.}

{\em Celkový počet individuálních lokalit}~ 115

{\em Archeologický kontext nálezu}~ funerální 9, sídelní 47 (z toho obchodní 20), náboženský 79, sekundární 44, jiný 3, neznámý 208

{\em Materiál}~ kámen 385 (mramor 237, z Prokonnésu 1; vápenec 104, jiný 17, z toho syenit 6, varovik 1, granit 1, tuf 1, břidlice 1; neznámý 27), keramika 2, neznámý 3

{\em Dochování nosiče}~ 100 \letterpercent{} 45, 75 \letterpercent{} 47, 50 \letterpercent{} 74, 25 \letterpercent{} 60, oklepek 6, kresba 25, ztracený 8, nemožno určit 125

{\em Objekt}~ stéla 215, architektonický prvek 145, socha 18, jiný 1, neznámý 11

{\em Dekorace}~ reliéf 259, malovaná 1, bez dekorace 130; reliéfní dekorace figurální 115 nápisů (vyskytující se motiv: jezdec 59, sedící osoba 2, stojící osoba 7, skupina lidí 3, zvíře 1, Artemis 1, Asklépios 3, Héraklés 1, Zeus a Héra 1, Dionýsos 2, scéna lovu 2, funerální scéna/symposion 3, funerální portrét 13, socha 1, jiný 8), architektonické prvky 136 nápisů (vyskytující se motiv: naiskos 12, sloup 49, báze sloupu či oltář 70, architektonický tvar/forma 17, geometrický motiv 2, florální motiv 15, věnec 2, jiný 13)

{\em Typologie nápisu}~ soukromé 203, veřejné 176, neurčitelné 11

{\em Soukromé nápisy}~ funerální 76, dedikační 128, vlastnictví 2, jiný 5\postponenotes\footnote{Několik nápisů mělo vzhledem ke své nejednoznačnosti kombinovanou funkci, proto je součet nápisů obou typů vyšší než celkový počet soukromých nápisů.}

{\em Veřejné nápisy}~ seznamy 7, honorifikační dekrety 97, státní dekrety 11, náboženský 1, jiný 52, neznámý 9

{\em Délka}~ aritm. průměr 8,62 řádku, medián 7, max. délka 270, min. délka 1

{\em Obsah}~ dórský dialekt 1, iónsko-attický 1, latinský text 12 nápisů, písmo římského typu 168; hledané termíny (administrativní termíny 53 - celkem 502 výskytů, epigrafické formule 26 - 315 výskytů, honorifikační 9 - 22 výskytů, náboženské 30 - 177 výskytů, epiteton 24 - počet výskytů 55)

{\em Identita}~ řecká božstva 13, egyptská božstva 0, římská božstva 1, pojmenování míst a funkcí typických pro řecké náboženské prostředí, nárůst počtu lokálních kultů, regionální epiteton 11, subregionální epiteton 13, kolektivní identita 38 termínů, celkem 162 výskytů - obyvatelé řeckých obcí z oblasti Thrákie 20, mimo ni 3; kolektivní pojmenování etnik či kmenů (barbaros 3, Thráx 64, Rómaios 11, Kampános 1, Makedón 1), obyvatelé thráckých vesnic 10; celkem 1129 osob na nápisech, 134 nápisů s jednou osobou; max. 154 osob na nápis, aritm. průměr 2,89 osoby na nápis, medián 1; komunita multikulturního charakteru se zastoupením řeckého, římského a thráckého prvku, se nadpoloviční přítomností římského prvku, jména pouze řecká (6,66 \letterpercent{}), pouze thrácká (3,58 \letterpercent{}), pouze římská (27,43 \letterpercent{}), kombinace řeckého a thráckého (2,56 \letterpercent{}), kombinace řeckého a římského (19,23 \letterpercent{}), kombinace thráckého a římského (5,12 \letterpercent{}), kombinovaná řecká, thrácká a římská jména (10,25 \letterpercent{}), jména nejistého původu (8,18 \letterpercent{}), beze jména (16,92 \letterpercent{}){\bf ;} geografická jména z oblasti Thrákie 24, geografická jména mimo Thrákii 19;
\stopDelimitedTable
\flushnotes

Ve 3. st. n. l. je úroveň epigrafické produkce přibližně o polovinu vyšší než v předcházejícím století. I nadále pocházejí tři čtvrtiny nápisů z vnitrozemských oblastí, a to zejména z lokalit nalézajících se podél římské cesty {\em Via Diagonalis,} jako je Augusta Traiana a Filippopolis. Tato cesta procházející ze severovýchodu na jihovýchod Thrákie spojovala Evropu s Malou Asií a sloužila k poměrně častému přesunu vojsk oběma směry, což s sebou neslo zvýšenou potřebu společenské regulace (Madzharov 2009, 70-131). Zbývající čtvrtina epigrafické produkce i nadále pochází z tří regionů řeckých měst na pobřeží: z okolí Perinthu a Byzantia na pobřeží Marmarského moře, z okolí Abdéry a Maróneie na pobřeží Egejského moře, a dále z Odéssu a Dionýsopole na pobřeží Černého moře. Konkrétní polohu míst nálezů nápisů ilustruje mapa 6.09a v Apendixu \in[Apendix2:::Apendix2].\footnote{Archeologický kontext je podobně jako u předcházejícího období z velké části neznámý, avšak zhruba u 18 \letterpercent{} lokalit je kontext určen jako náboženský a zhruba u 11 \letterpercent{} lokalit sídelní. Oproti předcházejícímu období dochází k mírnému nárůstu u obou kategorií, a téměř každý třetí nápis pochází buď z osídlení, či svatyně. Předpokládá se, že materiál na výrobu nápisů pocházel z místních zdrojů, jako např. z ostrova Prokonnésos v Marmarském moři. Zcela v tomto období chybí nápisy na kovových předmětech.}

Soukromé nápisy představují přes polovinu celého souboru. Podobně jako u nápisů datovaných do 2. až 3. st. n. l. převládají dedikační nápisy nad nápisy funerálními. Celý soubor doplňuje nebývale vysoký počet veřejných nápisů, které představují 45 \letterpercent{} všech nápisů z daného období.

\subsection[funerální-nápisy-15]{Funerální nápisy}

Funerálních nápisů ze 3. st. n. l. se dochovalo celkem 76, což představuje nárůst zhruba o 40 \letterpercent{} oproti funerálním nápisům z předcházejícího století. Podobně jako v předcházejícím období se na funerálních nápisech uchovávají standardní formule, typické pro tento druh nápisů. Jejich celkový výskyt se nicméně snižuje, pravděpodobně v reakci na proměňující se funerální ritus.\footnote{Např. typická formule na památku ({\em mnémé charin} či {\em mnéiás charin)} ve dochovala ve 21 případech. Místo pohřbu je nejčastěji nazýváno {\em mnémeion} ve třech, {\em larnax} v jednom případě pro sarkofág, {\em latomeion} ve třech případech taktéž pro sarkofág, {\em soros} v sedmi případech pro urnu, {\em chamosorion} pro plochou hrobku umístěnou na zemi. Celkem tři nápisy zmiňují vztyčení stély a ve dvou případech i zhotovení textu nápisu. Celkem čtyři nápisy oslovují okolo jdoucího poutníka ({\em chaire/chairete parodeita}), což je znatelně méně než v předcházejícím období.}

Typický text nápisu nese jméno nebožtíka, jeho zařazení v rámci komunity, jeho původ a jeho dosažené postavení či vykonané skutky. Dále je zde uveden zhotovitel nápisu a jeho vztah k zemřelému. Většinou se jedná o člena rodiny, přítele či kolegu z armády. Důvody pro uvádění pozůstalých jsou pravděpodobně spojeny s dědickými nároky a souvisejícími povinnostmi (MacMullen 1982; Meyer 1990).

V této době je možné pozorovat větší nárůst geografických jmen, a to nejen u funerálních nápisů, často označujících geografický původ osoby, odkazující zejména na místa v Malé Asii, z měst Malé Asie jako je Níkaia či Smyrna. Zvýšená přítomnost osob z Malé Asie svědčí o migračních proudech z této oblasti, které začaly již koncem 2. st. n. l., ale na nápisech se projevily zejména až ve 3. st. n. l. (Slawisch 2007a, 171; Sharankov 2011, 141, 143; Delchev 2013, 18-19). Dále je možné pozorovat i zvýšenou přítomnost osob ze západních římských provincií, která ale nedosahuje úrovně přítomnosti osob z maloasijských regionů. I přes zvýšenou míru výskytu uvádění geografického původu nebyl nejdůležitějším a ani jediným faktorem identifikace jednotlivce.\footnote{V jednom případě na nápise {\em IG Bulg} 1,2 1 z lokality Tvardica v regionu Caron Limen se setkáváme s označením barbar, které se vztahuje ke skupině lupičů blíže nespecifikovaného původu a etnicity, jimž se podařilo uniknout knězi Aureliovi Flaviovi Markovi. Termín {\em barbaros} tak nelze spojovat výlučně s thráckým etnikem, a pokud ano, jedná se o etický emotivně zabarvený popis nespecifické etnické skupiny.} Naopak zmínka o povolání či zvolené životní dráze zemřelého se dochovala na pětině nápisů, což poukazuje na narůstající váhu prezentace povolání a dosaženého společenského postavení.\footnote{Setkáváme se s šesti vojáky různých hodností, dále s čtyřmi gladiátory nejrůznějších specializací, jedním strážným, jedním knězem, jedním propuštěncem, jedním členem {\em búlé} a jedním členem {\em gerúsie}.}

Dle výskytu osobních jmen v průběhu ve 3. st. n. l. dochází k většímu zapojení osob nesoucích thrácká jména do praxe vztyčování funerálních nápisů a jejich podíl je nyní již takřka pětinový.\footnote{Celkem se na nápisech dochovalo 182 jmen, z nichž 43 \letterpercent{} je řeckých, 32 \letterpercent{} římských, 18,5 \letterpercent{} thráckých a 6,5 \letterpercent{} je nejistého původu. Řecká jména, ať už samotná či v kombinaci, se vyskytovala zejména v okolí řeckých měst na pobřeží Egejského moře, Perinthu a Byzantia. Ve vnitrozemské Thrákii se řecká jména vyskytovala zejména v údolí středního toku Strýmónu. Až polovina osob nesoucí řecké jméno nesla i jméno římské, což značí přijetí římského onomastického systému. Římská jména bez kombinace s thráckým, řeckým či jiným jménem se dochovala pouze na deseti nápisech rovnoměrně rozmístěných na území Thrákie.} Thrácká jména pocházejí především z nápisů nalezených v údolí středního toku Strýmónu a částečně v okolí Maróneie a Topeiru. Polovina thráckých jmen se vyskytovala v kombinaci se jménem římským, což taktéž naznačuje proměnu onomastické tradice a obecně rozšířené přijímání římských jmen. Přijetí římského jména a jeho uvedení na nápisech mohlo v dřívějších dobách signalizovat jak společenské postavení dané osoby, tak i jeho nejbližší rodiny, vzhledem k tomu, že až do roku 212 n. l. se udílelo především za individuální zásluhy. V roce 212 n. l. však bylo uděleno římské občanství a právo nosit jméno římského původu všem obyvatelům římské říše. Předpokládá se, že právo nosit římské jméno tak postupně ztrácelo váhu a společenskou prestiž, kterou mělo před rokem 212 n. l. (Blanco-Perez 2016, 271).

Převaha textů byla i nadále v řeckém jazyce a latinské nápisy se objevovaly především v souvislosti s vojáky či veterány a pocházely z okolí míst s permanentně umístěnou vojenskou posádkou.\footnote{Latinský text se objevil u funerálních nápisů pouze pětkrát, a to u zemřelých vojáků sloužících v římské armádě ve čtyřech případech a u propuštěnce v jednom případě. Čistě latinský text nápisu se objevil třikrát v Byzantiu, a to zejména u vojáků nesoucích jména římského či nethráckého původu. Jeden bilingvní překlad identického řeckého textu se objevil ve Filippopoli, a to u vojáka nesoucího římské a thrácké jméno, který byl thráckého původu, ale kariéru si vybudoval v rámci římské armády. Latinský text byl uvedený na prvním místě a řecký text až jako druhý.} Ač latina sloužila jako oficiální jazyk římské armády a do jisté míry ovlivňovala i podobu funerálních nápisů zemřelých vojáků, tak se zdá, že v Thrákii volba jazyka byla dána spíše funkcí nápisu a publikem, jemuž byl nápis určen.\footnote{Pokud byl text určen pouze vojákům nethráckého původu, případně pokud se mělo poukazovat na postavení zemřelého v rámci armády, byla volena spíše latina jako je tomu u nápisů z Byzantia. Pokud však měl být nápis určen jak pro vojenskou, tak nevojenskou komunitu, tak docházelo k prolínání latiny a řečtiny či ke kompletním překladům textů, jako je tomu i nápisů z Perinthu, viz výše.}

Ve 3. st. n. l. dále narůstá míra standardizace funerálních nápisů a souvisejících procedur. Ve 18 případech se setkáváme s frází na konci nápisu postihující případné nové použití hrobky, ale zejména sarkofágu a urny, stejně jako v 1. a 2. st. n. l.\footnote{Nejvíce takovýchto nápisů bylo nalezeno v Byzantiu, celkem šest, dále pak v Maróneii čtyři nápisy, v Perinthu tři nápisy, dva v Topeiru a po jednom v Abdéře, Sélymbrii a Filippopoli. Převážně se jednalo o osoby nesoucí původní řecká jména a nově přijaté jméno Aurelios či Aurelia. V jednom případě se jedná o člena {\em gerúsie} z Filippopole, člena {\em búlé} z Perinthu a dceru veterána z Maróneie. Sarkofágy většinou sloužily pro dva a více členů rodiny a byly věnovány partnery či potomky zemřelých.} Z poměrně vysoké frekvence a rozšíření tohoto zvyku můžeme soudit, že se jednalo o častou praxi, kdy byla místa pohřbu využívána sekundárně a bylo nutné se chránit. Politická autorita města, pod jehož ochranu nápis spadal, tak určitou ochranu pravděpodobně zajišťovala, jinak by nedošlo k tak hojnému rozšíření této epigrafické formule do několika produkčních center.

\subsection[dedikační-nápisy-15]{Dedikační nápisy}

Dedikačních nápisů se ze 3. st. n. l. dochovalo 128, což představuje zhruba trojnásobný nárůst oproti 2. st. n. l. Dedikační nápisy i nadále převládají nad nápisy funerálními, podobně jako u skupiny nápisů datovaných do 2. až 3. st. n. l.\footnote{Na konci 3. st. n. l. však dochází ke změně a celkový počet funerálních nápisů opět převládá nad dedikacemi.} Pokračujícím fenoménem ve 3. st. n. l. jsou velké svatyně v blízkosti velkých měst či v podhorských oblastech v dostupnosti cest, které se staly populární jak mezi Thráky, tak i lidmi nesoucí jiná než thrácká jména.\footnote{Takřka polovina nápisů pochází z jedné svatyně v regionu města Serdica. Dalších 18 nápisů pochází z několika míst z regionu Augusty Traiany, a dalších devět nápisů z údolí středního toku Strýmónu.} Zvyk věnovat nápisy nebyl ve 3. st. n. l. omezen pouze na jednu část společnosti, ale podíleli se na něm lidé jak thráckého, tak jiného původu a různého společenského postavení.

Věnování jsou určena zejména Asklépiovi ve 27 případech, Diovi ve 12 případech, Apollónovi v pěti případech, Áreovi ve dvou případech. Tato božstva nesla většinou místní epiteton, což naznačuje pokračující propojení místních a původně řeckých kultů, stejně jako v 2. st. př. n. l.\footnote{Mezi tato božstva patří např. Zeus {\em Zbelthiúrdos}, Zeus {\em Paisúlénos}, Apollón {\em Dortazénos}, Asklépios {\em Kúlkússénos}, Árés {\em Saprénos}, dále {\em theos} {\em Salénos}, {\em Aularchénos}, {\em hérós} {\em Tisasénos} a {\em Marón}, {\em theos} {\em Asdúlos}.} Ve většině případů se jedná o prostou dedikaci bez většího množství detailů, ale zhruba čtvrtina nápisů poskytuje detailnější informace o dedikantech. Dedikace ve 14 případech zhotovili vojáci, ve třech případech členové {\em búlé}, ve dvou {\em thrakarchové} a v jednom případě {\em gymnasiarchés} a {\em archón} v jedné osobě. V jednom případě se jedná o hromadnou dedikaci {\em saltariů}, tj. římské verze polesných a lidí starajících se o lesní porost (Mihailov 1966, 275). Osobní jména poukazují na převahu původně římských jmen.\footnote{Celkem se dochovalo 185 jmen, z čehož zhruba polovina je římského původu, 23 \letterpercent{} řeckého původu, 18 \letterpercent{} thráckého původu a 10 \letterpercent{} jmen nebylo možné přesněji určit.} Dedikanti thráckého původu se vyskytovali téměř na pětině nápisů, což je zhruba o 8 \letterpercent{} více než ve století předcházejícím.\footnote{V kombinaci s římským jménem se thrácké jméno objevilo na 13 nápisech, v kombinaci s řeckým na devíti nápisech, a samostatně stojících na devíti nápisech.} Nápisy s thráckými jmény pocházely výhradně z vnitrozemí z okolí městských center: 11 z regionu Augusty Traiany, čtyři z regionu města Serdica, čtyři z regionu Filippopole. Celkem v šesti případech se jednalo o věnování vojáka či zastupitele zastávajícího vysokou pozici v administrativním aparátu provincie. Deset dedikačních nápisů s thráckým jménem neslo dekoraci v podobě jezdce na koni, což je prvek tradičně spojovaný právě s thráckou populací.

Již na přelomu 2. a 3. st. se objevují nové motivy reliéfní dekorace, jako jsou například výjevy znázorňující Asklépia, Héraklea, Dia a Héru, Dionýsa. Nejčastěji se opakujícím motivem je však jezdec na koni, který v mnoha případech může být spojován s fenoménem tzv. thráckého jezdce (Kazarow 1938; Dimitrova 2002; Oppermann 2006).\footnote{Jezdec na koni se objevil na nápisech ze 3. st. n. l. celkem 55krát, z čehož 44 nápisů pochází ze svatyně Asklépia {\em Liménia} ze Slivnice v okolí města Serdica, dalších šest ze svatyně u vesnice Viden v regionu města Augusta Traiana (Boteva 1985; Tabakova-Tsanova 1961).} Celkem 30 nápisů s dekorací jezdce na koni nese osobní jména: deset z nich obsahuje thrácká jména samostatně stojící či v kombinaci, a 20 obsahuje jiná než thrácká jména. Neznamená to tedy, že dedikace s vyobrazením jezdce na koni byla výsadně záležitostí thrácké populace, ale naopak, tzv. thrácký jezdec se stal ve 3. st. n. l. populární i mezi lidmi nesoucími řecká a římská jména. Nelze s jistotou tvrdit, že fenomén thráckého jezdce byl rozšířen výhradně v komunitě vojáků thráckého původu jak naznačuje Boteva (2005, 204), ale jedná se o celospolečenský fenomén. Celkem u osmi nápisů o sobě dedikant přímo uvádí, že se jedná o vojáka, což představuje zhruba 15 \letterpercent{} všech dedikací nesoucích motiv jezdce na koni z daného období.

Jako ilustrativní příklad složení epigraficky aktivní populace a náboženských zvyklostí 3. st. n. l. mohou sloužit nálezy ze svatyně Asklépia {\em Liménia} ze Slivnice. Tato svatyně patří k nejvýznamnějším svatyním, alespoň co do počtu nalezených nápisů. Bylo zde nalezeno celkem 69 dedikací nesoucích nápis a 286 anepigrafických votivních předmětů (Boteva 1985, 31; Mihailov 1997, 318).\footnote{Celkem 62 z těchto dedikací splňovalo chronologická kritéria a bylo datováno s přesností do jednoho až dvou století.} Celkem 24 nápisů bylo určeno Asklépiovi, z toho 14 s přízviskem {\em Liménios}, dále ve čtyřech případech označovaný jako {\em theos}, v sedmi jako {\em kyrios} a ve jednom případě jako {\em sótér}, tj. zachránce. Velká část dedikací, konkrétně 44, nese vyobrazení jezdce na koni, pouhé tři nápisy nesou reliéfní zobrazení Asklépia, Hygiee a Télesfora. Pokud jde o identitu dedikantů, zdá se, že tato lokalita byla navštěvována obyvatelstvem nesoucím řecká i thrácká jména, ale i římskými občany a vojáky.\footnote{Celkem 29 nápisů nese osobní jména: osm řeckých, čtyři thrácká, 21 římských a 11 bez přesného určení. Římská jména se vyskytovala samostatně na osmi nápisech a v kombinaci na 13 nápisech. V 11 případech se opakuje římské jméno Aurelios v kombinaci s řeckým či thráckým jménem, což značí že dedikanti přijali po roce 212 n. l. toto jméno, co by znak římského občanství. Nápisy věnovali vojáci v šesti případech, což představuje pouze 9,5 \letterpercent{} dedikací nesoucích nápis z této lokality.} Předměty nesoucí nápisy představují pouze pětinu dochovaných votivních předmětů, což značí že zvyk věnovat nápisy nebyl zcela běžnou součástí rituálu, ale spíše ojedinělou záležitostí. Větší zapojení thrácké populace je pravděpodobně důsledkem nárůstu gramotnosti mezi thráckou populací a proměny funerálních zvyklostí v souvislosti s vojenskou či civilní službou Thráků.

\subsection[veřejné-nápisy-15]{Veřejné nápisy}

Oproti 2. st. n. l. je pozorovatelný další nárůst počtu veřejných nápisů zhruba o 50 \letterpercent{} na 176 exemplářů. Veřejné nápisy pocházejí většinou z thráckého vnitrozemí z bezprostředního okolí {\em Via Diagonalis} a městských center ležících na této významné spojnici. Mezi největší producenty veřejných nápisů patří Augusta Traiana s 38 nápisy, Filippopolis s 37 nápisy, Serdica s 27 nápisy, Nicopolis ad Istrum s 22 nápisy, Perinthos (Hérakleia) s 9 nápisy. Naopak veřejné nápisy téměř vymizely z Byzantia, odkud pocházejí pouze dva texty, či z Maróneie, kde byl nalezen jeden veřejný nápis.\footnote{Dekrety jsou stále nejčastějším typem dokumentu s 105 nápisy, z nichž 95 představuje dekrety honorifikační. Dále sem patří sedm seznamů a 51 nápisů jiného typu, z čehož 42 jsou milníky, tři hraniční kameny a jeden nápis dokumentují stavební aktivity.}

Honorifikační nápisy jsou i nadále nejpočetnější skupinou veřejných nápisů a pocházejí především z velkých městských center té doby s existujícími samosprávními institucemi a velkým počtem obyvatelstva. Většina textů pochází z vnitrozemí: 23 nápisů pochází z Augusty Traiany, 21 nápisů z Nicopolis ad Istrum, 20 z Filippopole, 13 z Perinthu (Hérakleie) a pět z města Serdica. Nápisy jsou vydávány místní samosprávou, reprezentovanou {\em búlé} a {\em démem}, případně {\em gerúsií}, avšak zcela pod patronátem římského císaře, kterého zastupují místodržící a vysoce postavení úředníci daného města, např. {\em epimelúmenos} a {\em logistés}, lat. {\em curator}, {\em argyrotamiás}, lat. {\em questor}, či {\em thrakarchés} (Mason 1974, 25; 46). Mezi honorovanými jedinci se objevují členové {\em búlé}, {\em gymnasiarchové}, vojáci, kněží, nižší úředníci jako např. {\em grammateus}, či sportovci ({\em agonothétés}). Jak dokazují opakující se formule, na počest významných jedinců byly vztyčovány stély či jejich sochy byly umístěny na veřejných místech, což poukazuje na důležitou roli, jakou v tehdejší společnosti hrál společenský status (Van Nijf 2015, 233-243). Většina osobních jmen na veřejných nápisech byla římského původu, což může naznačovat, že vysoké funkce zastávali zejména římští občané italického původu, či místní Thrákové zcela upustili od tradice zachovávání thráckých jmen a přijali plně systém tří římských jmen.\footnote{Pokud by tomu tak bylo, ve většině případů tyto dvě kategorie nedokážeme navzájem odlišit, vzhledemk tomu, že specificky neudávají geografický svůj původ.}

Významem zůstává obsah honorifikačních nápisů podobný, nicméně každé větší město používá specifické formule a slovní obraty, které se na jiných místech nevyskytují, či se vyskytují v upravené formě.\footnote{Příkladem může být Filippopolis, které je vždy označována jako {\em lamprotaté polis}, tedy nejslavnější z měst, a Perinthos, který v textech nápisů figuruje jako {\em lampra polis} čili jako slavné město.} Z variability formulací je možné soudit, že podoba honorifikačních nápisů nebyla striktně regulována jednotnou politickou autoritou, ale že se jednalo vždy o místní interpretaci dané společesnké funkce nápisu. Je pravděpodobné, že se jednotlivé městské samosprávy navzájem inspirovaly, nicméně finální podoba nařízení byla ponechána zcela místním institucím.

Jinak tomu však bylo v případě milníků, které vykazují shodný charakter včetně formy a obsahu napříč různými částmi římského impéria. To může poukazovat jednak na snahu o zjednodušení komunikace, zvýšení srozumitelnosti sdělení i v rámci mnohonárodnostního římského vojska. Dále existence milníků dokumentuje jistou míru centralizovanosti byrokratického aparátu, který měl zřizování cest, a tedy i milníků, na starosti. Jedná se tak o jeden z nepřímých projevů narůstající společenské organizace a centrálně řízeného propracovaného systému uchovávání a šíření informací, typického pro komplexní společnosti (Johnson 1973, 3-4). Dochované milníky taktéž dokumentují existenci fungující infrastruktury a stavebních aktivit v průběhu 3. st. n. l. Především koncem 2. st. a začátkem 3. st. n. l. dochází k zintenzivnění aktivit spojených s výstavbou a zejména opravou již existujících cest, které sloužily primárně pro přesuny římské armády, ale i k obchodu a zásobování provincií (Madzharov 2009, 64). Milníky udávaly vzdálenost do nejbližšího města, které zároveň bylo i samosprávní jednotkou s povinností se starat o údržbu daného úseku cesty.\footnote{V deseti případech to byla Filippopolis, osmi případech Serdica, v sedmi Augusta Traiana, v pěti Pautália, ve dvou Perinthos (Hérakleia) a Hadrianúpolis, a v jednom Traianúpolis. Udávané vzdálenosti se pohybovaly v rozmezí dvou až 37 mil, tj. zhruba tří až 55 kilometrů.} Většina milníků pochází z trasy {\em Via Diagonalis}, která spolu s {\em Via Egnatia} spojovala evropské provincie s maloasijskými. Stavební aktivity započaté za Severovců pokračují i v průběhu celého 3. st. n. l., jak dokazuje počet dochovaných milníků.\footnote{Z doby severovské dynastie se dochovalo celkem 14 milníků, z období po roce 235 n. l. do konce století dalších 24, a do doby na přelomu 3. a 4. st. n. l. dalších pět.}

O rozsahu centrálně řízených stavebních aktivit a existující infrastruktury vypovídají dochované veřejné nápisy.\footnote{Zmínky na nápisech dokumentují stavbu agory v blíže neznámém městě v údolí řeky Strýmónu, jako je tomu v případě nápisu {\em IG Bulg} 4 2264 z moderního města Sandanski (antick8 Parthicopolis - Paroikopolis; Mitrev 2017, 106-108). Další nápis {\em I Aeg Thrace} 433 z Traiánúpole poukazuje na existující praxi vyměřování vnitřního rozdělení a uspořádání území Traiánúpole na egejském pobřeží v roce 202 n. l. Velmi podobný nápis {\em I Aeg Thrace} 447, avšak hůře dochovaný, pochází i z nedaleké Alexandrúpole. Nápis datovaný taktéž do r. 202 n. l. nasvědčuje, že za Severovců došlo k novému vyměření území a jejich vnitřní struktury u nejméně dvou měst na egejském pobřeží.} O vnějším uspořádání měst a jejich území svědčí hraniční kameny ({\em horoi}), kterých se dochovalo celkem osm, což je největší počet ze všech předcházejících období. Města těmito kameny vymezovala rozsah svého území, a to jednak pro samosprávní účely, výběr daní, ale například i pro jasnější financování infrastrukturních projektů, jako byla např. údržba cest.\footnote{Hraniční kameny se našly na hranicích území dvou měst, jako je to v případě jednoho nápisu z Odéssu a dvou nápisů z Marcianopolis, a dále na hranicích samosprávné jednotky na úrovni vesnice ({\em kómé, chórion}) či jiného osídlení neměstského typu jako v případě tří nápisů z Bendipary v blízkosti Filippopole, a jednoho nápisu patřící Eresénským, tedy obyvatelům Eresy(?) v blízkosti Maróneie. Hraniční kameny byly vydány pod hlavičkou tehdejšího císaře či místodržícího úředníka, čili se jednalo o státem podporovanou aktivitu.}

Stavební aktivita se na konci 2. st. n. l. a na počátku 3. st. n. l. nevztahovala pouze na cesty, ale v epigrafických záznamech se postupně objevují nově vzniklá {\em emporia}~ Discoduraterae, Pizos, dále sem patří i Pirentensium a Pautália (Lozanov 2015, 84-85). Tato {\em emporia} spadala pod samosprávu nejbližšího města, což byla Augusta Traiana, či Nicopolis ad Istrum. Tato lokální centra obchodu zajišťovala městským celkům a vojenským jednotkám jednak zemědělské produkty, řemeslné výrobky, ale zároveň byla umístěna v blízkosti důležitých vojenských, ale i obchodních cest a umožňovala výměnu zboží mezi městem a venkovem. Nápis {\em IG Bulg} 3,1 1690 o délce 270 řádek zmiňuje {\em emporion} Pizos a podává výčet obyvatel vesnic z okolí, kteří se podíleli na zakládání {\em emporia}.\footnote{Pizos ležel na {\em Via Diagonalis} v blízkosti přepřahacích stanic ({\em mutatio}) Ranilum, Arzus a Cillae a nedaleko opevněného osídlení vojenského charakteru v Carasuře. Odbytiště pro produkty a výrobky z {\em emporia} zajišťovala z velké části římská armáda a částečně i sama Augusta Traiana. Jedno z privilegií nového {\em emporia} bylo oproštění od dovozních cel do Augusty Traiany, což oproti místním vesnicím poskytovalo {\em emporiu} nespornou ekonomickou výhodu. {\em Emporion} ve 3. st. n. l. tak usnadňovalo výměnu zboží mezi městskými centry a venkovem. Podobně tomu tak bylo i v případě dalších {\em emporií}, z nichž se nám však na nápisech dochovalo výrazně méně detailních informací o jejich fungování.} To vzniklo sestěhováním obyvatel nejméně devíti thráckých vesnic v době vlády Septimia Severa v roce 202 n. l.\footnote{Jména sestěhovaných vesnic jsou typicky thrácká: {\em Skedabria, Stratopara, Krasalopara, Skepte, Gelúpara, Kúrpisos, Bazopara, Strúneilos a Búsipara}. Zajímavým faktem je, že jedna skupina obyvatel měla sestěhování nařízeno od místodržícího provincie a další skupina se sestěhování účastnila dobrovolně (Boyanov 2014, 185).} V textu se vyskytují pouze mužská jména a mělo se jednat o první osadníky nově vzniklého {\em emporia}, které se nacházelo v regionu Augusty Traiany. Jména 154 sestěhovaných obyvatel jsou z více než dvou třetin thrácká a poukazují na thrácký původ jak nositelů samotných, tak jejich rodičů či sourozenců. Vyšší společenský status některých z nich v rámci rurální společnosti dokládají použité funkce jako {\em toparchés} či {\em búleutés}.\footnote{Heller (2015, 266) na příkladech honorifikačních nápisů z Malé Asie dokazuje, že {\em búleutés} zaujímal spíše střední až nižší postavení v rámci hierarchie provinciálních úředníků, nicméně toto postavení bylo stále vyšší než u většiny běžné populace, a proto bylo na nápisech vyzdvihováno.}

Proměnu onomastických zvyklostí ve 3. st. n. l. v reakci nové politické uspořádání nejlépe ilustrují poměrně obsáhlé seznamy osob, kde lze velmi dobře sledovat proměňující se přístup k přijímání římských jmen v rámci vývoje širší společenskopolitické situace v římské říši. Před rokem 212 n. l. bylo užití jmen jako Flavios, Iúlios, Ulpios, Klaudios pouze záležitostí velmi úzké skupiny lidí, což zvyšovalo jejich prestiž (Parissaki 2007, 286). V dřívějších dobách bylo právo nosit římské jméno výsadou vysloužilých vojáků či vysokých úředníků jako odměna za jejich služby, avšak po roce 212 n. l. mohl toto jméno nosit každý svobodný obyvatel římské říše (Beshevliev 1970, 28-32). V roce 212 n. l. císař Caracalla ediktem známým jako {\em Constitutio Antoniniana} udělil římské občanství všem obyvatelům římské říše a spolu s ním i právo nosit rodové jméno římského císaře (Beshevliev 1970, 31-32). Z této doby se dochovalo sedm nápisů obsahujících údaje o 437 osobách a 658 osobních jmen\footnote{Nejvíce osob se nachází na již diskutovaném nápise {\em IG Bulg} 3,1 1690 o založení emporia Pizos, a to 154 osob. Dále sem patří tři nápisy se seznamy efébů z Odéssu ({\em IG Bulg} 1,2 47, 47bis, 48) a jeden z Dionýsopole ({\em IG Bulg} 1,2 14), jeden seznam věřících Dionýsova kultu z Cillae ({\em IG Bulg} 3,1 1517) a jeden seznam věřících blíže neznámého kultu z Augusty Traiany ({\em SEG} 58:679).} a seznamy vydané bezprostředně po roce 212 n. l. vykazují výrazně vyšší poměr přijatých římských jmen, především jména Aurelios, než nápisy např. z poloviny století.\footnote{Příkladem z Thrákie jsou nápisy se seznamy efébů {\em IG Bulg} 1,2 14 a 47 datované do doby krátce po r. 212 n. l., kde přes 95 \letterpercent{} osob přijalo ke svému jménu císařské jméno Aurelios. Na dalším seznamu efébů {\em IG Bulg} 1,2 47bis z roku 221 n. l. tento poměr klesl na 85 \letterpercent{} a u nápisu {\em IG Bulg} 3,1 1517 datovaného do let 241-244 n. l. klesl na 80 \letterpercent{}. Na seznamech efébů se většinou vyskytují muži s kombinovanými řeckými a římskými jmény a zcela výjimečně jména thrácká v necelých 2,4 \letterpercent{}.} Tento trend se ve velké míře objevuje i v dalších částech římské říše, jak dokazuje nedávná studie z Malé Asie (Blanco-Perez 2016, 279). Jedno z možných vysvětlení může naznačovat, že s přibývajícím časovým odstupem od hromadného udělení římského občanství společenská hodnota a prestiž v rámci komunity klesala, a tudíž se jméno Aurelios, a vše, co toto jméno představovalo, na nápisech objevovalo méně často.

Spolu s měnícími se onomastickými zvyklostmi je ve 3. st. n. l. taktéž pozorovatelná větší provázanost identity jednotlivce s institucemi a společenskou organizací římské říše, a to jak na veřejných, tak na soukromých nápisech. Politická identita, respektive přináležitost do samosprávní jednotky na úrovni města, případně samosprávy na úrovni vesnice, se stává důležitou součástí veřejných nápisů.\footnote{Výskyt kolektivního vyjádření identity s příslušností k městu se vyskytuje na soukromých i veřejných nápisech ze 3. st. n. l. celkem 55krát, s příslušností k vesnici sedmkrát.} Politická identita zcela nahrazuje etnickou příslušnost, která tak pozbývá na společenské důležitosti a tudíž i její zmiňování na nápisech.\footnote{Kolektivní termíny Thrákové a Thrákie jsou zmiňovány na 66 soukromých a veřejných nápisech pouze v souvislostí s obyvateli provincie {\em Thracia}, či jako obecné pojmenování typu gladiátora. Zmínka o konkrétních thráckých kmenech je omezena pouze na Serdicu, o níž se na 17 nápisech hovoří jako o městu kmene Serdů ({\em hé Serdón polis}).}

\subsection[shrnutí-19]{Shrnutí}

Dochované materiály ze 3. st. n. l. poukazují na výsadní roli římské administrativy a vojenské organizace, což se projevovalo i v rámci epigrafické produkce. Epigrafické záznamy dokládají nárůst stavebních aktivit a aktivit spojených s udržováním již existující infrastruktury, jako vojenských cest či táborů, ale i intenzivní rozvoj městských center, zejména ze začátku 3. st. n. l. Narůstají komplexita společnosti se projevuje i ve zvyšujícím se množství specializovaných funkcí, které se na nápisech objevují. Většina z nich je do větší či menší míry spojena s administrativním řízením provincie či s vojenskou službou.

Složení epigraficky aktivní společnosti je podobné jako ve 2. st. př. n. l., ale dochází k nárůstu výskytu osob thráckého původu, zejména na nápisech soukromé povahy. Thrákové již zcela běžně slouží v římské armádě, kde zastávají nižší a střední posty. V rámci civilní samosprávy se podílejí na vedení provincie, a to dokonce i v roli vyšších úředníků. Přítomnost Thráků je ve zvýšené míře zaznamenána u nápisů pocházejících z lokálních svatyní ve vnitrozemí, nicméně i v těchto kultech nepřesahuje zastoupení osob s thráckými jmény jednu pětinu dochovaných jmen. Z toho plyne, že i místní kulty byly otevřeny i osobám jiného než thráckého původu.

Ve 3. st. n. l. podobně jako v předcházejícím století spíše než původ hraje roli status a postavení v rámci komunity: velký důraz je kladen na dosažené postavení, zastávané funkce, získané pozice v armádě, i na afiliaci s římskou říší v podobě proměněných onomastických zvyků, které poukazují na římské občanství a dosažený status nositele.

\stopcomponent