
\environment ../env_dis
\startcomponent section-nápisy-z-6.-až-8.-st.-n.-l.
\section[nápisy-z-6.-až-8.-st.-n.-l.]{Nápisy z 6. až 8. st. n. l.}

Vytvořená HAT databáze obsahuje celkem 12 nápisů ze 6. st. n. l., které odpovídají kritériím datace, tedy nápisů s koeficientem 1. Dále databáze obsahuje i dva nápisy ze 7. až 8. st. n. l., tedy nápisy s koeficientem 0,5. Tato skupina nápisů vykazuje stejné rysy jako nápisy z 5. a 5. až 6. st. n. l., a proto je zmíním alespoň ve stručnosti.

\startDelimitedTable
 {\em Celkem}~ 12 nápisů (6. st.) + 2 nápisy (7.-8. st.)

{\em Region měst na pobřeží}~ Maximianopolis 2, Traianopolis 1, Perinthos (Herakleia) 5, Mesámbriá 1 (celkem 9 nápisů)

{\em Region měst ve vnitrozemí}~ Plotinopolis 1, Augusta Traiana 1 (celkem 2 nápisy)\postponenotes\footnote{Celkem tři nápisy nebyly nalezeny v rámci regionu známých měst, editoři korpusů udávají jejich polohu vzhledem k nejbližšímu modernímu sídlišti, či uvádí muzeum, v němž se nachází.}

{\em Celkový počet individuálních lokalit}~ 7

{\em Archeologický kontext nálezu}~ funerální 1, sekundární 2, neznámý 9

{\em Materiál}~ kámen 10 (mramor 7; jiný 3), keramika 4

{\em Dochování nosiče}~ 75 \letterpercent{} 1, 50 \letterpercent{} 2, 25 \letterpercent{} 4, nemožno určit 7

{\em Objekt}~ stéla 8, architektonický prvek 6

{\em Dekorace}~ reliéf 7, bez dekorace 7; reliéfní dekorace figurální 0 nápisů , architektonické prvky 7 nápisy (vyskytující se motiv: kříž 5, razidlo 4)

{\em Typologie nápisu}~ soukromé 8, veřejné 4, neurčitelné 2

{\em Soukromé nápisy}~ funerální 8

{\em Veřejné nápisy}~ razidlo na cihlách 4

{\em Délka}~ aritm. průměr 4,64 řádku, medián 5, max. délka 8, min. délka 2

{\em Obsah}~ latinský text 0, písmo řím. typu 1; hledané termíny (administrativní termíny 4 - 1 výskyt, epigrafické formule 3 - 1 výskyt, honorifikační 0, náboženské 2 - 1 výskyt, epiteton 0)

{\em Identita}~ křesťanská náboženská terminologie, vymizení lokálních kultů z nápisů, regionální epiteton 0, subregionální epiteton 0, kolektivní identita 0 termínů; celkem 14 osob na nápisech, 6 nápisů s jednou osobou; max. 2 osoby na nápis, aritm. průměr 0,57 osoby na nápis, medián 0,5; uzavřené komunity, bez prolínání onomastických tradic, jména pouze řecká (36 \letterpercent{}), pouze thrácká (0 \letterpercent{}), pouze římská (7 \letterpercent{}), kombinace řeckého a thráckého (0 \letterpercent{}), kombinace řeckého a římského (7 \letterpercent{}), kombinace thráckého a římského (0 \letterpercent{}), kombinovaná řecká, thrácká a římská jména (0 \letterpercent{}), jména nejistého původu (0 \letterpercent{}), beze jména (50 \letterpercent{}); geografická jména z oblasti Thrákie 0, mimo Thrákii 2;
\stopDelimitedTable
\flushnotes


Nápisy pocházejí převážně pobřežních oblastí, pouze dva nápisy pocházejí z vnitrozemí. Produkční centra jsou prakticky stejná jako v 5. st. př. n. l., jak je patrné z \in{Mapy}[Apendix2:::6.11a] v \in{Apendixu}[Apendix2:::Apendix2]. Objekty nesoucí nápis jsou převážně z mramoru s vyobrazením křesťanského kříže a křesťanská tematika\index{křesťanství} se odráží i v textu nápisů, které jsou převážně funerálního zaměření. Zbylé čtyři nápisy označují výrobce keramických cihel, nesoucí vyobrazení kříže a pořečtěnou verzi termínu {\em kúbikúlarios}, tedy eunuch sloužící v předpokojích římského a později byzantského císaře. Složení epigraficky aktivní společnosti je dle původu osobních jmen z třetiny řecké, s odkazy na místní komunity jako je Filippopolis či na obyvatele řeckých měst z Malé Asie z oblasti Galatie a Ankýry. Tato skupina nápisů tedy nijak nevybočuje z trendu epigrafické produkce nastavené již na konci 4. st. n. l., a zejména pak v průběhu 5. st. n. l., kde se křesťanství, a zejména jeho projevy v rámci funerálního ritu, staly jediným přeživším motivem epigrafické produkce v Thrákii.

\stopcomponent