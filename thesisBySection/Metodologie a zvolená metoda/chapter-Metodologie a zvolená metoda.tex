
\environment ../env_dis
\startcomponent chapter-Metodologie a zvolená metoda
\chapter{Metodologie a zvolená metoda}
Základním cílem této práce je studovat společnost antické Thrákie v časovém průřezu na základě studia souboru dochovaných nápisů. V rámci analýzy epigrafického materiálu sleduji nejen kvantitativní výskyt nápisů v místě a čase, ale zároveň se zaměřuji na kvalitativní analýzu obsahu nápisů a jejich reflexi uspořádání tehdejší společnosti na základě výskytu charakteristických prvků. Jako součást kvalitativní analýzy sleduji zejména diverzifikaci politického uspořádání a společenských rolí, stupeň provázanosti obyvatelstva a společenské organizace, míru fluktuace obyvatelstva různého kulturního pozadí, proměny publikačních a onomastických zvyklostí a v neposlední řadě šíření symbolických hodnot, tradičně považovaných za konzervativní prvek společnosti. Na základě dostupných dat zařazuji sledované prvky do kontextu společnosti, v níž se vyskytovaly s cílem zhodnotit, zda se jedná o přímý či nepřímý důsledek kontaktů s řeckým světem a řeckou kulturou.

\stopcomponent