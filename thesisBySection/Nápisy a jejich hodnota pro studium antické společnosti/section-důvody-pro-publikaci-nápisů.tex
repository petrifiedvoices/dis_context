
\environment ../env_dis
\startcomponent section-důvody-pro-publikaci-nápisů
\section[důvody-pro-publikaci-nápisů]{Důvody pro publikaci nápisů}

Zvyk zhotovovat nápisy do kamene, známý též jako též tzv. {\em epigraphic habit}, nemá jednu univerzální příčinu, ale je kombinací mnoha faktorů, které měly za následek, že některé komunity začaly nápisy produkovat a jiné nikoliv (MacMullen 1982, 233; Bodel 2001, 12-13).

Nápis je v prvé řadě materializovanou zprávou určenou dalším členům společnosti, v níž nápis vznikl. Psaní na nosič trvalého charakteru, jako je kámen, kov, keramika atp., je chápáno jako předem promyšlený akt, kterým člověk sleduje určitý záměr a předává konkrétní zprávu. Tím, že člověk napíše určité sdělení na trvanlivý materiál a umožní tak, aby si ho kdokoliv v~současnosti, ale i v budoucnosti mohl přečíst, přisuzuje obsahu sdělení velkou váhu a nepřímo tak poukazuje na systém hodnot, které byly v tehdejší společnosti důležité. Úsilí jedince a mnohdy i nemalé finanční prostředky nutné k vytvoření nápisu ještě více zdůrazňují důležitost sdělení v rámci tehdejší společnosti a potvrzují nezastupitelnou roli nápisů při studiu vývoje hodnotových trendů tehdejší společnosti (McLean 2008, 13-14).

Motivace pro zhotovení nápisu jsou různé a nejde je přisuzovat pouze jediné příčině. Nápisy bývají publikovány v důležitých životních situacích a slouží jako reflexe měnících se podmínek. V případě soukromých nápisů, tj. publikovaných jménem jednotlivce či skupiny, to může jednak být reakce na nemoc či smrt blízké osoby, reakce na nelehkou životní situaci a obrácení se o pomoc k božstvu, signalizace vlastnictví předmětu v případě, že hrozí jeho krádež či pozbytí, či v neposlední řadě v době změny společenského postavení, jako např. při udělení občanství či získání svobody (Cooley 2012, 52-54). V rámci komunikační strategie vůči cílové skupině čtenářů mohl zhotovitel zdůraznit určité charakteristiky na úkor jiných či se záměrně stylizovat do určité zamýšlené role.\footnote{Například někteří badatelé se pokouší vysvětlit zvýšenou epigrafickou produkce na konci 2. a na začátku 3. st. n. l. jako motivovanou finančními zájmy příbuzenstva, spojenými s dědictvím a zároveň jako snahu o navýšení společenské prestiže pomocí poukázání na dosažené postavení (Meyer 1990, 78, 95; Cooley 2012, 54).} V případě veřejných nápisů, tj. nápisů publikovaných politickou autoritou, se může jednat o reakci na hrozící či proběhlou změnu společenského uspořádání či o veřejnou signalizaci moci a autority v rámci komunity vedoucí k upevnění mocenské pozice.

Důvody a motivace pro publikování nápisů byly na první pohled velmi odlišného charakteru, podmíněné společenskou a kulturní funkcí, kterou měl nápis plnit. Na počátku epigrafické produkce stály pravděpodobně zájmy jednotlivců, avšak za jejím rozšířením v míře, s jakou se setkáváme ve starověkém světě, zásadní roli hrál postupný vývoj společensko-politické organizace, též známý jako společenská komplexita nebo-li provázanost.

\stopcomponent