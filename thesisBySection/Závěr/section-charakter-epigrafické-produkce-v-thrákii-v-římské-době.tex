
\environment ../env_dis
\startcomponent section-charakter-epigrafické-produkce-v-thrákii-v-římské-době
\section[charakter-epigrafické-produkce-v-thrákii-v-římské-době]{Charakter epigrafické produkce v Thrákii v římské době}

Epigrafická produkce pocházející z období římské nadvlády vykazuje oproti předcházejícímu období prudký nárůst, zejména v průběhu 2. a 3. st. n. l. Navíc se zcela stírají rozdíly mezi odlišným užitím nápisů na pobřeží a ve vnitrozemí, nápisy tak dostávají velmi podobný charakter a epigrafická produkce se z pobřeží přesouvá do okolí center městského charakteru ve vnitrozemí. Politický a kulturní vliv řeckých měst je však natolik oslaben, že tuto proměnu epigrafické kultury není možné spojovat s civilizační tendencí řecké kultury, známou pod pojmem hellénizace, či tzv. římská hellénizace.

Ze studia dochovaných nápisů vyplývá, že k hlavnímu rozvoji epigrafické produkce došlo nikoliv v souvislosti s řeckou přítomností v Thrákii, ale v přímé souvislosti s nárůstem společenské organizace a rozvinutím potřebné infrastruktury v době římské. V předřímské době nápisy pocházely převážně z řeckých měst na pobřeží, případně z ekonomických a kulturních center ve vnitrozemí. V době římské epigrafická produkce pocházela z přímého okolí městských center, která se v této době začala objevovat ve zvýšené míře i ve vnitrozemí, a dále v okolí římských silnic, které sloužily pro přesuny vojsk i civilního obyvatelstva a výraznou měrou přispěly k propojení vzdálených regionů a zintenzivnění kulturních kontaktů. Urbanizace thráckého vnitrozemí měla na projevy epigrafické produkce přímý vliv, stejně tak jako centralizace politické moci, jevy obecně spojené s růstem společenské komplexity. V této době zároveň došlo k rozdělení práce, zintenzivnění produkce a zajištění potřebné infrastruktury nutné k produkci nápisů ve velkém měřítku.

Jedním z hlavních důvodů nárůstu epigrafické produkce v římské době bylo zvýšené zapojení místních obyvatel do služeb římské armády, pozorovatelné již od 1. st. n. l. Veteráni, kteří se po dlouhé vojenské službě vraceli do Thrákie, s sebou přinesli nově získané kulturní zvyklosti související s jejich službou v armádě a pobytem na územích se zcela odlišnou kulturou. Je více než pravděpodobné, že za dobu služby vojáci získali alespoň základní stupeň gramotnosti a seznámili se se zvykem publikovat nápisy, což se po konci služby projevilo i ve změně přístupu k zhotovování nápisů. Na nápisech v římské době se taktéž objevuje větší zastoupení thráckých jmen než v době předřímské, což je přímý důsledek většího zapojení Thráků do epigrafické produkce. S větším zapojením Thráků souvisí i nárůst počtu dedikací věnovaných místním božstvům, zejména ve 2. a 3. st. n. l. Tento jev by mohl představovat nárůst uvědomění si thrácké identity, nicméně taktéž se může jednat o pouhý epigrafický záznam již existujícího trendu, který se podařilo zachytit právě díky většímu zapojení thrácké populace na epigrafické produkci. Tím, že se Thrákové více zapojovali do chodu římské říše, zejména službou v armádě a civilní správě, se jim dostalo náležitého vzdělání a tím více se následně mohli zapojovat i do produkce nápisů, čemuž odpovídá i větší zastoupení thráckého prvku na dochovaných nápisech.

\subsection[proměny-identity-a-vliv-řecké-kultury]{Proměny identity a vliv řecké kultury}

Jeden ze základních rysů hellénizace v podobě proměny původní identity osob a její postupné nahrazení řeckou identitou není na nápisech z Thrákie pozorovatelný. Naopak, po celou dobu dochází ke kladení důrazu na lokální identitu a kontinuitu tradičních hodnot. Zejména v římské době se osoby více ztotožňují s politickou autoritou na úrovni měst a vesnických samospráv, státních i místních institucí a v neposlední řadě i vojska.

Identifikace s thráckým etnikem na nápisech v předřímské době téměř neexistuje a v římské době se objevuje v souvislosti se systémem verbování a tvoření vojenských jednotek římské armády, ale soudě dle nápisů, v životě civilního obyvatelstva nehraje etnicita takřka žádnou roli{\bf .} Naopak identifikace a snaha zařazení osob do nejbližších komunit se vyskytuje jako jednotící prvek po celou dobu, a to zejména na nápisech soukromého charakteru, kde se setkáváme s uváděním biologického původu a odkazů na předky.

Osobní jména v předřímské době nevykazují velkou míru prolínání onomastických tradic a jednotlivé komunity spíše dodržují konzervativní charakter předávání osobní jmen z generace na generaci. Nicméně Thrákové přecházejí na systém identifikace s uváděním jmen rodičů, jak je zvykem v řecky mluvícím prostředí. V římské době dochází k určité proměně onomastických zvyklostí, kdy jsou k tradičním jménům řeckého či thráckého původu přidávána jména římská. Přijetí tradičního římského onomastického systému poukazuje na dosažené společenského postavení nositele, jakožto symbolu získání římského občanství, jehož vlastnictví s sebou nese společenské výhody, ale nedokazuje proměnu identity. Naopak fakt, že se většinou římská jména vyskytují v kombinaci se jmény thráckého či řeckého původu spíše poukazuje na adaptaci obyvatel na nové společenskopolitické podmínky za současného uchování původní identity.

Řečtina je i nadále využívána jako hlavní jazyk nápisů, ale spíše než o projev hellénizace společnosti se jedná o projev převzetí fungujícího systému komunikace a zaznamenávání informací bez nutnosti změny jeho formy. Tomu nasvědčuje i velká konzervativnost epigrafického jazyka a obsahu nápisů, kde tradiční epigrafické formule zůstávají neměnné i po několik století. Nicméně proměňující se dekorace a provedení nosiče nápisu naznačují změnu vkusu a poptávky zhotovitelů. V římské době se objevuje více variant dekorace nosičů nápisů, které nicméně vycházejí z předřímských vzorů.

Nápisy z římské doby vykazují několik společných rysů, které se nevyskytovaly v předcházejícím období a které jsou pozorovatelné i mimo Thrákii na území ostatních římských provincií. Společným rysem funerálních nápisů z římské doby je tendence spolu s identifikací zemřelého uvádět i členy rodiny, či přátele, kteří nápis nechali zhotovit. V předřímské době je na nápisech uváděn pouze zemřelý či rodiče, případně partner. V římské době se tento okruh lidí značně rozšiřuje i na sourozence, vnuky a přátele, případně kolegy z armády. Je více než pravděpodobné, že se tak dělo z důvodů zajištění dědických práv a z povinností dědiců vycházejících z tehdy platných právních norem. Dalším prvkem, pozorovatelným v době římské na celém území Thrákie, ale i mimo něj, je snaha jedince prezentovat na nápisech dosažené společenské postavení a zapojení do institucionálních struktur tehdejší společnosti. Děje se tak vědomou prezentací na nápisech ve formě konkrétní podoby osobního jména, dále ve formě uvádění dosažených životních úspěchů a společenského postavení, ale i ve formě identifikace s politickou či náboženskou komunitou. Podobně se objevuje ve velké míře na nápisech z této doby i zvyk udávat věk zemřelého, což je zvyk typický pro nápisy římské doby. Tyto trendy nejsou typické pouze pro nápisy z Thrákie, ale vyskytují se téměř na celém území římské říše a naznačují ovlivnění podoby epigrafické produkce tehdy platnými společenskými normami a společenskou organizací, kterou představovala centrální autorita římské říše.

\stopcomponent