\mainlanguage[cz]
\environment env_dis

\starttext
\startchapter[title={Použité zkratky a bibliografie}, reference={bibliografie}, marking={Seznam použitých zkratek a bibliografie}]


\setuplocalinterlinespace[line=2.8ex] %spacing 1
%\setuplayout[double]

\setuplayout[location=doublesided]
\setuppagenumbering[alternative=doublesided]
\unprotect
\usemodule[database]

\section{Seznam použitých zkratek}
\setuplocalinterlinespace[line=1.5ex]
\startlines
{\bf \em AE}: {\em L'Année épigraphique}. ed. Cagnat, R. Paris 1888-1992. ed. Corbier, M. Paris 1992- .

{\bf \em CIL}: {\em Corpus Inscriptionum Latinarum, consilio et auctoritate Academiae litterarum regiae Borussicae editum}. 1863 - , Berlin. \goto{http://cil.bbaw.de/cil_en/dateien/datenbank_eng.php}[url(http://cil.bbaw.de/cil_en/dateien/datenbank_eng.php)] 

{\bf \em EDH}: {\em Epigraphic Database Heidelberg}, databáze řeckých a 
latinských nápisů, \goto{http://edh-www.adw.uni-heidelberg.de/home/}[url(http://edh-www.adw.uni-heidelberg.de/home/)]

{\bf \em GIS}: geografický informační systém

{\bf \em Gyuzelev 2002}: Gyuzelev, M. (2002). Ancient Funerary Monuments at Sozopol Archaeological Museum. {\em Izvestija na narodnija muzej Burgas} 4, 119--129.

{\bf \em Gyuzelev 2005}: Gyuzelev, M. (2005). Ancient Funerary Monuments Found at the Necropolis of Kalfata (Sozopol) in the Year of 2002. In Stoyanov, T. (Ed.). {\em Heros Hephaistos: Studia in honorem Liubae Ognenova-Marinova}, 131--138. Sofia: Faber Publisher.

{\bf \em Gyuzelev 2013}: Gyuzelev, M. (2013). Tituli sepulcrales in necropoli antiqua locis dictis Kalfata and Budzaka prope urbem Sozopolim reperti in effosionibus annorum MMIV et MMV. {\em Il Mar Nero} 7, 115--148.

{\bf \em HAT}: {\em Hellenization of Ancient Thrace}, databáze vytvořená pro účely této disertační práce, dostupná na adrese \goto{https://github.com/petrajanouchova/hat_project}[url(https://github.com/petrajanouchova/hat_project)] 

{\bf \em I Aeg Thrace}: Loukopoulou, L. D., Parissaki, M. G., Psoma, S., Zournatzi, A. (2005). {\em Inscriptiones antiquae partis Thraciae quae ad ora maris Aegaei site est: praefecture Xanthes, Rhodopes et Hebri}. Athens: Diffusion de Boccard.

{\bf \em IG}: {\em Inscriptiones Graecae}. 1903- , Berlin. 

{\bf \em IG Bulg}: Mihailov, G. (1956, 1958, 1961, 1964, 1966, 1970, 1997). {\em Inscriptiones graecae in Bulgaria repertae}, 5 vols. Sofia: Academiae Litterarum Bulgaricae.

{\bf \em IK Byzantion}: Lajtar, A. (2000). {\em Die Inschriften von Byzantion}. Bonn: Rudolf Habelt Verlag.

{\bf \em IK Sestos}: Krauss, J. (1980). {\em Die Inschriften von Sestos und der Thrakischen Chersonesos}. Bonn: Rudolf Habelt Verlag.

{\bf \em LSJ}: Liddell, H. G., Scott, R., Jones, H. S., McKenzie, R. (1996). {\em A Greek-English Lexicon}. Oxford: Clarendon Press.

{\bf \em Manov 2008}: Manov, M. (2008). {\em Selishnijat Zhivot v dolinata na Sredna Struma spored antichnite epigrafski pametnitsi ot IV/III v. pr. Ch. - III v. sl. Ch.} Sofia: BAN - NAIM.

{\bf \em QGIS}: program {\em QuantumGIS} nebo také {\em QGIS} sloužící k vytváření, zobrazování a analyzování geografických dat, \goto{http://www.qgis.org/}[url(http://www.qgis.org/)]

{\bf \em Perinthos-Herakleia}: Sayar, M. H. (1998). {\em Perinthos-Herakleia (Marmara Ereğlisi) und Umgebung. Geschichte, Testimonien, griechische und lateinische Inschriften}. Vienna: Österreichische Akademie der Wissenschaften.

{\bf \em PHI}: {\em Packard Humanities Institute}, databáze řeckých nápisů {\em Searchable Greek Inscriptions},  \goto{http://inscriptions.packhum.org/}[url(http://inscriptions.packhum.org/)]

{\bf \em R}: statistický program {\em R}, \goto{https://www.r-project.org/about.html}[url(https://www.r-project.org/about.html)]

{\bf \em SEG}: {\em Supplementum Epigraphicum Graecum}. Vols. 1-11, ed. Jacob E. Hondius, Leiden 1923-1954. Vols. 12-25, ed. Arthur G. Woodhead. Leiden 1955-1971. Vols. 26-41, eds. Henry W. Pleket a Ronald S. Stroud. Amsterdam 1979-1994. Vols. 42-44, eds. Henry W. Pleket, Ronald S. Stroud a Johan H. M. Strubbe. Amsterdam 1995-1997. Vols. 45-49, eds. Henry W. Pleket, Ronald S. Stroud, Angelos Chaniotis a Johan H. M. Strubbe. Amsterdam 1998-2002. Vol. 50, eds. Angelos Chaniotis, Ronald S. Stroud, Johan H. M. Strubbe. Amsterdam. 2003. Vols. 51-58, eds. Angelos Chaniotis, Ronald S. Stroud, Nicolas Papazarkadas, and R. A. Tybout. Leiden 2005-2012. Vols. 59- , eds. Angelos Chaniotis, Thomas Corsten, Nicolas Papazarkadas, and R. A. Tybout. Leiden. 2013- . \goto{http://referenceworks.brillonline.com/browse/supplementum-epigraphicum-graecum}[url(http://referenceworks.brillonline.com/browse/supplementum-epigraphicum-graecum)]

{\bf \em TRAP}: {\em The Tundzha Regional Archaeological Project}, archeologický projekt vedený S. A. Rossem a A. Sobotkovou, Macquarie University, Sydney, Australia, \goto{www.tundzha.org}[url(www.tundzha.org)]

{\bf \em Velkov 1991}: Velkov, V. (1991). Nadpisi ot Kabile. In Velkov, V. (Ed.), {\em Kabile} 2, 7--53. Sofia: Balgarska Akademia na naukite.

{\bf \em Velkov 2005}: Velkov, V. (2005). Inscriptions antiques de Messambria (1964-1984). In  Karayotov, I. (Ed.), {\em Nessebre III}, 159--193. Burgas: Spring Ltd.

\stoplines

\section{Seznam zkratek antických autorů a děl}

%\setuplocalinterlinespace[line=1.5ex]
\startlines
{\bf \em Aischinés  De falsa legatione}: Aischinés, {\em O falešném vyslanectví}

{\bf \em Arist.}: Aristotelés

{\bf \em Aristoph.  Ach.}: Aristofanés, {\em Acharňané}

{\bf \em Arr.  Anab.}: Arriános, {\em Anabasis}

{\bf \em Arr.  Peripl. Ponti Euxini}: Arriános, {\em Periplous Ponti Euxini}

{\bf \em Dem.  Contra Timocratem (23)}: Démosthenés, {\em Proti Tímokratovi}

{\bf \em Dem.  De Chersoneso (8)}: Démosthenés, {\em O záležitostech na Chersonésu}

{\bf \em Dem.  Philippica quarta (10)}: Démosthenés, {\em Čtvrtá řeč proti Filippovi}

{\bf \em Dem.  De Corona (18)}: Démosthenés, {\em O věnci}

{\bf \em Dio. Cass.}: Dión Cassius

{\bf \em D. S.}: Diodóros Sicilský

{\bf \em Eutrop.  Breviarium}: Eutropius, {\em Breviarium}

{\bf \em Hom.  Il.}: Homér, {\em Ílias}

{\bf \em Hdt.}: Hérodotos

{\bf \em Klaudios Ptolemaios,  Geogr.}: Klaudios Ptolemaios, {\em Geographica}

{\bf \em Plat.  Resp.}: Platón, {\em Ústava}

{\bf \em Plin.  H. N.}: Plinius Starší, {\em Historia Naturalis}

{\bf \em Plut.  Kim.}: Plútarchos, {\em Kimón}

{\bf \em Pomp. Mela}: Pomponius Mela

{\bf \em Steph. Byz.}: Stefanos Byzantský, {\em Ethica}

{\bf \em Strabo}: Strabón

{\bf \em Suetonius  Aug.,  Tib.}: Suetonius, {\em Augustus}, {\em Tiberius}

{\bf \em Tac.  Ann.}: Tacitus, {\em Annales}

{\bf \em T. Liv.}: Titus Livius

{\bf \em Thuc.}: Thúkýdidés

{\bf \em Xen.  Anab.}: Xenofón, {\em Anabasis}

\stoplines

\section{Primární antické literární prameny}
* Jméno antického autora je uvedeno v podobě uvedené v citované publikaci 
\crlf

\placelistofpublications % aka \placebtxrendering
[ancient] % rendering defined above
[method=dataset] % i.e. all entries


\section{Primární epigrafické prameny}
\placelistofpublications % aka \placebtxrendering
[epigraphic] % rendering defined above
[method=dataset] % i.e. all entries

\section{Sekundární prameny}
\placelistofpublications % aka \placebtxrendering
[secondary] % rendering defined above
[method=dataset] % i.e. all entries

\stopchapter
\stoptext


