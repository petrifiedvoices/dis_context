\environment env_dis
\setupfillinrules[width=fit]%

%\setuplayout[location=singlesided]
%\setuppagenumbering[alternative=singlesided]

\startcomponent titlepage

\startstandardmakeup[doublesided=no, page=yes, align=middle]
\tfc\bf {Univerzita Karlova} \blank[2*big]
\tfc\bf {Filozofická fakulta} \blank[3*big]
\vfill
\tfd\bf {Disertační práce} \blank[2*big] \blank[3*big]
\vfill
\leftaligned{\tfb \em Vedoucí práce: PhDr. Jan Souček, CSc.}
\vfill
{\tfb \em \currentdate[year]} \hfill  {\tfb \em Mgr. Petra Janouchová} 
\blank \blank 
\stopstandardmakeup
\startstandardmakeup[doublesided=no, page=yes, align=middle]
\tfc\bf {Univerzita Karlova} \blank[2*big]
\tfc\bf {Filozofická fakulta} 
\vfill
\tfb Ústav řeckých a latinských studií \blank
\tfb studijní program: Historické vědy \blank
\tfb studijní obor: Dějiny antického starověku \blank
\vfill 
\tfd\bf {Disertační práce}
\vfill 
\tfb {\em Hellénizace antické Thrákie ve světle epigrafických nálezů} \blank[2*big]
\tfb {\em Hellenisation of Ancient Thrace based on epigraphical evidence}
\vfill
\leftaligned{\tfb \em Vedoucí práce: PhDr. Jan Souček, CSc.}
\vfill
{\tfb \em \currentdate[year]} \hfill {\tfb \em Mgr. Petra Janouchová} 
\vfill
\stopstandardmakeup
\startstandardmakeup[doublesided=no, page=yes, align=right]
\vfill
{\it Prohlášení:}
\crlf
\crlf
{\tf Prohlašuji, že jsem disertační práci napsal/a samostatně s využitím pouze uvedených a řádně citovaných pramenů a literatury a že práce nebyla využita v rámci jiného vysokoškolského studia či k získání jiného nebo stejného titulu.}
\crlf
\fillinline[width=5cm]{\tf V Praze, dne } 
\blank
\hfill \rightaligned {\tf Mgr. Petra Janouchová}  
\vfill
\stopstandardmakeup
\page
\setuppagenumbering[state=stop]
\noindentation 
\subject{\it Poděkování:} \crlf
Ráda bych na tomto místě poděkovala množství lidí, bez nichž by tato práce nikdy nemohla vzniknout. V první řadě je to má rodina, zejména moje maminka Ivanka a můj partner Jan, kteří mě po celou dobu studií bezvýhradně podporovali jak emocionálně, tak materiálně. Jejich bezbřehá trpělivost umožnila, abych se věnovala disertačnímu projektu po několik dlouhých let, a to i na úkor osobního života a volného času.

Mé díky dále zcela přirozeně patří školiteli Dr. Janu Součkovi, který mě na dlouhé cestě vedoucí k sepsání této práce trpělivě usměrňoval a opravoval mé omyly, ale zároveň mi dával dostatek prostoru na poučení se z chyb a hledání vlastní cesty. Vděčné díky patří i Adéle Sobotkové a Shawnu Rossovi z Macquarie University v Sydney, kteří mě v roce 2007 poprvé vzali s sebou na výzkum do Bulharska a ukázali možnosti, jaké výzkum v této oblasti antického světa nabízí. V průběhu let mi byli mnohou inspirací a nedovolili mi ustrnout na jednom místě a podporovali můj rozvoj, ať již v oblasti výzkumu, akademické kariéry, tak i v osobním životě. Dále můj dík patří i Barboře Weissové, s níž jsem měla tu čest spolupracovat na několika archeologických projektech v Bulharsku, konzultovat detailně strukturu své práce a sdílet slasti a strasti doktorandského života. Jsem velmi vděčná i Brianu Ballsun-Stantonovi z Macquarie University za jeho pomoc při řešení mnoha technologických výzev, které s sebou tato práce nesla. V neposlední řadě chci poděkovat kolegům spoludoktorandům z Ústavu řeckých a latinských studií za jejich komentáře k mé práci a k tolik potřebné psychické podpoře, jmenovitě zejména Pavlu Nývltovi a Editě Wolf, kteří si, i přes svou vytíženost, našli čas připomínkovat části mé práce. 

Zhotovení databáze nápisů, která je stěžejním základem disertační práce, bylo v letech 2013 a 2014 finančně podpořeno Grantovou agenturou Univerzity Karlovy (číslo grantu GAUK 546813/2013, Řecko-Thrácká společnost ve světle epigrafických nálezů). Taktéž bych ráda poděkovala svým spolupracovníkům Markétě Kobierské, Janu Ctiborovi a Barboře Weissové za neocenitelnou pomoc při řešení tohoto grantového projektu. Dík patří i všem těm, s nimiž jsem měla možnost svou práci konzultovat v průběhu konferencí, terénních prací, zahraničních stáží a studijních pobytů.

\page

\noindentation 
\subject{\it Abstrakt:} \crlf
Z území antické Thrákie z oblasti jihovýchodního Balkánu pochází více než 4600 převážně řecky psaných nápisů. Tyto nápisy poskytují jedinečný zdroj demografických a sociologických informací o tehdejší populaci, umožňující hodnotit případnou proměnu vzorců chování v reakci na mezikulturní kontakt a vývoj společenské organizace. Řecky psané nápisy bývají považovány za jeden ze základních projevů hellénizace obyvatelstva antické Thrákie, tedy postupného a nevratného procesu adopce řecké kultury a identity. V této práci hodnotím na základě časoprostorové analýzy dochovaných nápisů relevanci hellénizace jako výchozího interpretačního rámce pro studium antické společnosti. Zároveň s tím uplatňuji alternativní přístup, který respektuje jednak specifika epigrafického materiálu, ale i poznatky současného bádání v oblasti mezikulturního kontaktu. Tento metodologický přístup umožňuje podrobně hodnotit produkci nápisů nejen v průřezu staletími, ale i navzájem srovnávat jednotlivé regiony a zapojení tamní populace. Z časoprostorové analýzy nápisů je zjevné, že rozvoj epigrafické produkce v Thrákii nemůže být spojován pouze s kulturním a politickým vlivem řeckých komunit, ale z velké části jde o jev úzce spojený s narůstající komplexitou politické organizace společnosti a s tím souvisejícími proměnami vzorců chování tehdejší populace. Tento jev je patrný zejména v době římské, kdy dochází k významnému rozvoji epigrafické produkce na celém území Thrákie, nikoliv pouze v okolí původně řeckých kolonií, a ve zvýšené míře i k zapojení thrácké populace do procesu publikace nápisů. Použitá metodologie je do velké míry inovativní kombinací řady moderních přístupů z příbuzných disciplín, za zachování tradičních principů epigrafické práce. Využití digitální technologie umožňuje studovat nápisy z nové perspektivy a umístit je do regionálního kontextu, což bylo dříve jen velmi obtížné. 
\blank

\noindentation 
\subject{\it Abstract:} \crlf
More than 4600 inscriptions in the Greek language come from Thrace, the area located in the Southeastern Balkan Peninsula. These inscriptions provide socio-demographic data, allowing the study of changing behavioural patterns in reaction to cross-cultural interactions. Traditionally, one of the essential indications of the influence of the Greek culture on the population of ancient Thrace was the practice of commissioning inscriptions in the Greek language. By using quantitative and systematic analysis, the inscriptions can be studied from a new perspective that places them into broader regional context. I use this methodology to assess the concept of Hellenization as one of the possible interpretative frameworks for the study of ancient society. Using a spatiotemporal analysis of inscriptions, this research shows that epigraphic production cannot be solely linked with the cultural and political influence of Greek speaking communities. However, the phenomenon of epigraphic production is closely connected to the growth of social complexity and consequent changes in the behavioural patterns of the population. The growth in social complexity is followed by an increase of epigraphic production of public and private character alike; while at the time of socio-economic crisis and political unrest, the production of inscriptions significantly drops. The sudden change in the character of epigraphic production is obvious in the Roman period, where the production substantially intensifies as a result of growing role of sociopolitical organization. Moreover, the Thracian population becomes more involved in the whole process of commissioning inscriptions as a result of their participation in the civic and military service. The spatiotemporal analysis of inscriptions allows the discussion of the societal function of the epigraphic production over time, places them into broader regional context, and evaluates the degree of involvement of the population in the practice of commissioning of inscriptions. This research combines the specifics of epigraphic evidence and the current scholarship on crosscultural contact. This innovative methodological approach combines a range of modern theoretical concepts from related disciplines, such as archaeology and anthropology, while maintaining the epigraphic discipline best practices. 
\blank

\noindentation 
\subject{\it Klíčová slova:} \crlf
nápisy; epigrafika; epigrafická produkce; Thrákie; hellénizace; kulturní kontakt; změny společnosti; komplexní společnost; digital humanities
\blank

\noindentation 
\subject{\it Keywords:} \crlf
inscriptions; epigraphy; epigraphic production; Thrace; hellenization; cultural contact; social change; complex society; digital humanities


\stopcomponent