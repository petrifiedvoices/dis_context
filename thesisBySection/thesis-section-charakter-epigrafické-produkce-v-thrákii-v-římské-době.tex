
\section[charakter-epigrafické-produkce-v-thrákii-v-římské-době]{Charakter epigrafické produkce v Thrákii v římské době}

Epigrafická produkce pocházející z období římské nadvlády vykazuje oproti předcházejícímu období prudký nárůst, zejména v průběhu 2. a 3. st. n. l. Navíc se zcela stírají rozdíly mezi odlišným užitím nápisů na pobřeží a ve vnitrozemí, nápisy tak dostávají velmi podobný charakter a epigrafická produkce se z pobřeží přesouvá do okolí center městského charakteru ve vnitrozemí. Politický a kulturní vliv řeckých měst je však natolik oslaben, že tuto proměnu epigrafické kultury není možné spojovat s civilizační tendencí řecké kultury, známou pod pojmem hellénizace, či tzv. římská hellénizace.

Ze studia dochovaných nápisů vyplývá, že k hlavnímu rozvoji epigrafické produkce došlo nikoliv v souvislosti s řeckou přítomností v Thrákii, ale v přímé souvislosti s nárůstem společenské organizace a rozvinutím potřebné infrastruktury v době římské. V předřímské době nápisy pocházely převážně z řeckých měst na pobřeží, případně z ekonomických a kulturních center ve vnitrozemí. V době římské epigrafická produkce pocházela z přímého okolí městských center, která se v této době začala objevovat ve zvýšené míře i ve vnitrozemí, a dále v okolí římských silnic, které sloužily pro přesuny vojsk i civilního obyvatelstva a výraznou měrou přispěly k propojení vzdálených regionů a zintenzivnění kulturních kontaktů. Urbanizace thráckého vnitrozemí měla na projevy epigrafické produkce přímý vliv, stejně tak jako centralizace politické moci, jevy obecně spojené s růstem společenské komplexity. V této době zároveň došlo k rozdělení práce, zintenzivnění produkce a zajištění potřebné infrastruktury nutné k produkci nápisů ve velkém měřítku.

Jedním z hlavních důvodů nárůstu epigrafické produkce v římské době bylo zvýšené zapojení místních obyvatel do služeb římské armády, pozorovatelné již od 1. st. n. l. Veteráni, kteří se po dlouhé vojenské službě vraceli do Thrákie, s sebou přinesli nově získané kulturní zvyklosti související s jejich službou v armádě a pobytem na územích se zcela odlišnou kulturou. Je více než pravděpodobné, že za dobu služby vojáci získali alespoň základní stupeň gramotnosti a seznámili se se zvykem publikovat nápisy, což se po konci služby projevilo i ve změně přístupu k zhotovování nápisů. Na nápisech v římské době se taktéž objevuje větší zastoupení thráckých jmen než v době předřímské, což je přímý důsledek většího zapojení Thráků do epigrafické produkce. S větším zapojením Thráků souvisí i nárůst počtu dedikací věnovaných místním božstvům, zejména ve 2. a 3. st. n. l. Tento jev by mohl představovat nárůst uvědomění si thrácké identity, nicméně taktéž se může jednat o pouhý epigrafický záznam již existujícího trendu, který se podařilo zachytit právě díky většímu zapojení thrácké populace na epigrafické produkci. Tím, že se Thrákové více zapojovali do chodu římské říše, zejména službou v armádě a civilní správě, se jim dostalo náležitého vzdělání a tím více se následně mohli zapojovat i do produkce nápisů, čemuž odpovídá i větší zastoupení thráckého prvku na dochovaných nápisech.

