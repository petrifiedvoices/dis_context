
\section[charakteristika-epigrafické-produkce-v-1.-st.-n.-l.-až-2.-st.-n.-l.]{Charakteristika epigrafické produkce v 1. st. n. l. až 2. st. n. l.}

Nápisy datované do 1. a 2. st. n. l. pocházejí i nadále z pobřežních oblastí, s hlavním produkčním centrem v Perinthu. Soukromé nápisy funerální typu i nadále převažují, prodlužuje se průměrná délka nápisu a stejně tak i celkový poměr přítomnosti římských jmen v epigraficky aktivních komunitách. Osoby nesoucí thrácká jména se dostávají do izolace, nedochází k jejich kombinaci s řeckými či římskými jmény v rámci jednoho nápisu.

\placetable[none]{}
\starttable[|l|]
\HL
\NC {\em Celkem:} 94 nápisů

{\em Region měst na pobřeží:} Abdéra 2, Anchialos 1, Apollónia Pontská 1, Bizóné 1, Byzantion 6, Lýsimacheia 1, Maróneia 12, Mesámbria 3, Odéssos 1, Perinthos (Hérakleia) 53, Sélymbria 1, Topeiros 1, Zóné 1 (celkem 84 nápisů)

{\em Region měst ve vnitrozemí:} Augusta Traiana 2, Hérakleia Sintská 1, Filippopolis 2, údolí středního toku řeky Strýmón 3 (celkem 8 nápisů)\footnote{Celkem dva nápisy nebyly nalezeny v rámci regionu známých měst, editoři korpusů udávají jejich polohu vzhledem k nejbližšímu modernímu sídlišti.}

{\em Celkový počet individuálních lokalit}: 26

{\em Archeologický kontext nálezu:} funerální 1, sídelní 2, náboženský 2, sekundární 6, neznámý 83

{\em Materiál:} kámen 89 (mramor 81, jiný 1, neznámý 7), kov 2, neznámý 3

{\em Dochování nosiče}: 100 \letterpercent{} 16, 75 \letterpercent{} 3, 50 \letterpercent{} 13, 25 \letterpercent{} 13, kresba 5, nemožno určit 43

{\em Objekt:} stéla 51, architektonický prvek 17, socha 1, nádoba 1, jiný 19 (z toho sarkofág 16), neznámý 2

{\em Dekorace:} reliéf 43, bez dekorace 51; reliéfní dekorace figurální 20 nápisů (vyskytující se motiv: jezdec 3, stojící osoba 4, skupina lidí 1, zvíře 3, obětní scéna 1, Héraklés 1, Artemis 1, tři Nymfy 2, funerální scéna/symposion 1, funerální portrét 1, jiný 2), architektonické prvky 31 nápisů (vyskytující se motiv: naiskos 11, sloup 1, báze sloupu či oltář 7, architektonický tvar/forma 7, geometrický motiv 2, florální motiv 12, věnec 1, jiný 2)

{\em Typologie nápisu:} soukromé 73, veřejné 15, neurčitelné 6

{\em Soukromé nápisy:} funerální 57, dedikační 15, jiný 1

{\em Veřejné nápisy:} seznamy 1, honorifikační dekrety 4, státní dekrety 1, funerální na náklady obce 2, náboženský 3, jiný 4 (z toho veřejné dedikace 3)

{\em Délka:} aritm. průměr 6,05 řádku, medián 5, max. délka 18, min. délka 1

{\em Obsah:} dórský dialekt 1, latinský text 6 nápisů, písmo římského typu 14; hledané termíny (administrativní termíny 17 - celkem 44 výskytů, epigrafické formule 18 - 109 výskytů, honorifikační 2 - 2 výskytů, náboženské 20 - 35 výskytů, epiteton 5 - počet výskytů 5)

{\em Identita:} řecká božstva 10, egyptská božstva 1, pojmenování míst a funkcí typických pro řecké náboženské prostředí, místní thrácká božstva, regionální epiteton 3, subregionální epiteton 2, kolektivní identita 7 termínů, celkem 7 výskytů - obyvatelé řeckých obcí z oblasti Thrákie 7, mimo ni 0; celkem 132 osob na nápisech, 40 nápisů s jednou osobou; max. 8 osob na nápis, aritm. průměr 1,4 osoby na nápis, medián 1; komunita multikulturního charakteru se zastoupením řeckého, římského a thráckého prvku, jména pouze řecká (23,4 \letterpercent{}), pouze thrácká (2,12 \letterpercent{}), pouze římská (18,08 \letterpercent{}), kombinace řeckého a thráckého (0 \letterpercent{}), kombinace řeckého a římského (22,34 \letterpercent{}), kombinace thráckého a římského (0 \letterpercent{}), kombinovaná řecká, thrácká a římská jména (5,31 \letterpercent{}), jména nejistého původu (9,55 \letterpercent{}), beze jména (19,14 \letterpercent{}); geografická jména z oblasti Thrákie 2, geografická jména mimo Thrákii 4;

\NC\AR
\HL
\HL
\stoptable

Nápisů pocházejících z období mezi 1. a 2. st. n. l. je přibližně o 67 \letterpercent{} víc než v předcházejícím období. Většina nápisů pochází z oblasti na pobřeží, z okolí původně řeckých měst. Nápisy ve vnitrozemí se objevují podél cest, které jsou pravidelně udržovány a spravovány od druhé poloviny 1. st. n. l., a zejména v okolí rozvíjejících se regionálních urbánních center, jako je Filippopolis či Kabylé, a dále v oblasti středního toku řeky Strýmón, jak ilustruje mapa 6.07 v Apendixu 2. Největším epigrafickým producentem je od druhé poloviny 1. st. n. l. Perinthos (Hérakleia) odkud pochází 56 \letterpercent{} celkové produkce.\footnote{Nárůst epigrafické produkce Perinthu souvisí s faktem, že se v polovině 1. st. n. l. stalo hlavním městem nově vytvořené provincie {\em Thracia} a je logické, že se zde soustředila většina produkce a byla zde nalezena většina nápisů.} Pozici středního producenta si i nadále udržuje Maróneia, zatímco Byzantion, které hrálo roli největšího epigrafického producenta v předchozích dvou stoletích, nyní produkuje zhruba 6 \letterpercent{} celkové produkce.

Materiálem nosiče nápisů je z 94 \letterpercent{} kámen, z převážné části je to mramor, dále vápenec a místní zdroj kamene. Nápisy na kovových předmětech se dochovaly pouze dva.\footnote{Do skupiny nápisů datovaných do 1. až 2. st. n. l. patří vojenský diplom AE 2007, 1260 psaný na bronzové destičce, který uděluje římské občanství, právo nosit římské jméno, pořídit si manželku a právo na vlastní pozemek. Text je psaný latinsky a pochází z okolí vesnice Trapoklovo v jihovýchodním Bulharsku. Bohužel text je velmi poškozený, a tudíž není možné dělat žádné další závěry. Nicméně se jedná o jeden z prvních projevů udělení římského občanství vojákovi za jeho dlouholeté služby, který byl nalezen přímo na území Thrákie. Zároveň se jedná o další důkaz narůstajícího vlivu římské armády nejen na složení obyvatelstva, a jejich onomastické zvyky, ale ovlivňující i přítomnost veteránů ve vnitrozemské Thrákii (Dana 2013, 239-264).} Nejoblíbenější formou kamenného nápisu je i nadále stéla, nicméně v této době se stávají populární i funerální nápisy na sarkofázích, pocházejících v 15 případech z Perinthu (Hérakleii) a v jednom z Byzantia.

