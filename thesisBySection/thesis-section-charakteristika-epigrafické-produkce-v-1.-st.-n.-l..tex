
\section[charakteristika-epigrafické-produkce-v-1.-st.-n.-l.]{Charakteristika epigrafické produkce v 1. st. n. l.}

Většina nápisů z 1. st. n. l. i nadále pochází z pobřežních oblastí, kde však dochází k proměně komunit směrem k větší otevřenosti a multikulturalitě a narůstající roli institucionální epigrafické produkce. Celková délka nápisů se prodlužuje, stejně tak se zvyšuje i počet osob vystupujících na nápisech. Římský kulturní prvek se začíná prosazovat jak na poli proměňujících se osobních jmen, tak i přítomností latinsky psaných nápisů či jejich částí.

\placetable[none]{}
\starttable[|l|]
\HL
\NC {\em Celkem:} 68 nápisů

{\em Region měst na pobřeží:} Abdéra 2, Abritus 1, Anchialos 1, Apollónia Pontská 1, Byzantion 15, Ferai 1, Koila 1, Madytos 1, Maróneia 13, Mesámbria 1, Odéssos 2, Perinthos (Hérakleia) 13, Topeiros 3 (celkem 55 nápisů)

{\em Region měst ve vnitrozemí:} Filippopolis 4, Neiné 1, Nicopolis ad Nestum 1, Serdica 1, údolí středního toku řeky Strýmón 1 (celkem 8 nápisů)\footnote{Celkem čtyři nápisy nebyly nalezeny v rámci regionu známých měst, editoři korpusů udávají jejich polohu vzhledem k nejbližšímu modernímu sídlišti či muzeu, kde se v současnosti nacházejí. Jeden nápis byl nalezen mimo území Thrákie, avšak editoři uvádějí jeho původ jako thrácký na základě širšího kontextu.}

{\em Celkový počet individuálních lokalit}: 27

{\em Archeologický kontext nálezu:} sídelní 5, náboženský 3, sekundární 8, neznámý 52

{\em Materiál:} kámen 64 (mramor 55, z toho z Thasu 1, z okolí Maróneii 2; vápenec 1; jiný 4, z toho póros 1; neznámý 4), kov kombinovaný s kamenem 1, neznámý 3

{\em Dochování nosiče}: 100 \letterpercent{} 6, 75 \letterpercent{} 7, 50 \letterpercent{} 7, 25 \letterpercent{} 10, oklepek 2, kresba 2, ztracený 1, nemožno určit 33

{\em Objekt:} stéla 46, architektonický prvek 13, socha 1, jiný 2, neznámý 6, (celkem 3 sarkofágy)

{\em Dekorace:} reliéf 25, malovaná 1, bez dekorace 42; reliéfní dekorace figurální 13 nápisů (vyskytující se motiv: jezdec 1, stojící osoba 2, lovecká scéna 1, funerální scéna/symposion 2, obětní scéna 2, jiný 1), architektonické prvky 11 nápisů (vyskytující se motiv: naiskos 3, sloup 2, báze sloupu či oltář 4, architektonický tvar/forma 4, florální motiv 2, jiný 2)

{\em Typologie nápisu:} soukromé 37, veřejné 28, neurčitelné 3

{\em Soukromé nápisy:} funerální 24, dedikační 10, jiný 1, neznámý 1

{\em Veřejné nápisy:} seznamy 3, honorifikační dekrety 10, státní dekrety 2, náboženské 5, nařízení 1, funerální na náklady obce 1, jiný 4, neznámý 2

{\em Délka:} aritm. průměr 7,53 řádku, medián 5, max. délka 94, min. délka 1

{\em Obsah:} latinský text 12 nápisů, písmo římského typu 1; hledané termíny (administrativní termíny 29 nápisů - celkem 52 výskytů, epigrafické formule 13 - 24 výskytů, honorifikační 5 - 7 výskytů, náboženské 25 - 45 výskytů, epiteton 5 - počet výskytů 9)

{\em Identita:} řecká božstva 11, egyptská božstva 2, pojmenování míst a funkcí typických pro řecké náboženské prostředí, objevující se místní thrácká božstva, regionální epiteton 1, subregionální epiteton 4, kolektivní identita 6 termínů, celkem 10 výskytů - obyvatelé řeckých obcí z oblasti Thrákie 3, mimo ni 0, kolektivní pojmenování Thráx 4, Rómaios 2, Kimbros 1; celkem 152 osob na nápisech, 27 nápisů s jednou osobou; max. 34 osob na nápis, aritm. průměr 2,35 osoby na nápis, medián 1; komunita multikulturního charakteru se zastoupením řeckého, thráckého prvku a narůstající pozicí římského prvku, jména pouze řecká (13,23 \letterpercent{}), pouze thrácká (4,4 \letterpercent{}), pouze římská (25 \letterpercent{}), kombinace řeckého a thráckého (8,82 \letterpercent{}), kombinace řeckého a římského (16,17 \letterpercent{}), kombinace thráckého a římského (4,4 \letterpercent{}), kombinovaná řecká, thrácká a římská jména (5,88 \letterpercent{}), jména nejistého původu (7,31 \letterpercent{}), beze jména (14,7 \letterpercent{}); geografická jména z oblasti Thrákie 18, z toho pojmenování stratégií v Thrákii 11, geogr. jména mimo Thrákii 0;

\NC\AR
\HL
\HL
\stoptable

Do 1. st. n. l. bylo celkem datováno 68 nápisů, což je zhruba stejně jako v 1. st. př. n. l. Jak je dobře vidět na mapě 6.07 v Apendixu 2, nápisy pocházejí převážně z pobřežních oblastí, nicméně osm nápisů bylo nalezeno i ve vnitrozemí. Zatímco velké produkční centrum v této době chybí, produkční centra střední velikosti se nacházejí i nadále v Byzantiu, dále v Perinthu a Maróneii.

Téměř výhradně je použitým materiálem kámen, nejčastější formou nosiče jsou i nadále stély, avšak v malé míře se začínají objevovat i sarkofágy. Přes polovinu nápisů představují funerální texty, nicméně počet veřejných nápisů narostl na více než 40 \letterpercent{}, což naznačuje větší míru institucionálního zapojení na epigrafické produkci. Dělo se tak zejména v okolí Perinthu, který se stal v druhé polovině století hlavním městem nově vzniklé provincie {\em Thracia} (Sharankov 2005, 521; Sayar 1998, 74).

