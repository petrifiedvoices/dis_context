
\section[charakteristika-epigrafické-produkce-v-1.-st.-př.-n.-l.]{Charakteristika epigrafické produkce v 1. st. př. n. l.}

Celková epigrafická produkce v 1. st. př. n. l. poměrně výrazně klesá. Epigraficky aktivní komunity jsou i nadále z poloviny řecké, avšak i nadále dochází k jejich postupnému otevírání a mísení onomastických a náboženských tradic. Veřejné nápisy představují až třetinu celkové produkce a poprvé se objevují nápisy obsahující latinský text. V malé míře narůstají i počty dochovaných římských jmen.

\placetable[none]{}
\starttable[|l|]
\HL
\NC {\em Celkem:} 69 nápisů

{\em Region měst na pobřeží:} Abdéra 2, Apollónia Pontská 1, Byzantion 36, Dionýsopolis 1, Maróneia 10, Mesámbria 11, Odéssos 4, Perinthos (Hérakleia) 1, Sélymbria 1, Séstos 1 (celkem 68 nápisů)

{\em Region měst ve vnitrozemí:} Didymoteichon (Plótinúpolis) 1

{\em Celkový počet individuálních lokalit}: 13

{\em Archeologický kontext nálezu:} funerální 2, sídelní 1, náboženský 1, sekundární 6, neznámý 59

{\em Materiál:} kámen 67 (mramor 67, z toho mramor z Thasu 1), neznámý 2

{\em Dochování nosiče}: 100 \letterpercent{} 4, 75 \letterpercent{} 5, 50 \letterpercent{} 5, 25 \letterpercent{} 12, kresba 2, nemožno určit 41

{\em Objekt:} stéla 64, architektonický prvek 3, neznámý 2

{\em Dekorace:} reliéf 48, bez dekorace 21; reliéfní dekorace figurální 34 nápisů (vyskytující se motiv: jezdec 1, stojící osoba 2, sedící osoba 1, funerální scéna/symposion 3), architektonické prvky 16 nápisů (vyskytující se motiv: naiskos 5, báze sloupu či oltář 3, věnec 1, florální motiv 3, architektonický tvar/forma 5)

{\em Typologie nápisu:} soukromé 50, veřejné 17, neurčitelné 2

{\em Soukromé nápisy:} funerální 42, dedikační 11, jiný 1\footnote{Jeden nápis měl vzhledem ke své nejednoznačnosti kombinovanou funkci funerálního a zároveň dedikačního nápisu, proto je součet nápisů obou typů vyšší než celkový počet soukromých nápisů.}

{\em Veřejné nápisy:} náboženské 2, seznamy 3, honorifikační dekrety 2, státní dekrety 8, jiné 1, neznámý 1

{\em Délka:} aritm. průměr 5,4 řádku, medián 2, max. délka 49, min. délka 1

{\em Obsah:} dórský dialekt 9, latinský text 1 nápis; hledané termíny (administrativní termíny 18 - celkem 44 výskytů, epigrafické formule 9 - 28 výskytů, honorifikační 13 - 17 výskytů, náboženské 23 - 35 výskytů, epiteton 4 - počet výskytů 5)

{\em Identita:} řecká božstva 10, egyptská božstva 3, pojmenování míst a funkcí typických pro řecké náboženské prostředí, regionální epiteton 4, kolektivní identita 4 termíny, celkem 6 výskytů - obyvatelé řeckých obcí z oblasti Thrákie 1, ale i mimo ni 1, kolektivní pojmenování Thráx 3, Rómaios 1; celkem 141 osob na nápisech, 41 nápisů s jednou osobou; max. 20 osob na nápis, aritm. průměr 2,04 osoby na nápis, medián 1; komunita řeckého charakteru s částečným zastoupením římského a thráckého prvku, jména pouze řecká (46,47 \letterpercent{}), pouze thrácká (1,44 \letterpercent{}), pouze římská (4,34 \letterpercent{}), kombinace řeckého a thráckého (4,34 \letterpercent{}), kombinace řeckého a římského (5,79 \letterpercent{}), kombinace thráckého a římského (1,44 \letterpercent{}), kombinovaná řecká, thrácká a římská jména (5,79 \letterpercent{}), jména nejistého původu (21,71 \letterpercent{}), beze jména (8,69 \letterpercent{}); geografická jména z oblasti Thrákie 3, geografická jména mimo Thrákii 2;

\NC\AR
\HL
\HL
\stoptable

Do 1. st. př. n. l. bylo datováno 69 nápisů, což znamená pokles epigrafické produkce o 40 \letterpercent{} oproti předcházejícímu období. Jak je patrné na mapě 6.06 v Apendixu 2, nápisy pocházejí téměř výhradně z pobřežních oblastí s výjimkou lokality Didymoteichon, která se nachází v regionu pozdějšího města Plótinúpolis na řece Tonzos. Hlavní produkční centrum je i nadále v Byzantiu, odkud pochází 43 \letterpercent{} nápisů. Další středně velká produkční centra se nachází v Maróneii, v Mesámbrii, a částečně i v Odéssu, podobně jako v předcházejícím období.

Archeologický kontext míst nálezu je bohužel opět z velké části neznámý, případně sekundární a materiál použitý na výroby nosičů nápisů je opět výhradně kámen, zejména mramor. Nejrozšířenějším nosičem je kamenná stéla a opět zcela chybí nápisy na keramice či na kovových předmětech, které se v Thrákii vyskytovaly mezi 5. a 3. st. př. n. l.

