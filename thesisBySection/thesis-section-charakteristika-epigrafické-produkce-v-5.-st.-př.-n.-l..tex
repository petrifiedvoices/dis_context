
\section[charakteristika-epigrafické-produkce-v-5.-st.-př.-n.-l.]{Charakteristika epigrafické produkce v 5. st. př. n. l.}

V 5. st. př. n. l. dochází ke znatelnému nárůstu epigrafické produkce a rozšíření nápisů do většího počtu řeckých komunit na pobřeží, kde i nadále převládají funerální nápisy. V thráckém vnitrozemí se poprvé objevují nápisy použité v rámci thráckého funerálního kontextu, patrně patřící thrácké aristokracii a související s prominentní pozicí, kterou aristokraté ve společnosti zastávali.

\placetable[none]{}
\starttable[|l|]
\HL
\NC {\em Celkem:} 60 nápisů

{\em Region měst na pobřeží:} Abdéra 10, Apollónia Pontská 9, Mesámbria 4, Perinthos (Hérakleia) 2, Strýmé 20, Zóné 9 (celkem 54 nápisů)

{\em Region měst ve vnitrozemí:} Pulpudeva (pozdější Filippopolis) 5\footnote{Jeden nápis byl nalezen mimo region známých měst, a není ho možné zařadit do regionu měst na pobřeží, ani ve vnitrozemí.}

{\em Celkový počet individuálních lokalit:} 13

{\em Archeologický kontext nálezu:} funerální 12, sídelní 1, nábož. 1, sekundární 11, neznámý 35

{\em Materiál:} kámen 54 (mramor 26, z toho mramor z Thasu 1, vápenec 6, jiný 20, z čehož je pískovec 1, póros 5; neznámý 2), kov 3 (stříbro, zlato), keramika 3

{\em Dochování nosiče}: 100 \letterpercent{} 10, 75 \letterpercent{} 10, 50 \letterpercent{} 23, 25 \letterpercent{} 7, kresba 2, nemožno určit 8

{\em Objekt:} stéla 49, architektonický prvek 7, nádoba 3, jiné 1

{\em Dekorace:} reliéf 12, malovaná dekorace 0, bez dekorace 48; reliéfní dekorace figurální celkem 2 nápisy (vyskytující se motiv: stojící osoba 2, sedící osoba 1, obětní scéna 1), architektonické prvky 14 (vyskytující se motiv: naiskos 3, sloup 2, báze sloupu či oltář 5, florální motiv 1, architektonický tvar 5)

{\em Typologie nápisu:} soukromé 55, veřejné 3, neurčitelné 2

{\em Soukromé nápisy:} funerální 43, dedikační 6, vlastnictví 4, jiný 1, neznámý 1

{\em Veřejné nápisy:} nařízení 1, jiný 2

{\em Délka:} aritm. průměr 2,68 řádku, medián 2, max. délka 16, min. délka 1

{\em Obsah:} dórský dialekt 2, iónsko-attický 4; bústrofédon 1, stoichédon 1; hledané termíny (administrativní 4 - celkem 4 výskyty, epigrafické formule 3 - celkem 7 výskytů, honorifikační 0, náboženské 6 - celkem 8 výskytů, epiteton 6 - celkem 6 výskytů)

{\em Identita:} řecká božstva, regionální epiteta, kolektivní identita 4 - pouze obyvatelé řeckých obcí mimo oblast Thrákie, celkem 56 osob na nápisech, 46 nápisů s jednou osobou; max. 2 osoby na nápisech, aritm. průměr 0,93 osoby na nápis, medián 1; komunita převládajícího řeckého charakteru, jména pouze řecká (65 \letterpercent{}), thrácká (1,67 \letterpercent{}), kombinace řeckého a thráckého (1,67 \letterpercent{}), jména nejistého původu (10 \letterpercent{})

\NC\AR
\HL
\HL
\stoptable

Do 5. století celkem spadá 60 nápisů a k celkovému rozšíření nálezových lokalit na 13 oproti třem lokalitám z 6. st. př. n. l., a s tím i spojeným nárůstem epigrafické produkce.\footnote{Nálezové lokality nesou jméno dle nejbližšího moderního osídlení, pokud není známo jejich antické jméno. V databázi jsou vedeny jako „{\em Modern Location}”. V regionu antického města je zpravidla více nálezových lokalit, které mohou, ale nutně nemusí korespondovat s archeologickou lokalitou.} Mapa 6.02 v Apendixu 2 ilustruje rozložení nápisů na území Thrákie v 5. st. př. n. l. Většina nálezových lokalit pochází z území řeckých kolonií na pobřeží Egejského, Černého a Marmarského moře, v maximální vzdálenosti do 20 km od pobřeží. Největším producentem s 20 nápisy je řecké osídlení Strýmé poblíž moderního Molyvoti na egejském pobřeží.\footnote{Z dalších měst, která v té době existovala, se nápisy do dnešní doby nedochovaly, či ještě nebyly objeveny. Možné je však také, že tato města v 5. st. př. n. l. nápisy neprodukovala, ať už z důvodů nestabilních podmínek, či odlišného přístupu tamních obyvatel k publikaci nápisů.} Z thráckého vnitrozemí se dochovaly nápisy v řádu pěti kusů z okolí moderní vesnice Duvanlij. Nápisy byly objeveny jako součást bohaté pohřební výbavy thráckých aristokratů v monumentálních hrobkách severně od řeky Hebros v regionu thráckého osídlení {\em Pulpudeva}, pozdější Filippopole. Tyto vnitrozemské nápisy se neodlišují pouze svou polohou, ale i použitým materiálem a funkcí, kterou objekt zastával v rámci společnosti.

Podobně jako v 6. st. př. n. l. je převážná většina 90 \letterpercent{} nosičů nápisu zhotovena z kamene, nicméně zbývajících 10 \letterpercent{} nápisů se nachází na kovových předmětech a na keramických nádobách.\footnote{Z nápisů tesaných do kamene má výrazná většina 82 \letterpercent{} charakter soukromého nápisu, tedy nápisu sloužícího pro soukromé účely jedince či skupiny lidí. Téměř ze tří čtvrtin převládají funerální nápisy se 43 exempláři, dedikační nápisy představují zhruba 10 \letterpercent{} a zbývající nápisy není možné přesněji určit.} Hlavní role nápisů psaných na kameni byla předat a sdělit specifickou zprávu v rámci lokální komunity, proto byly z převážné většiny určené k veřejnému vystavení, přístupné všem. Na lokální původ nápisů tesaných do kamene poukazuje i použitý materiál jako je vápenec, pískovec či porézní kámen, tzv. póros.\footnote{Tento fakt poukazuje na regionální charakter místních komunit, které primárně využívaly místně dostupný materiál a dále podporuje teorii, že nápisy byly produkovány místně a nestávaly se předmětem dálkového obchodu, alespoň v 5. st. př. n. l.} Oproti tomu primární funkce nápisů na kovových předmětech a keramických nádobách bylo v pěti z šesti nápisů označení vlastnictví či autorství a v jednom případě dedikace božstvu. Tato skupina nápisů sloužila primárně pro interní potřebu majitele a zpravidla se nejednalo o nápisy veřejně přístupné komukoliv, ale pouze vybraným členům z okolí majitele předmětu. Použitý materiál byl často drahý kov, jako je zlato či stříbro, což naznačuje na vysoké společenské postavení majitele. Mnohdy měly tyto předměty po smrti majitele využití jako sekundární funerální nápisy, a to zejména ve společnostech s kmenovým uspořádáním založených na charismatu a společenské prestiži jedince (Whitley 1991, 354-361; Bliege Bird a Smith 2005, 221-222, 233-234).

