
\section[charakteristika-epigrafické-produkce-ve-2.-až-1.-st.-př.-n.-l.]{Charakteristika epigrafické produkce ve 2. až 1. st. př. n. l.}

Nápisy datované do 2. a 1. st. př. n. l. pocházejí i nadále převážně z řeckého prostředí, nicméně se začíná projevovat vliv římské přítomnosti v regionu. To s sebou nese zvýšenou přítomnost římských jmen a stále pokračující nárůst počtu veřejných nápisů, v nichž se Řím objevuje jako mocný spojenec. Nápisy pocházejí převážně z pobřežních oblastí, avšak je možné pozorovat nárůst epigrafické aktivity v oblasti středního toku Strýmónu. Zapojení thrácké aristokracie na produkci epigrafických pramenů je pozorovatelné v malé míře.

\placetable[none]{}
\starttable[|l|]
\HL
\NC {\em Celkem:} 125 nápisů

{\em Region měst na pobřeží:} Abdéra 3, Apollónia Pontská 1, Byzantion 79, Maróneia 10, Mesámbria 7, Odéssos 14, Perinthos (Hérakleia) 1, Sélymbria 3 (celkem 118 nápisů)

{\em Region měst ve vnitrozemí:} Beroé (Augusta Traiana) 2, (Marcianopolis) 1, údolí střední toku řeky Strýmónu 3, Hérakleia Sintská 1

{\em Celkový počet individuálních lokalit}: 21

{\em Archeologický kontext nálezu:} funerální 3, sídelní 1, sekundární 9, neznámý 112

{\em Materiál:} kámen 125 (mramor 118, z toho mramor z Prokonnésu 1, místní mramor 1, jiné 1; z čehož je varovik 1)

{\em Dochování nosiče}: 100 \letterpercent{} 8, 75 \letterpercent{} 14, 50 \letterpercent{} 10, 25 \letterpercent{} 8, kresba 1, nemožno určit 84

{\em Objekt:} stéla 119, architektonický prvek 4, jiné 1

{\em Dekorace:} reliéf 103, malovaná dekorace 1, bez dekorace 22; reliéfní dekorace figurální 81 nápisů (vyskytující se motiv: jezdec 2, stojící osoba 5, sedící osoba 6, skupina lidí 2, zvíře 5, funerální scéna/symposion 15, scéna oběti 4, jiný 2), architektonické prvky 24 nápisů (vyskytující se motiv: naiskos 6, sloup 1, báze sloupu či oltář 1, věnec 1, florální motiv 11, architektonický tvar/forma 2, jiný 2)

{\em Typologie nápisu:} soukromé 107, veřejné 14, neurčitelné 4

{\em Soukromé nápisy:} funerální 98, dedikační 11, jiné 1\footnote{Součet nápisů jednotlivých typů je vyšší než počet veřejných nápisů vzhledem k možným kombinacím jednotlivých typů v rámci jednoho nápisu.}

{\em Veřejné nápisy:} náboženské 1, seznamy 1, honorifikační dekrety 1, státní dekrety 6, neznámý 2

{\em Délka:} aritm. průměr 3,92 řádku, medián 3, max. délka 60, min. délka 1

{\em Obsah:} dórský dialekt 11; hledané termíny (administrativní termíny 16 - celkem 44 výskytů, epigrafické formule 13 - 46 výskytů, honorifikační 14 - 19 výskytů, náboženské 24 - 35 výskytů, epiteton 2 - počet výskytů 2)

{\em Identita:} řecká božstva, pojmenování míst a funkcí typických pro řecké náboženské prostředí, místní thrácká božstva, regionální epiteton 2, kolektivní identita 3 termíny, celkem 3 výskyty - obyvatelé řeckých obcí z oblasti Thrákie 1, ale i mimo ni 1, kolektivní pojmenování barbaroi 1; celkem 217 osob na nápisech, 86 nápisů s jednou osobou; max. 63 osob na nápis, aritm. průměr 1,73 osoby na nápis, medián 1; komunita řeckého charakteru se zastoupením římského a thráckého prvku, jména pouze řecká (56 \letterpercent{}), pouze thrácká (2,4 \letterpercent{}), pouze římská (1,6 \letterpercent{}), kombinace řeckého a thráckého (3,2 \letterpercent{}), kombinace řeckého a římského (8 \letterpercent{}), kombinace thráckého a římského (0,8 \letterpercent{}), jména nejistého původu (15,2 \letterpercent{}), beze jména (11,2 \letterpercent{}){\bf ;} geografická jména z oblasti Thrákie 0, geografická jména mimo Thrákii 7;

\NC\AR
\HL
\HL
\stoptable

Oproti nápisům datovaným do 3. až 2. st. př. n. l. je u skupiny nápisů datovaných do 2. až 1. st. př. n. l. pozorovatelný téměř 200 \letterpercent{} nárůst celkového počtu nápisů. Většina produkčních center se nachází na pobřeží, nicméně individuální nápisy byly nalezeny i v thráckém vnitrozemí, zejména v okolí řeky Strýmón, jak je možné vidět na mapě 6.05 v Apendixu 2. Hlavním produkčním centrem je Byzantion, avšak pozici menších produkčních center si i nadále udržuje Odéssos a Maróneia.

Materiálem, z nějž jsou nápisy zhotovovány, je výhradně kámen a většina nápisů má tvar stély. Převládající funkce nápisů je funerální a v malé míře i dedikační. Objevují se i nápisy veřejné, ač v menším počtu než v předcházejících obdobích.

