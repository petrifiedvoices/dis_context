
\section[charakteristika-epigrafické-produkce-ve-2.-st.-n.-l.]{Charakteristika epigrafické produkce ve 2. st. n. l.}

Ve 2. st. n. l. dochází k prudkému nárůstu epigrafické aktivity, a to zejména ve vnitrozemí, odkud pochází více nápisů než z pobřeží. Institucionální uspořádání a politický vliv Říma mají zásadní vliv na rozšíření epigrafické produkce. Veřejné nápisy představují téměř polovinu celého souboru, častější je i výskyt hledaných termínů a latinského textu. Římská jména se vyskytují na polovině všech nápisů, ať už samostatně, či v kombinaci. Dochází k prolínání kulturních a náboženských tradic, objevují se ale i nové prvky a jména kultů. Politická příslušnost a status nabírají na důležitosti, a proto narůstá i počet honorifikací.

\placetable[none]{}
\starttable[|l|]
\HL
\NC {\em Celkem:} 254 nápisů

{\em Region měst na pobřeží:} Abdéra 5, Anchialos 2, Apollónia Pontská 1, Byzantion 22, Dionýsopolis 1, Kallipolis 2, Madytos 1, Maróneia 19, Mesámbria 1, Odéssos 17, Perinthos (Hérakleia) 21, Sélymbria 1, Topeiros 4, Zóné 1 (celkem 98 nápisů)

{\em Region měst ve vnitrozemí:} Augusta Traiana 23, Carasura 1, Discoduraterae 3, Filippopolis 29, Hadriánopolis 1, Hérakleia Sintská 2, Marcianopolis 4, Neiné 1, Nicopolis ad Istrum 28, Pautália 5, Plótinúpolis 2, Serdica 18, Traianúpolis 2, údolí středního toku řeky Strýmón 30 (celkem 149 nápisů)\footnote{Celkem sedm nápisů nebylo nalezeno v rámci regionu známých měst, editoři korpusů udávají jejich polohu vzhledem k nejbližšímu modernímu sídlišti, či uvádí muzeum, v němž se nachází.}

{\em Celkový počet individuálních lokalit}: 69

{\em Archeologický kontext nálezu:} funerální 5, sídelní 43 (z toho obchodní 10), náboženský 5, sekundární 25, jiný 3, neznámý 173

{\em Materiál:} kámen 246 (mramor 169, z Chalkedónu 1, z Prokonnésu 1; vápenec 49, jiný 10, z toho syenit 5, póros 1; neznámý 18), kov 2, neznámý 6

{\em Dochování nosiče}: 100 \letterpercent{} 36, 75 \letterpercent{} 35, 50 \letterpercent{} 44, 25 \letterpercent{} 43, oklepek 2, kresba 8, ztracený 8, nemožno určit 78

{\em Objekt:} stéla 148, architektonický prvek 89, socha 2, jiný 8 (z toho {\em instrumentum domesticum} 1, vojenský diplom 1) neznámý 7

{\em Dekorace:} reliéf 144, bez dekorace 110; reliéfní dekorace figurální 44 nápisů (vyskytující se motiv: jezdec 8, stojící osoba 6, skupina lidí 1, zvíře 1, Artemis 1, scéna lovu 1, funerální scéna/symposion 10, funerální portrét 9, jiný 3), architektonické prvky 99 nápisů (vyskytující se motiv: naiskos 9, sloup 9, báze sloupu či oltář 45, architektonický tvar/forma 18, geometrický motiv 4, florální motiv 18, věnec 2, jiný 7)

{\em Typologie nápisu:} soukromé 130, veřejné 117, neurčitelné 7

{\em Soukromé nápisy:} funerální 84, dedikační 49, vlastnictví 1, jiný 2\footnote{Několik nápisů mělo vzhledem ke své nejednoznačnosti kombinovanou funkci, proto je součet nápisů obou typů vyšší než celkový počet soukromých nápisů.}

Veřejné nápisy: seznamy 2, honorifikační dekrety 63, státní dekrety 6, nařízení 5, náboženský 13, jiný 24, neznámý 4

Délka: aritm. průměr 6,52 řádku, medián 5, max. délka 62, min. délka 1

{\em Obsah:} dórský dialekt 2, latinský text 22 nápisů{\bf ,} písmo římského typu 85; hledané termíny (administrativní termíny 40 - celkem 284 výskytů, epigrafické formule 29 - 199 výskytů, honorifikační 10 - 13 výskytů, náboženské 46 - 156 výskytů, epiteton 26 - počet výskytů 33)

{\em Identita:} řecká božstva 19, egyptská božstva 2, římská božstva 2, thrácká božstva 2, pojmenování míst a funkcí typických pro řecké náboženské prostředí{\bf ,} regionální epiteton 15, subregionální epiteton 11, kolektivní identita 18 termínů, celkem 55 výskytů - obyvatelé řeckých obcí z oblasti Thrákie 12, mimo ni 0; kolektivní pojmenování etnik či kmenů (Thráx 13, Rómaios 9, barbaros 1, Asianos 1, Kappadox 1), člen fýly 1; celkem 511 osob na nápisech, 94 nápisů s jednou osobou; max. 29 osob na nápis, aritm. průměr 2,02 osoby na nápis, medián 1{\bf ;} komunita multikulturního charakteru se zastoupením řeckého, římského a thráckého prvku, se silnou přítomností římského prvku, jména pouze řecká (11,06 \letterpercent{}), pouze thrácká (1,18 \letterpercent{}), pouze římská (31,25 \letterpercent{}), kombinace řeckého a thráckého (3,95 \letterpercent{}), kombinace řeckého a římského (15,81 \letterpercent{}), kombinace thráckého a římského (3,95 \letterpercent{}), kombinovaná řecká, thrácká a římská jména (6,32 \letterpercent{}), jména nejistého původu (12,23 \letterpercent{}), beze jména (14,22 \letterpercent{}); geografická jména z oblasti Thrákie 15, geografická jména mimo Thrákii 4;

\NC\AR
\HL
\HL
\stoptable

Do 2. st. n. l. bylo datováno celkem 254 nápisů, což představuje nárůst o 272 \letterpercent{} oproti předcházejícímu období. Jak je patrné z mapy 6.08 v Apendixu 2, poprvé v tomto období převládá epigrafická produkce ve vnitrozemí, a nikoliv na pobřeží. Většina lokalit s nápisy ve vnitrozemí se nacházela v povodí velkých řek či na trase {\em Via Diagonalis}, římské cesty spojující severozápad s jihovýchodem, procházející skrz města Serdicu, Filippopolis, Perinthos a Byzantion (Jireček 1877; Madzharov 2009, 70-31). Nápisy pocházely převážně z níže položených sídel, v horských oblastech se našly jednotlivé nápisy pouze v oblasti cest a průsmyků či v okolí vojenských posádek. Hlavními produkčními centry byla oblast údolí středního toku Strýmónu s městy Hérakleia Sintská a Neiné, Filippopolis, Nicopolis ad Istrum, Augusta Traiana, Byzantion a Perinthos. Oproti předcházejícím obdobím dochází k přesunu epigrafické produkce do většího počtu měst do jednoho dominantního centra. Oproti 1. st. n. l. upadá celková produkce Byzantia a Perinthu, což souvisí s přesunem hlavního města provincie do Filippopole, kde je naopak možné pozorovat markantní nárůst nápisů (Topalilov 2012, 13).

Použitým materiálem je výhradně kámen, a to zejména mramor ze 70 \letterpercent{}, vápenec z 18 \letterpercent{} a syenit z 2 \letterpercent{}. Nosiče nápisů mají ve dvou třetinách tvar stély, v necelé třetině tvar architektonického prvku, dále čtyři nápisy se nacházejí na sochách, jeden na mozaice a 19 na sarkofázích či jejich fragmentech.

