
\section[charakteristika-epigrafické-produkce-ve-2.-st.-př.-n.-l.]{Charakteristika epigrafické produkce ve 2. st. př. n. l.}

Nápisy datované do 2. st. př. n. l. pocházejí výhradně z měst na pobřeží. Mezi oblasti, kde se nápisy začínají objevovat nově, patří Thrácký Chersonésos a lokality na pobřeží Marmarského moře. Téměř pětinu nápisů představují nápisy veřejné, a to zejména dekrety vydávané politickou autoritou. Dochází také k nárůstu výskytů hledaných slov, nápisy se stávají delší a obsahově komplexnější. Celkově dochází k většímu otevírání původně řeckých komunit na pobřeží a k výskytu nových prvků. Zároveň s tím ale dochází k útlumu epigrafických aktivit thrácké aristokracie ve vnitrozemí.

\placetable[none]{}
\starttable[|l|]
\HL
\NC {\em Celkem:} 115 nápisů

{\em Region měst na pobřeží:} Abdéra 7, Ainos 1, Anchialos 1, Apollónia Pontská 2, Bisanthé 1, Bizóné 3, Byzantion 69, Dionýsopolis 1, Lýsimacheia 1, Maróneia 15, Mesámbria 5, Odéssos 2, Perinthos (Hérakleia) 1, Sélymbria 1, Séstos 1, Topeiros 1 (celkem 112)

{\em Region měst ve vnitrozemí:} 0\footnote{Celkem tři nápisy byly nalezeny mimo území Thrákie, avšak editoři korpusů je vzhledem k jejich obsahu zařadili mezi nápisy pocházející z Thrákie.}

{\em Celkový počet individuálních lokalit}: 23

{\em Archeologický kontext nálezu:} sídelní 3, náboženský 4, sekundární 12, neznámý 96

{\em Materiál:} kámen 113 (mramor 108, z toho mramor z Prokonnésu 2, vápenec 1, jiné 1), keramika 1, neznámý 1

{\em Dochování nosiče}: 100 \letterpercent{} 6, 75 \letterpercent{} 8, 50 \letterpercent{} 12, 25 \letterpercent{} 11, oklepek 1, kresba 1, nemožno určit 76

{\em Objekt:} stéla 106, architektonický prvek 5, socha 1, nádoba 1;

{\em Dekorace:} reliéf 78, bez dekorace 37; reliéfní dekorace figurální 55 nápisů (vyskytující se motiv: jezdec 1, stojící osoba 1, sedící osoba 1, skupina lidí 1, funerální scéna/symposion 8, jiné 1), architektonické prvky 29 nápisů (vyskytující se motiv: naiskos 3, sloup 1, báze sloupu či oltář 4, věnec 1, florální motiv 11, geometrický motiv 0, architektonický tvar/forma 4, jiné 1)

{\em Typologie nápisu:} soukromé 86, veřejné 25, neurčitelné 4

{\em Soukromé nápisy:} funerální 80, dedikační 8, jiné (jméno autora) 1\footnote{V určitých případech může docházet ke kumulaci jednotlivých typů textů v rámci jednoho nápisu, či jejich nejednoznačnost neumožňuje rozlišit mezi několika typy. V těchto případech pak součet všech typů nápisů může přesahovat celkové číslo nápisů.}

{\em Veřejné nápisy:} náboženské 1, seznamy 2, honorifikační dekrety 7, státní dekrety 15, funerální 2\footnote{Možnost kombinace kategorií, součet všech typů může přesahovat celkové číslo pro danou kategorii.}

{\em Délka:} aritm. průměr 7,93 řádku, medián 2, max. délka 107, min. délka 1

{\em Obsah:} dórský dialekt 10; hledané termíny (administrativní termíny 38 - celkem 125 výskytů, epigrafické formule 16 - 44 výskytů, honorifikační 27 - 91 výskytů, náboženské 26 - 54 výskytů, epiteton 7 - počet výskytů 7)

{\em Identita:} řecká božstva 12 pojmenování, egyptská božstva 2 pojmenování, římská božstva 2 pojmenování, dále názvy míst a funkcí typických pro řecké náboženské prostředí, regionální epiteton 7, kolektivní identita 15 termínů, celkem 25 výskytů - obyvatelé řeckých obcí z oblasti Thrákie, ale i mimo ni, kolektivní pojmenování kmenové příslušnosti (Bíthýnos, Thráx), kolektivní pojmenování Římanů (Rómaios), celkem 186 osob na nápisech, 78 nápisů s jednou osobou; max. 25 osob na nápis, aritm. průměr 1,61 osoby na nápis, medián 1; komunita převládajícího řeckého charakteru, jména pouze řecká (60 \letterpercent{}), pouze thrácká (1,73 \letterpercent{}), pouze římská (4,34 \letterpercent{}), kombinace řeckého a thráckého (4,34 \letterpercent{}), kombinace řeckého a římského (2,6 \letterpercent{}), jména nejistého původu (19,97 \letterpercent{}), beze jména (6,95 \letterpercent{}); geografická jména z oblasti Thrákie 7, geografická jména mimo Thrákii 6;

\NC\AR
\HL
\HL
\stoptable

Celkový počet nápisů datovaných do 2. st. zůstává přibližně na stejné úrovni jako ve 3. st. př. n. l. Naprostá většina nápisů pochází z pobřežních oblastí, z bezprostřední blízkosti řeckých měst, jak dokazuje mapa 6.05 v Apendixu 2. Hlavní produkční centrum pro celý region se přesunulo z Mesámbrie a Maróneie do Byzantia, což je pravděpodobně odrazem nárůstu politického postavení Byzantia, co by spojence Říma (Jones 1973, 7).

Materiálem nesoucím nápisy je téměř výhradně kámen, s jednou výjimkou nápisu {\em SEG} 54:633 na střepu keramické nádoby.\footnote{Na pomezí soukromého a veřejného nápisu je dochované {\em ostrakon} {\em SEG} 54:633 na střepu keramické nádoby, která pochází z Apollónie Pontské a nese řecké jméno Aris{[}s{]}teidés. Pokud by se skutečně jednalo o {\em ostrakon} v pravém slova smyslu, byl by to první důkaz o použití ostrakismu ve 2. st. př. n. l. na území Thrákie. Vzhledem k ojedinělosti nálezu je však možné, že se jedná o označení vlastníka nádoby, či zkušební materiál, na němž se mohl dotyčný cvičit v psaní vlastního jména. Tato poslední možnost je však relativně málo pravděpodobná, vzhledem k tomu, že v té době jistě existovala celá řada materiálů, na něž bylo jednodušší psát než na poměrně tvrdou keramiku. Do doby než, však bude nalezeno více podobných nápisů, které by dosvědčovaly existenci ostrakismu v Thrákii, je vhodné tento konkrétní nápis považovat spíše za označení vlastnictví, než za projev poměrně sofistikované metody vyjádření politického názoru a uplatnění politické moci v Apollónii Pontské.} Nosiče nápisů jsou převážně zhotovovány z místního zdroje kamene a dá se tedy i nadále předpokládat, že materiál pro zhotovování nápisů byl získáván v nejbližším okolí produkčních center a nebyl předmětem dálkového obchodu.

