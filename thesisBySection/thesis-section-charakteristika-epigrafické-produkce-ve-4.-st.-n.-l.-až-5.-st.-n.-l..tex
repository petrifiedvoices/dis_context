
\section[charakteristika-epigrafické-produkce-ve-4.-st.-n.-l.-až-5.-st.-n.-l.]{Charakteristika epigrafické produkce ve 4. st. n. l. až 5. st. n. l.}

Nápisy datované do 4. až 5. st. n. l. pocházejí převážně z velkých měst na pobřeží. Pokračuje úbytek jak celkového počtu nápisů, tak i epigraficky aktivních komunit. Komunity jsou spíše uzavřené, nedochází ke kulturní výměně na úrovni předcházejících století. Zcela převládají soukromé nápisy funerální funkce, na nichž je patrný silný vliv křesťanství, který ovlivňuje jak podobu, tak obsah nápisů. Dochází k opětovné individualizaci nápisů, ač rozsah nápisů se nijak nemění.

\placetable[none]{}
\starttable[|l|]
\HL
\NC {\em Celkem:} 35 nápisů

{\em Region měst na pobřeží:} Byzantion 4, Maróneia 13, Perinthos (Hérakleia) 13 (celkem 30 nápisů)

{\em Region měst ve vnitrozemí:} Augusta Traiana 2, Serdica 2, Traianúpolis 1 (celkem 5 nápisů)

{\em Celkový počet individuálních lokalit}: 8

{\em Archeologický kontext nálezu:} funerální 2, sídelní 2, náboženský 1, sekundární 6, neznámý 24

{\em Materiál:} kámen 32 (mramor 23; jiný 1, neznámý 8), kov 2 (olovo 2), jiný 1

{\em Dochování nosiče}: 100 \letterpercent{} 4, 75 \letterpercent{} 3, 50 \letterpercent{} 3, 25 \letterpercent{} 10, oklepek 1, nemožno určit 14

{\em Objekt:} stéla 32, architektonický prvek 1, mozaika 1, jiný 1

{\em Dekorace:} reliéf 18, jiná 1, bez dekorace 16; reliéfní dekorace figurální 2 nápisy (vyskytující se motiv: zvíře 1, scéna lovu 1, skupina lidí 1, stojící osoba 1, socha 1), architektonické prvky 17 nápisů (vyskytující se motiv: naiskos 9, florální motiv 1, jiný 9 (kříž 9)

{\em Typologie nápisu:} soukromé 30, veřejné 1, neurčitelné 4

{\em Soukromé nápisy:} funerální 28, jiný 2 (proklínací destička 2)

{\em Veřejné nápisy:} jiný 1 (z toho hraniční kámen 1)

{\em Délka:} aritm. průměr 6,67 řádku, medián 6, max. délka 22, min. délka 1

{\em Obsah:} latinský text 1 nápis, písmo římského typu 6; hledané termíny (administrativní termíny 8 - celkem 12 výskytů, epigrafické formule 7 - 23 výskytů, honorifikační 0 - 0 výskytů, náboženské 0 - 0 výskytů, epiteton 0 - počet výskytů 0)

{\em Identita:} vymizení náboženské terminologie, včetně vymizení lokálních kultů z nápisů, regionální epiteton 0, subregionální epiteton 0, kolektivní identita 0 termínů; celkem 28 osob na nápisech, 15 nápisů s jednou osobou; max. 5 osob na nápis, aritm. průměr 0,8 osoby na nápis, medián 1; komunita řeckého a římského charakteru, thrácký prvek zastoupen minimálně, jména pouze řecká (17,14 \letterpercent{}), pouze thrácká (0 \letterpercent{}), pouze římská (8,57 \letterpercent{}), kombinace řeckého a thráckého (2,85 \letterpercent{}), kombinace řeckého a římského (14,28 \letterpercent{}), kombinace thráckého a římského (0 \letterpercent{}), kombinovaná řecká, thrácká a římská jména (0 \letterpercent{}), jména nejistého původu (14,27 \letterpercent{}), beze jména (42,85 \letterpercent{}); geografická jména z oblasti Thrákie 0, mimo Thrákii 0;

\NC\AR
\HL
\HL
\stoptable

Celková produkce je ve 4. až 5. st. n. l. zhruba desetinová při porovnání s dobou největší epigrafické aktivity, tedy 2. a 3. st. n. l. Epigrafická produkce se do jisté míry vrátila na úroveň 1. či 2. st. př. n. l., kdy většina nápisů byla produkována v pobřežních oblastech a ve vnitrozemí se nápisy objevily pouze sporadicky. Většina z 35 nápisů z daného období pochází z přímořských oblastí, a to zejména z regionu Maróneie a Hérakleie, jak je patrné na mapě 6.10 v Apendixu 2. Nápisy ve vnitrozemí pocházejí převážně z urbánních center, jako je Serdica či Augusta Traiana, kde dochází k velmi omezenému přežívání epigrafické kultury.

Nejčastější formou objektu nesoucí nápis je i nadále mramorová stéla, z jiných materiálů se dochovaly dvě proklínací destičky psané na olovo, jeden nápis na mozaice a jeden nápis na nástěnné malbě uvnitř hrobky.

