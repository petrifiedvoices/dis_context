
\section[epigrafická-produkce-a-proměny-společnosti]{Epigrafická produkce a proměny společnosti}

Dochovaný epigrafický materiál, který bylo možné datovat s přesností do jednoho až dvou století, poskytuje poměrně ucelený soubor dat o vývoji společnosti antické Thrákie v průběhu více než 11 století, přičemž ale z některých staletí máme k dispozice více dochovaného materiálu. Soubor analyzovaných 2036 nápisů umožňuje sledovat vývoj epigrafické produkce v jednotlivých časových obdobích a navzájem je srovnávat se známými společensko-politickými změnami.

Podíváme-li se na celkový počet nápisů, jak ho zobrazuje graf 6.01 v Apendixu 2, skupiny nápisů datovaných s přesností do jednoho a do dvou století vykazují velmi podobné trendy. K výraznému nárůstu produkce dochází v 5. a 4. st. př. n. l. Další období růstu počtu nápisů jsou ve 2. st. n. l. a pak zejména v 3. st. n. l. Ze srovnání se známými historickými jevy a událostmi tak plyne, že období růstu epigrafické produkce mohou být spojována s dobou intenzivního působení řeckých obcí na území Thrákie v klasické době, kdy došlo k rozšíření epigrafického zvyku i mimo bezprostřední území řecký {\em poleis}, a kdy se začalo ve velmi omezené míře na nápisech objevovat i thrácké obyvatelstvo, zejména aristokratického původu. Další období růstu souvisí s dobou relativní politické stability řeckých obcí ve 2. st. př. n. l., kdy byla velká část Thrákie součástí makedonského království. Nejvýraznější nárůst epigrafické produkce nastává ve 2. a 3. st. n. l., kdy dochází k velkému množství administrativních reforem v rámci římské říše a s tím spojeným větším zapojení běžného obyvatelstva do státních a vojenských institucí.

Naopak období poklesu epigrafické produkce oproti předcházejícímu století nastalo na přelomu 4. a 3. st. př. n.l., výrazněji pak v 1. st. př. n. l. a nejstrmější propad pochází z přelomu 3. a 4. st. n. l. Doby poklesu epigrafické aktivity mohou být přičítány období ekonomické a společenské krize, vyčerpání a následné transformace, která vede i k odlišnému pojetí epigrafické produkce. K poklesu produkce dochází zejména na přelomu 1. st. př. n. l. a 1. st. n. l., kdy Thrákie prochází reformou politické moci a vazalské krále thráckého původu střídá autorita římské říše. Tento propad epigrafické produkce nicméně není tak markantní jako téměř ukončení epigrafické produkce na konci 3. st. př. n. l., kdy dochází k poklesu produkce o více než 90 \letterpercent{}, což souvisí s celkovou destabilizací poměrů v římské říši, vleklou ekonomickou krizí, zvyšujícími se nájezdy nepřátelských kmenů a v neposlední řadě i narůstající společenskou rolí křesťanské víry.

Opakující se trendy růstu v době prosperity a stability a poklesu v době nejistoty a úpadku se projevují i napříč typologickými skupinami nápisů. Změny vnitřní infrastruktury se projeví se zpožděním určité doby na celkovém počtu veřejných nápisů, které jsou produktem organizované administrativy a aktivity fungující politické autority. Jejich nárůst je patrný ve 3. st. př. n. l. jakožto důsledek zvýšených aktivit makedonských králů, avšak i přesto stále převažují nápisy soukromé povahy. Další nárůst produkce veřejných nápisů je pozorovatelný ve 2. st. n. l., kdy dochází k nárůstu regulace v souvislosti s administrací římské provincie a stavebních aktivit financovaných státním aparátem. Tento růst dále pokračuje i v 3. st. n. l., kdy se římská říše dlouhodobě potýká s krizí, nárůstem byrokratického aparátu, zvyšování moci armády a centralizací řízení rozsáhlé říše. Rostoucí komplexita politické a společenské organizace vede k nutnosti efektivněji uchovávat a předávat informace, nutné k řízení stratifikované společnosti o velikosti větší než několik set členů. Proto v komplexních společnostech dochází k vytvoření systému uchovávání a předávání informací, jehož částečným projevem jsou i dochované veřejné nápisy (Johnson 1973, 3-4). V 2. a 1. st. př. n. l. je naopak možné pozorovat snížení celkového počtu nápisů, které je nejmarkantnější ve 4. st. n. l., kdy dochází k destabilizaci politické autority, která využívala nápisy pro účely organizace a správy území a obyvatelstva. Probíhající reformy organizace říše, ekonomická a politická nestabilita vedou ve 4. st. n. l. ke konečnému úpadku byrokratického aparátu v jeho původní podobě, a tedy i k razantnímu opuštění epigrafické produkce.

Nápisy soukromé povahy tvořily většinu nápisů po celou sledovanou dobu a do jisté míry u nich probíhaly podobné změny růstu a poklesu produkce jako u veřejných nápisů, jak je patrné z grafu 6.02 v Apendixu 2. Tento jev souvisí s propojeností základní infrastruktury nutné k vytvoření nápisů, kterou sdílí jak nápisy veřejné, tak i soukromé povahy. V dobách prosperity, kdy dochází k nárůstu společenské komplexity, současně se totiž objevuje i infrastruktura nutná ke zhotovení nápisů, jakou je profesní specializace, intenzifikace výroby, navýšení gramotnosti obyvatelstva a akumulace ekonomického potenciálu osobami podílejícími se na produkci nápisů. Pokud jsou tyto podmínky naplněny, tím pádem může snadněji dojít k většímu zapojení obyvatelstva na produkci nápisů, a tedy i nárůstu produkce soukromých nápisů.

Prudké nárůsty produkce soukromých nápisů se objevují v 5. st. př. n. l., dále ve 2. st. n. l., tedy v době prosperity a stability, kdy soukromé osoby mají dostatek prostředků na zhotovení nápisu. Naopak ve 3. st. př. n. l., 1. st. př. n. l. a zejména ve 4. st. n. l. dochází k úpadku publikační činnosti soukromých osob. Tento jev je možné spojovat s obdobím společenské a ekonomické nejistoty, kdy lidé upouští od publikování nápisů a dávají přednost zajištění primárních životních potřeb. V těchto dobách také může klesat počet obyvatel schopných psát, spolu s tím, jak v dobách krize mizí i nutná infrastruktura potřebná pro vznik gramotné vrstvy obyvatel, případně pro vytváření nápisů samých (Tainter 1988, 118-123, 137).

Zajímavým jevem je současný nárůst veřejných nápisů a současný pokles nápisů funerálních, ke kterému došlo ve 3. st. př. n. l. a také ve 3. st. n. l., jak je patrné v grafu 6.02 v Apendixu 2. Možné vysvětlení tohoto trendu je krize tehdejšího politického a společenského uspořádání a snaha politické autority o reformy, navýšení regulace a kontroly byrokratického aparátu. Přímým důsledkem je zvýšené finanční zatížení středních vrstev společnosti, což v konečném důsledku vede k relativně okamžitému poklesu produkce soukromých nápisů. Ve 3. st. př. n. l. tento jev může souviset se snahou původně autonomních řeckých měst se vyrovnat s působením makedonských králů v Thrákii, které je možné datovat již od poloviny 4. st. n. l. s výraznějšími projevy zejména ve století následujícím. Ve 3. st. n. l. je to naopak reakce na krizi římské říše, způsobenou jak vojenskými převraty a konflikty, nebezpečím na vnějších hranicích říše, finanční zátěží a přebujelým administrativním aparátem (Tainter 1988, 137-140). Administrativní aparát se za každou cenu snaží udržet v chodu i za cenu zvyšujících se nákladů, které dopadají především na střední vrstvu. To může vést až ke snižování vzdělanosti a schopnosti obyvatelstva publikovat a financovat zhotovování nápisů, což má za následek úbytek soukromých nápisů ve 3. st. n. l. O století později tento trend vyústí v konečný úpadek epigrafické produkce ve veřejné sféře a přeměnu publikačních zvyků v soukromé sféře, vedoucí k přežívání epigrafické funerální produkce v křesťanských komunitách.

Trendy v chování vyšší a střední vrstvy, tedy těch, kteří si mohli dovolit publikovat nápisy, dobře dokumentují trendy zobrazené v grafu 6.03 v Apendixu 2. Po většinu doby převládaly funerální nápisy, jejichž funkce byla označovat hrob a připomínat zemřelého. Jejich celkové počty reagovaly na aktuální změny ve společnosti, tj. pokles v době krize a nárůst v době stability. Změna nastává na konci 1. st. n. l., kdy se Thrákie stává součástí římské říše a objevuje se prudký nárůst dedikačních nápisů, doprovázený pomalým poklesem funerálních nápisů. Ve 3. st. n. l. jsou dedikační nápisy dokonce četnější než nápisy funerální, což svědčí o proměnách chování tehdejší populace. S tím souvisí i nárůst epigrafické produkce související s lokálními kulty, které vykazují prvky jak místní thrácké víry, tak řeckého náboženství. Došlo tak k unikátnímu spojení a přeměně zvyků tehdejší společnosti, kdy se funkce převážné části epigrafické produkce změnila z připomínání zemřelého směrem k vyjádření náboženského přesvědčení. Ve 4. st. n. l. dochází k všeobecnému úpadku a navrácení se k funerální funkci dochovaných nápisů.

