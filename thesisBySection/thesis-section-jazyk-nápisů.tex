
\section[jazyk-nápisů]{Jazyk nápisů}

Analyzovaný soubor nápisů obsahuje převážně řecky psané nápisy, které tvoří až 98 \letterpercent{} všech nápisů. Databáze také obsahuje nápisy nesoucí zároveň řecký a latinský text (1 \letterpercent{}), a v neposlední řadě nápisy nesoucí pouze latinsky psaný text (1 \letterpercent{}).\footnote{Takto nízké číslo latinských nápisů nereflektuje jejich skutečný stav dochování, ale je výsledkem omezení daného výzkumu a zaměření práce pouze na řecky psané texty. V budoucnosti je možné práci rozšířit i o latinsky psané texty, což v současné době přesahuje možnosti a omezení dané doktorským projektem. Doplnění současného projektu o latinsky psané texty by vyžadovalo vytvoření badatelského týmu a jednalo by se o projekt dlouhodobého charakteru, jehož přínos v rámci současného stavu poznání by byl zcela jistě neoddiskutovatelný.} Nápisy, které bývají někdy označovány jako thrácké, neřadím do samostatné kategorie, vzhledem k jejich problematické povaze, ale zahrnuji je do nápisů řeckých s patřičným komentářem (Dimitrov 2009, 3-19; Dana 2015, 244-245).\footnote{Badatelé se nemohou shodnout, zda se skutečně jedná o thrácké nápisy, či pouze o nesrozumitelně psané nápisy řecké (např. Dimitrov 2009; Dana 2015). Jedná se o několik (méně než 10, konkrétní číslo je záležitostí definice „thráckého” nápisu jednotlivých badatelů) velmi krátkých textů psaných alfabétou, sestávajících z jednoho, či několika málo slov. Význam těchto nápisů zůstává nejasný, a tedy i jejich interpretační hodnota je značně omezená. I přesto jsem se rozhodla je zařadit do kategorie řeckých nápisů, avšak brát v potaz jejich speciální charakter.}

Velká část nápisů z Thrákie je psána v řečtině, a to i v době římské nadvlády. Převaha epigrafických textů v řečtině zcela odpovídá vývoji i v jiných částech Balkánského poloostrova a východních částí římského impéria obecně. Řečtina se stala oficiálním publikačním jazykem poměrně záhy po objevení prvních řeckých osídlení na thráckém pobřeží. První nápisy pocházely z čistě řeckého kontextu, nicméně v průběhu staletí se i thrácké obyvatelstvo naučilo používat alfabétu a později řečtinu jako prostředek písemné komunikace. V římské době nedošlo k zásadnímu kulturnímu přelomu v užívání publikačního jazyka a řečtina zůstala preferovaným jazykem epigrafických památek, alespoň na území provincie {\em Thracia}. Na hranicích mezi provincií {\em Thracia} a {\em Moesia Inferior} se v římské době setkávaly dvě jazykové tradice, latinská a řecká, které zásadně ovlivnily jazyk publikovaných nápisů. Přes území Balkánu vedla v době římské říše tzv. Jirečkova linie, která od sebe oddělovala území, kde byla valná většina textů psána řecky, od území s převahou latinsky psaných epigrafických památek (Jireček 1911, 36-39). Jirečkova linie procházela přibližně na místě dnešního pohoří Stara Planina a pomyslně oddělovala řecky píšící jih od latinsky píšícího severu. Přesné statistiky řecky vs. latinsky psaných textů je velmi obtížné uvést, vzhledem k nedostupnosti jednotného zdroje nápisů.\footnote{V roce 2006 začal vznikat digitální korpus nápisů z Bulharska pod patronátem Univerzity {\em Sv. Kliment Ohridski} v Sofii, který měl obsahovat zhruba 3500 řecky psaných nápisů. Bohužel i v roce 2017 je tento projekt stále ve fázi vývoje a oficiální webová stránka je veřejnosti nedostupná, http://telamon.proclassics.org/index.php (navštíveno 7. března 2017). Databáze {\em Packard Humanities Institute} s názvem {\em Searchable Greek Inscriptions} (PHI) obsahuje pouze řecky psané nápisy z území Thrákie v počtu okolo 4000 exemplářů, avšak s metadaty omezenými na dataci, jméno nálezové lokality a text nápisu. K poslední úpravě databáze došlo v září 2015, což znamená, že neobsahuje nové nápisy (\useURL[url19][http://inscriptions.packhum.org/][][{\em http://inscriptions.packhum.org/}]\from[url19], navštíveno 5. března 2017). Databáze {\em Epigraphic Database Heidelberg} (EDH) obsahuje jak latinské, tak řecké nápisy, nicméně není ani zdaleka kompletní. Ač dochází k neustálému doplňování nápisů, 5. března 2017 databáze obsahovala 397 nápisů z provincie {\em Thracia} a 1938 z provincie {\em Moesia Inferior}. (\useURL[url20][http://edh-www.adw.uni-heidelberg.de/home/][][{\em http://edh-www.adw.uni-heidelberg.de/home/}]\from[url20], navštíveno 5. března 2017). Databáze latinských nápisů {\em Corpus Inscriptionum Latinarum} (CIL) obsahuje k 5. březnu 2017 79 latinsky psaných nápisů z provincie {\em Thracia} a 387 z provincie {\em Moesia Inferior} (\useURL[url21][http://cil.bbaw.de/cil_en/dateien/datenbank_eng.php][][{\em http://cil.bbaw.de/cil_en/dateien/datenbank_eng.php}]\from[url21], navštíveno 5. března 2017).} Obecně se předpokládá, že tento poměr je pro Thrákii kolem 80:20 ve prospěch řecky psaných nápisů.\footnote{Přesné číslo latinsky psaných nápisů pocházejících z území Thrákie není známé. Jednotlivé nápisy jsou publikovány v několika zdrojích, a pravděpodobně některé nápisy zůstávají nepublikované. Odhady badatelů se tak do velké míry různí: Milena Minkova (2000, 1-7) poskytuje statistiku pouze pro území moderního Bulharska, kde předpokládá existenci zhruba 1200-1300 latinsky psaných nápisů. Nicolay Sharankov (2011, 145) uvádí, že řeckých nápisů je zhruba 20krát více než latinských, avšak tento odhad je pravděpodobně příliš nízký. Pokud vezmeme 4600 řeckých nápisů z {\em Hellenization of Ancient Thrace} databáze jako výchozí číslo, dle Sharankova by latinských nápisů bylo pouze 230. Pouze z okolí samotného města Novae pochází 447 latinsky psaných nápisů (Gerov 1989, 207).} V okolí větších měst a sídel vojenských jednotek je přítomnost latinsky psaných textů vyšší vzhledem k přítomnosti byrokratického a vojenského aparátu než na thráckém venkově, kde převládají řecky psané texty. V případě Thrákie se však latinský a řecký svět do značné míry prolínaly a docházelo k vzájemnému ovlivňování a lingvistickým výpůjčkám (Dana 2015, 253). \footnote{Agniezka Tomas (2007, 31-47; 2016) zpracovala výskyt řeckých a latinských nápisů v okolí města Novae na Dunaji v provincii {\em Moesia Inferior} na pomezí tzv. Jirečkovy linie. Novae bylo sídlem římské legie již od roku 45 n. l. a vzhledem k trvalé přítomnosti římského vojska se dá předpokládat převaha latinských nápisů nad nápisy psanými řecky. To platí pro bezprostřední okolí Novae a pro sídla ve vnitrozemí, která měla na starosti zásobování města a vojenských jednotek, jako např. {\em villa} v Pavlikeni. Ve venkovských sídlech se však objevují i nečetné řecky psané nápisy, a to zejména ve svatyních poblíž vesnic Paskalevec, Butovo a Obedinenie (Tomas 2007, 44). V okolí města Nicopolis ad Istrum, které se nachází zhruba 80 km jihovýchodně, je již převaha dochovaných nápisů psaná řecky, a to jak z města, tak z venkovských oblastí.}

