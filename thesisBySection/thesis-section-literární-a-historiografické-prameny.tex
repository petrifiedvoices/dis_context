
\section[literární-a-historiografické-prameny]{Literární a historiografické prameny}

Literární zmínky o Thrákii a Thrácích jsou však relativně hojné, vzhledem k tomu, že se jednalo o obyvatele nepříliš vzdáleného území, v nichž měli Řekové a později Římané eminentní zájmy, a mnoho Thráků žilo a sloužilo v řeckých městech (Xydopoulos 2010; Janouchová 2013) a v římské armádě (Dana 2013; Boyanov 2008; Boyanov 2012, 251-269).Literární prameny ve většině případů nezaměřují výhradně na popis Thrákie a jejích obyvatel, ale zmiňují se o oblasti spíše v rámci širšího výkladu, pro ilustraci celkového děje. I přes svou limitovanou výpovědní hodnotu měly literární prameny zásadní formativní vliv na tzv. hellénizační interpretační rámec, který sloužil jako nejčastěji používané vysvětlení společenských změn při kontaktu s řeckou kulturou (Dietler 2005, 33-47; Jones 1997, 33). Z tohoto důvodu věnuji obrazu Thrákie a Thráků v literárních pramenech poměrně velkou pozornost, ale stejně tak i charakteru těchto pramenů a jejich možné výpovědní hodnotě pro studium mezikulturních kontaktů.

