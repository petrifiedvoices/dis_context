
\section[nastínění-dalšího-vývoje]{Nastínění dalšího vývoje}

V rámci studia proměn thrácké společnosti a epigrafické produkce v Thrákii se nabízí rozšíření výzkumu na latinsky psané nápisy z oblasti, případně rozšíření zkoumaného území na oblasti severně od Dunaje či do oblasti Bíthýnie. V případě zapojení latinských nápisů se nabízí srovnání přístupů společnosti ve vztahu ke zvolenému publikačnímu nápisu a konkrétní podobě a obsahu nápisů. Zajímavé by mohlo být i srovnání přístupu obyvatelstva dle etnicity či společenského postavení a volby publikačního jazyka. Zapojení nápisů z oblasti severně od Dunaje a z oblasti maloasijské Bíthýnie, tedy regionů do jisté míry taktéž obývaných thráckými kmeny, by pomohlo zhodnotit nakolik jsou si obyvatelé podobní, jednak co se týče přístupu k epigrafické kultuře v průběhu staletí, ale i nakolik se podobají složením společnosti a vývojem společenského uspořádání.

Velmi zajímavým rozšířením této práce by bylo srovnávat epigrafická data s archeologickými daty z téhož regionu. Pokud bude v budoucnosti dostupný zdroj koherentních archeologických dat, toto srovnání by do velké míry přispělo k naší znalosti o složení a proměnách společenského uspořádání a zvyklostí obyvatelstva antické Thrákie.

