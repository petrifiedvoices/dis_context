
\section[rozmístění-nápisů-v-závislosti-na-jejich-typologii]{Rozmístění nápisů v závislosti na jejich typologii}

Rozmístění nápisů a jejich zařazení dle společenské funkce přináší nový pohled na roli, jakou nápisy v antické Thrákii hrály. Rozmístění veřejných nápisů odpovídá roli, jakou hrály byrokratické instituce a infrastruktura vytvářená komplexní společností v době existence římské provincie. Soukromé nápisy mohou existovat nezávisle na tomto uspořádání, nicméně k jejich rozšíření ve větším měřítku dochází právě v koexistenci s prostředím raného státu s centralizovanou mocí a organizovaným řízením. Proto se i soukromé nápisy nacházejí v okolí center či oblastí se zvýšenou mírou společenské a politické provázanosti jakou obyvatelům poskytují města. Naopak venkov je epigraficky málo aktivní, a to i v římské době, a aktivity souvisí spíše s iniciativou jedinců než obecným přístupem venkovské populace.

