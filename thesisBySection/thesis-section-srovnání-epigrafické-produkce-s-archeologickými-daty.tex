
\section[srovnání-epigrafické-produkce-s-archeologickými-daty]{Srovnání epigrafické produkce s archeologickými daty}

Ač se na první pohled může zdát, že dochované nápisy nám poskytují informace o interakci jednotlivých komunit v plném rozsahu, opak je často pravdou a dochování nápisů může do velké míry dílem náhody. Nápisy se dochovaly pouze z části tehdy existujících lokalit, a dokonce i ne ze všech řeckých kolonií té doby. Archeologické výzkumy neprobíhají na všech místech stejnou měrou a některé lokality jsou lépe prozkoumané. Pokud by produkce nápisů byla rovnoměrná po celém území, pak by z lépe prozkoumaných lokalit mělo také pocházet více nápisů, čemuž tak je pouze v určitých případech.\footnote{Např. v případě Apollónie Pontské ve 4. st. př. n. l. či v Byzantiu ve 2. a 1. st. př. n. l., více v sekcích věnovaných jednotlivým stoletím v kapitole 6.} Jaký je poměr epigraficky aktivních měst vůči těm, které nápisy neprodukovaly? Nebo jinými slovy, na kolik jsou nápisy relevantním historickým zdrojem pro studium společnosti jako celku? Co tedy stojí za faktem, že z určitých lokalit pochází velké množství nápisů a z podobných lokalit, které jsou i do stejné míry prozkoumané, se dochovalo nápisů velmi málo či dokonce žádné? Na tyto otázky se pokusím alespoň nastínit odpověď na dvou příkladech srovnávajících epigrafická a archeologická data z území Thrákie.

Zásadním problémem, na nějž toto a podobná srovnání naráží, je nekompletní povaha archeologických, ale i epigrafických dat. Naše současné znalosti postihují pouze zlomek tehdy existujících lokalit, který je podobně nahodilý jako v případě nápisů. Do značné míry tak srovnání založené na souborech archeologických a epigrafických dat může být nepřesné a částečně zkreslené charakterem dostupných dat. Proto v žádném případě nelze zde uváděná čísla brát jako kompletní a neměnná, ale spíše jako orientační a shrnující aktuální stav našich znalostí. Hlavním cílem tohoto srovnání je ukázat, že epigraficky aktivní byla pouze velmi malá část tehdejší společnosti a nápisy pocházejí pouze ze zlomku tehdy známých lokalit, ač by jejich relativně velké počty na první pohled mohly svědčit o opaku.

