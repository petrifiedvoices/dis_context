
\section[zhodnocení-vlivu-hellénizace-na-epigrafickou-produkci-v-thrákii]{Zhodnocení vlivu Hellénizace na epigrafickou produkci v Thrákii}

V této práci jsem na základě studia dochovaného epigrafického materiálu došla k závěru, že v dnešní době je teoretický koncept hellénizace do značné míry překonaný a jako interpretační rámec vysvětlující rozšíření epigrafické produkce v Thrákii je nedostačující. Hellénizační přístup totiž představuje jednostranně zaměřený model, zatížený množstvím předsudků, který nereflektuje pestrost mezikulturních kontaktů v celé jejich šíři, a nenabízí prostor pro aktivní zapojení thrácké populace v celém procesu kulturní změny. V duchu hellénizace je rozšíření epigrafické produkce v Thrákii interpretováno jako bezvýhradné přijetí zvyku publikovat nápisy místním obyvatelstvem se současným opuštěním vlastní identity a její postupné nahrazení identitou řeckou. Pokud by došlo k úplné a bezvýhradné hellénizaci, tak jak tento model navrhuje, všechny nápisy by byly produkovány Thráky, kteří by postupem času přijali i řeckou identitu a řecká jména a stali se tak v epigrafickém prostředí nerozlišitelnými od Řeků. Postupem času by thrácký prvek zcela vymizel nejen z onomastických záznamů, ale i z vyjádření příslušnosti k místním komunitám a projevům víry typických pro thrácké obyvatelstvo. Thrácká identita, spolu s jmény, by tak v pozdních fázích hellénizace neměla na nápisech vůbec figurovat. Univerzálním motivem rozšíření nápisů by byla přirozená touha po všem řeckém, po řecké kultuře a civilizaci. Neexistovaly by lokální varianty, místa s větší či menší mírou rezistence. Nápisy by byly rozmístěny rovnoměrně po celém území a zachovávaly by si uniformní charakter a formu. Jak je patrné z analyzovaného materiálu v jednotlivých časových obdobích a specifických regionech, tento uplatněný hellénizační model neodpovídá skutečné situaci. Naopak, prvky dokumentující existenci místní identity se vyskytují po celou dobu existence epigrafických památek na území Thrákie, a ve 2. a 3. st. n. l. dochází k intenzifikaci tohoto trendu. Řecká identita na epigrafických památkách tak nenahrazuje přináležitost k thrácké komunitě, ale jsme svědky adaptace nového společenského uspořádání a zvyklostí na všech zúčastněných stranách probíhající mezikulturní výměny. Jak dokazuje podrobná analýza nápisů, zvyk vytvářet a publikovat nápisy byl přijat různými vrstvami a skupinami společnosti odlišně, v různou dobu a s různorodými projevy, a proto je nutné jej jednoznačně interpretovat jako důsledek uniformního vlivu řecké kultury a civilizace.

Nápisy a písmo obecně v tehdejší společnosti sloužily jako prostředek vyjadřování a zaznamenávání informací, který se rozšířil z řeckých měst na pobřeží směrem do progresivních komunit ve vnitrozemí v klasické a hellénistické době, avšak k jeho rozšíření do prakticky všech komunit došlo až pod vlivem sjednocující autority římské říše. Pouhá kulturní dominance řeckého světa tak nemůže sloužit jako jediné vysvětlení epigrafické produkce v Thrákii, ale spíše je vhodné na rozšíření nápisů nahlížet jako na jeden z průvodních jevů rozvíjející se společensko-politické organizace a struktury.

