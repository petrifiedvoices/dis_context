
\section[zvolená-metoda]{Zvolená metoda}

Při analýze materiálu se držím principu falzifikovatelnosti a testovatelnosti použitých metod, ve snaze ověřit či vyvrátit z nich vycházející interpretace (Popper 2005, 57-73). Abych mohla nápisy z různých zdrojů a různé povahy vzájemně srovnávat, musela jsem přistoupit k jejich kategorizaci na základě interpretace jejich obsahu a formy. Tato míra jisté abstrakce umožňuje zjednodušený a kvantifikovatelný náhled do jednotlivých komunit s cílem je vzájemně porovnat v rámci několika časových úseků (O'Shea a Barker 1996, 13-19; Bodel 2001, 80-82). Při interpretaci nápisů se držím tradičních principů a kategorií řecké a latinské epigrafiky, stanovených již autory {\em Inscriptiones Graecae} a {\em Corpus Inscriptionum Latinarum}. Tyto principy, ač vznikly před více než 200 lety, tvoří i dnes pevné základy epigrafické disciplíny (Bodel 2001, 153-174; McLean 2002, 22, 181-182).

Pro usnadnění zpracování velkého množství nápisů jsem vytvořila elektronickou databázi, v níž jsem shromáždila přes 4600 nápisů z oblasti Thrákie. K analýze tak velkého množství nápisů jsem využívala moderní metody a nástroje, známé spíše z přírodních věd, z archeologie a z oblasti tzv. {\em digital humanities} (Bodel 2012, 275-293). Konkrétní podoba užitých postupů je však přizpůsobena specifikům epigrafického materiálu, a je reakcí na nutnost analyzovat množství materiálu s poměrně velkým počtem informací nejisté povahy, jako je například datace či umístění nápisu. Do velké míry se však jedná o inovativní přístup, který staví na několika pilotních studiích a kombinuje dohromady přístupy několika disciplín (Benefiel 2010, 45-65; Feraudi-Gruenais 2010; 14-16; Witschel 2010, 77-86; Janouchová 2014).

Ke studiu společnosti antické Thrákie na základě studia dochovaných nápisů přistupuji na třech vzájemně provázaných úrovních: a) na úrovni jednotlivých nápisů a funkce, jakou ve společnosti zastávaly; b) na úrovni společenských trendů; c) a na úrovni epigrafických produkčních center. Zajímají mě nejen geografické vzorce a relativní rozmístění nápisů vůči kulturním centrům a komunikacím, ale i všeobecné proměny společnosti v závislosti na známých společensko-politických událostech.

