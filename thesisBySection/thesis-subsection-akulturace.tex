
\subsection[akulturace]{Akulturace}

Pojem akulturace nikdy nepředstavoval jeden ucelený model, ale spíše teoretický směr vysvětlující a popisující proces kulturní změny a důsledky kontaktu odlišných kultur, a to jak společensky, kulturně, tak materiálně (Redfield {\em et al}. 1936, 149)\footnote{Redfield {\em et al.} 1936, 149: „{\em Acculturation compherehends those phenomena which result when groups of individuals having different cultures come into continuous first-hand contact, with subsequent changes in the original patterns of either or both groups}”.}. Akulturace se stala jedním z hlavních interpretačních přístupů pojednávající o kulturních změnách v první polovině 20. století, tedy doby silně ovlivněném evropským a americkým kolonialismem a imigrací (Cusick 1998, 24). Akulturace našla své uplatnění v psychologii, sociologii, či archeologii, vždy se ale jednalo o vysvětlení procesu změny u jedince či ve společnosti na základě kontaktu s jinou kulturou.

Akulturační teorie předpokládá adopci či úplné nahrazení kulturou s propracovanější společenskou strukturou v prostředí méně vyvinuté kultury. Evropská, či západní, kultura, a tedy i kultura starověkého Řecka a Říma, bývá v rámci akulturačních teorií chápána jako nadřazená kultuře původní, a ta je naopak chápána pouze jako pasivní příjemce a kultura méně vyspělá. Reakce na mezikulturní kontakt bývají v rámci akulturační teorie interpretovány různě, počínaje úplnou adopcí nové kultury a odvržením staré, adaptací a splynutím s původní kulturou, úplným odmítnutím, či tzv. {\em code-switching}, čili používáním jedné či druhé kulturní normy dle aktuální situace (Cusick 1998, 29).\footnote{V publikaci {\em Memorandum for the Study of Acculturation} z roku 1936 (Redfield {\em et al.} 1936, 149-152) autoři sumarizují tři možné druhy reakcí přijímající kultury jako následující: a) přijetí nové kultury za částečné či úplné ztráty kultury původní, b) adaptaci nové kultury a spojení s kulturou původní, c) reakci a odmítnutí nové kultury.}

Autoři zmíněného {\em Memoranda} (Redfield {\em et al.} 1936, 150-151) poukazují na možné způsoby studia akulturace. Jako první bod uvádějí určení druhu a charakteru kontaktů mezi oběma kulturami\footnote{Zda kontakty probíhají mezi celými skupinami, či vybranými jednotlivci se speciálním posláním, zda jsou kontakty přátelské či nepřátelské, zda probíhají mezi skupinami stejné velikosti, na stejném stupni společenské komplexity, zda probíhají na území obývané jednou ze stran, či na zcela novém území.}, dále analýzu situací, v nichž ke kontaktům dochází\footnote{Zda byly elementy nové kultury představeny silou, či se jednalo o dobrovolnou akci, zda mezi oběma stranami panovala společenská nerovnost, a pokud existovala, zda se jednalo o politickou či společenskou dominanci jedné strany.}, zhodnocení samotného procesu akulturace\footnote{Jaké prvky společnosti či jedince se proměňují ve vztahu k druhu kontaktu a vázané na konkrétní situaci; dále k jakým změnám dochází pod nátlakem a jaké prvky jsou odmítnuty a z jakého důvodu. Dalším bodem je oboustranné zhodnocení praktických výhod akulturace, např. ekonomického prospěchu či politické dominance, a jiných motivů, jako je např. získaní společenské prestiže, či návaznost na prvky v kultuře již dříve existující.} a nakonec integraci nových kulturních prvků do původní kultury.\footnote{Autoři hodnotí dobu, jaká uplynula od představení prvku, obtížnost přijetí nového prvku v rámci původní kultury, dále míru nutnosti přizpůsobení původní kultury novému uspořádání.} Autoři v {\em Memorandu} (Redfield {\em et al.} 1936) nastínili základní metodologii a přístupy ke studiu změn ve společnosti, které jsou přínosné i v dnešní době, avšak i přesto zůstává akulturační model příliš obecný a jednostranně zaměřený pouze na změnu kultury příjemce. Mezikulturní kontakty probíhají však oběma směry a dochází ke změnám v kultuře obou zúčastněných stran, nikoliv pouze v kultuře s méně propracovaným systémem ideových hodnot. Tento jednostranný a částečně diskriminující přístup byl kritizován celou řadou vědců pro své zjednodušování skutečnosti a opomíjení skutečného stavu, kde se kultury ovlivňují navzájem, a proto se dnes od jeho užívání upouští (Cusick 1998, 23).

\subsubsection[akulturace-v-archeologii-a-historických-vědách]{Akulturace v archeologii a historických vědách}

Uplatnění akulturačního přístupu v archeologii a historických vědách se do velké míry zaobírá pouze adopcí a adaptací materiální kultury. Přítomnost materiálních projevů typických pro jednu kulturu mimo obvyklé území bývá pak často vysvětlována právě akulturací tamního obyvatelstva; jinými slovy čím větší množství cizího materiálu je nalezeno, tím větší bývá přisuzován akulturační vliv příchozí kultury na tamní obyvatelstvo. Často bývají změny v materiální kultuře vykládány pouze jako jevy vedoucí ke změně identity osob, ač existují i jiná možná vysvětlení, jako například změna trendu a poptávky nebo změna technologie (Cusick 1998, 31). Materiální památky samy o sobě většinou neposkytují dostatek informací k odlišení akulturace, tedy adopcí či adaptací cizí kultury, a pouhého používání předmětů pocházejících z jiné kultury, a proto se upouští od užívání akulturace jako interpretačního rámce kulturního kontaktu (Jones 1997, 29-39).\footnote{Terminologie používaná pro vysvětlení akulturačních procesů se stala natolik běžnou, že je mnohdy užívána bez přímé souvislosti s akulturačním přístupem či bez reflexe a znalosti problematiky.}

