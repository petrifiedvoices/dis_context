
\subsection[charakteristika-obyvatel-thrákie-v-literárních-pramenech]{Charakteristika obyvatel Thrákie v literárních pramenech}

První zmínky o Thrácích jako o spojencích pocházejí již z homérských eposů, kde thráčtí králové vystupují po boku řeckých panovníků.\footnote{Nejslavnější polo-mýtický král Rhésos byl znám díky svým pověstným bílým koním a nádherné zbroji (Hom. {\em Il}. 10.435, 495), které se lstí podaří získat Odysseovi.} V archaické literatuře se zmínky o Thrákii týkají spíše její geografie a o jejích obyvatelích se mluví vždy v souvislosti s krajinou, kterou obývají.\footnote{Do této kategorie spadá např. Zmínky v homérských eposech, homérských hymnech, Hésiodovi, Hekatáiovi, Archilochovi, Simonidovi a v díle dalších archaických básníků (Xydopoulos 2004, 18-20).} Od klasické doby se pak setkáváme se dvěma hlavními tématy prolínající se téměř všemi dochovanými literárními prameny: Thrákové jsou vnímáni jednak jako blízcí sousedé a často i spojenci, kteří se podobají společenským uspořádáním a zvyklostmi dávné řecké minulosti, a dále jako krutí a bojovní válečníci, s nimiž není radno přijít do sporu (Xydopoulos 2005, 600; Sears 2013, 148). Z těchto důvodů často hráli nezanedbatelnou roli v politickém vývoji severního Egejdy jako spojenci Athén. Zejména v 5. a 4. st. př. n. l. byli panovníci z thráckého kmen Odrysů vnímáni jako mocní spojenci, a to pro velikost jejich vojska, bohatství, kterým disponovali, a výhodnou geografickou pozici jimi ovládané části Thrákie.

