
\subsection[dedikační-nápisy-15]{Dedikační nápisy}

Dedikačních nápisů se ze 3. st. n. l. dochovalo 128, což představuje zhruba trojnásobný nárůst oproti 2. st. n. l. Dedikační nápisy i nadále převládají nad nápisy funerálními, podobně jako u skupiny nápisů datovaných do 2. až 3. st. n. l.\footnote{Na konci 3. st. n. l. opět však dochází ke změně a celkový počet funerálních nápisů opět převládá nad dedikacemi.} Pokračujícím fenoménem ve 3. st. n. l. jsou velké svatyně v blízkosti velký měst či v podhorských oblastech v dostupnosti cest, které se staly populární jak mezi Thráky, tak i lidmi nesoucí jiná než thrácká jména.\footnote{Takřka polovina nápisů pochází z jedné svatyně v regionu města Serdica. Dalších 18 nápisů pochází z několika míst z regionu Augusty Traiany, a dalších devět nápisů z údolí středního toku Strýmónu.} Zvyk věnovat nápisy nebyl ve 3. st. n. l. omezen pouze na jednu část společnosti, ale podíleli se na něm lidé jak thráckého, tak jiného původu a různého společenského postavení.

Věnování jsou určena zejména Asklépiovi ve 27 případech, Diovi ve 12 případech, Apollónovi v pěti případech, Áreovi ve dvou případech. Tato božstva nesla většinou místní epiteton, což naznačuje pokračující propojení místních kultů s původně řeckými kulty, stejně jako v 2. st. př. n. l.\footnote{Mezi tato božstva patří např. Zeus {\em Zbelthiúrdos}, Zeus {\em Paisoulénos}, Apollón {\em Dortazénos}, Asklépios {\em Kúlkússénos}, Árés {\em Saprénos}, dále {\em theos} {\em Salénos}, {\em Aularchénos}, {\em hérós} {\em Tisasénos} a {\em Marón}, {\em theos} {\em Asdúlos}.} Ve většině případů se jedná o prostou dedikaci bez většího množství detailů, ale zhruba čtvrtina nápisů poskytuje detailnější informace o dedikantech. Dedikace ve 14 případech zhotovili vojáci, ve třech případech členové {\em búlé}, ve dvou {\em thrakarchové} a v jednom případě {\em gymnasiarchés} a {\em archón} v jedné osobě. V jednom případě se jedná o hromadnou dedikaci {\em saltariů}, tj. římské verze polesných a lidí starajících se o lesní porost (Mihailov 1966, 275). Osobní jména poukazují na převahu původně římských jmen.\footnote{Celkem se dochovalo 185 jmen, z čehož zhruba polovina je římského původu, 23 \letterpercent{} řeckého původu, 18 \letterpercent{} thráckého původu a 10 \letterpercent{} jmen nebylo možné přesněji určit.} Dedikanti thráckého původu se vyskytovali téměř na pětině nápisů, což je zhruba o 8 \letterpercent{} více než ve století předcházejícím.\footnote{V kombinaci s římským jménem se thrácké jméno objevilo na 13 nápisech, v kombinaci s řeckým na devíti nápisech, a samostatně stojících na devíti nápisech.} Nápisy s thráckými jmény pocházely výhradně z vnitrozemí z okolí městských center: 11 z regionu Augusty Traiany, čtyři z regionu města Serdica, čtyři z regionu Filippopole. Celkem v šesti případech se jednalo o věnování vojáka či zastupitele zastávajícího vysokou pozici v administrativním aparátu provincie. Deset dedikačních nápisů s thráckým jménem neslo dekoraci v podobě jezdce na koni, prvkem tradičně spojovaným právě s thráckou populací.

Již na přelomu 2. a 3. st. se objevují nové motivy reliéfní dekorace, jako jsou například výjevy znázorňující Asklépia, Héraklea, Dia a Héru, Dionýsa. Nejčastěji se opakujícím motivem je však jezdec na koni, který v mnoha případech může být spojován s fenoménem tzv. thráckého jezdce (Kazarow 1938; Dimitrova 2002; Oppermann 2006).\footnote{Jezdec na koni se objevil celkem 55krát, z čehož 44 nápisů pochází ze svatyně Asklépia {\em Liménia} ze Slivnice v okolí města Serdica, dalších šest ze svatyně u vesnice Viden v regionu města Augusta Traiana (Boteva 1985; Tabakova-Tsanova 1961).} Celkem 30 nápisů s dekorací jezdce na koni nese osobní jména: deset z nich obsahuje thrácká jména samostatně stojící či v kombinaci, a 20 obsahuje jiná než thrácká jména. Nelze tedy tvrdit, že dedikace s vyobrazením jezdce na koni byla výsadně záležitostí thrácké populace. Naopak, tzv. thrácký jezdec se stal ve 3. st. n. l. populární i mezi lidmi nesoucí řecká a římská jména. Celkem u osmi nápisů o sobě dedikant přímo uvádí, že se jedná o vojáka, což představuje zhruba 15 \letterpercent{} všech dedikací nesoucích motiv jezdce na koni z daného období a nelze tedy s jistotou tvrdit, že fenomén thráckého jezdce byl rozšířen především v komunitě vojáků thráckého původu, nicméně byl mezi vojáky oblíben (Boteva 2005, 204).

Jako ilustrativní příklad složení epigraficky aktivní populace a náboženských zvyklostí 3. st. n. l. mohou sloužit nálezy ze svatyně Asklépia {\em Liménia} ze Slivnice. Tato svatyně patří k nejvýznamnějším svatyním, alespoň co do počtu nalezených nápisů. Bylo zde nalezeno celkem 69 dedikací nesoucích nápis a 286 anepigrafických votivních předmětů (Boteva 1985, 31; Mihailov 1997, 318).\footnote{Celkem 62 z těchto dedikací splňovalo chronologická kritéria a bylo datováno s přesností do jednoho až dvou století.} Celkem 24 nápisů bylo určeno Asklépiovi, z toho 14 s přízviskem {\em Liménios}, ve čtyřech případech označený jako {\em theos}, v sedmi jako {\em kyrios} a ve jednom jako {\em sótér}, zachránce. Velká část dedikací, konkrétně 44, nese vyobrazení jezdce na koni, pouhé tři nápisy nesou reliéfní zobrazení Asklépia, Hygiei a Télesfora. Pokud jde o identitu dedikantů, zdá se, že tato lokalita byla navštěvována obyvatelstvem nesoucím řecká i thrácká jména, ale i římskými občany a vojáky.\footnote{Celkem 29 nápisů nese osobní jména: osm řeckých, čtyři thrácká, 21 římských a 11 bez přesného určení. Římská jména se vyskytovala samostatně na osmi nápisech a v kombinaci na 13 nápisech. V 11 případech se opakuje římské jméno Aurelios v kombinaci s řeckým či thráckým jménem, což značí že dedikanti přijali po roce 212 n. l. toto jméno, co by znak římského občanství. Nápisy věnovali vojáci v šesti případech, což představuje pouze 9,5 \letterpercent{} dedikací nesoucích nápis z této lokality.} Předměty nesoucí nápisy představují pouze pětinu dochovaných votivních předmětů, což značí že zvyk věnovat nápisy nebyl zcela běžnou součástí rituálu, ale spíše ojedinělou záležitostí. Větší zapojení thrácké populace je pravděpodobně důsledkem nárůstu gramotnosti mezi thráckou populací a proměny funerálních zvyklostí v souvislosti s vojenskou či civilní službou Thráků.

