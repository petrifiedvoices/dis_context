
\subsection[dedikační-nápisy-4]{Dedikační nápisy}

V tomto období se poprvé objevily i tři dedikační nápisy ve vnitrozemí, z celkově čtyř dochovaných nápisů. Tyto nápisy nalezené ve vnitrozemí pocházejí komunit tradičně označovaných jako řecké, jako je tomu v případě nápisu z Pistiru či řecko-thrácké se silnými řeckými konotacemi a aristokratickými vazbami na řeckou kulturu, v případe nápisu ze Sborjanova či z okolí Seuthopole.\footnote{Seuthopolis byla sídlem odryského panovníka Seutha III., kterou si nechal postavit na březích řeky Tonzos v dnešním Kazanlackém údolí ve střední Thrákii. Jednalo se o opevněné sídlo o velikosti 4 ha, postavené dle vzorů rezidencí hellénistických vladařů. Byly zde nalezeny domy řecké typu, pravoúhlé uspořádání domů a ulic, množství řeckých importů, graffit a mimo jiné i řecky psaný nápis, tzv. seuthopolský nápis {\em IG Bulg} 3, 2 1731, o němž podrobněji hovořím níže (Dimitrov a Chichikova 1978; Čičikova 1984; Domaradzka 2005, 299). Nedaleko vesnice Sborjanovo v severovýchodním Bulharsku se našlo sídlo gétských panovníků, často označované jako hlavní město kmene Getů a ztotožňované se sídlem panovníka Dromichaita, Chelis (Stoyanov 1997; 2001, 207-219). I odsud pocházejí graffiti s thráckými jmény (Domaradzka 2005, 298) a dedikační nápis {\em SEG} 55:739.} Nápis {\em SEG} 55:739 z lokality poblíž moderního Sborjanova obsahoval tradiční formuli dedikační nápisů ({\em euchén}) a byl věnován bohyni Fosforos, která je nejčastěji spojována s Artemidou.\footnote{V dalším thráckém městě Kabylé se bohyně Fosforos stala dokonce patronkou města a hlavním vyobrazením na mincích ražených v hellénismu v Kabylé (Janouchová 2013, 103-104). Kult Artemis {\em Fosforos} je znám i z Byzantia, kde se konaly slavnosti Bosporií, doprovázené průvodem s pochodněmi, podobně jako ve 4. st. př. n. l. v Athénách (Janouchová 2013, 97; Lajtar 2000, 39-41: {\em IK Byzantion} 11, nedatovaný nápis z Byzantia).} Zbylé texty neobsahují věnování božstvu, ale pouze krátkou identifikaci dedikanta.\footnote{Dochovaná jména jsou pouze řeckého původu, a to jak v případě jmen dedikantů, tak i jejich rodičů. Krátké texty jsou psány řecky a nevykazují žádné odchylky a nepravidelnosti v užití řečtiny.} Ač byly tyto dedikační nápisy nalezeny v thráckém vnitrozemí, vše nasvědčuje na udržení kontinuity řeckých náboženských tradic a nedokazuje ovlivnění místními thráckými náboženskými představami.

