
\subsection[dedikační-nápisy-6]{Dedikační nápisy}

Zvyk věnovat stély s nápisy se i nadále na přelomu 3. a 2. st. př. n. l. vyskytoval v řeckých komunitách na pobřeží. Celkem se dochovalo šest dedikačních nápisů, které pocházely z území řeckých měst a věnování provedly osoby nesoucí téměř výhradně řecká jména. Věnování byla určena božstvům řeckého a egyptského původu, jako je Ísis, Sarápis a Anúbis.\footnote{Věnování Ísidě, Sarápidovi a Anúbidovi na nápise {\em IG Bulg} 1,2 322ter z černomořské Mesámbrie, a Ísidě a Afrodíté na nápise {\em Perinthos-Herakleia} 42 z Perinthu.} Rozšíření kultu egyptských božstev v Thrákii bývá vysvětlováno zvýšenou přítomností hellénistických vojsk a politického vlivu Ptolemaiovců v oblasti Perinthu (Tacheva-Hitova 1983, 54-58; Barrett a Nankov 2010, 17). I nadále nemáme důkazy o zapojení thráckého obyvatelstva a thráckého náboženství do procesu epigrafické produkce.

