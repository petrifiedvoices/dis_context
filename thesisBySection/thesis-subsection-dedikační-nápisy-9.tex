
\subsection[dedikační-nápisy-9]{Dedikační nápisy}

Celkem se dochovalo 11 dedikačních nápisů, které pocházejí převážně z Mesámbrie, Maróneie a Byzantia, s výjimkou jednoho nápisu z lokality Didymoteichon na dolním toku řeky Tonzos. Čtyři nápisy byly věnované egyptským božstvům Sarápidovi, Ísidě, Anúbidovi a Harpokratiónovi. Egyptské kulty se objevují v Byzantiu, Maróneii a Mesámbrii, tedy ve všech hlavních produkčních centrech té doby, což poukazuje důležitost nejen pro jako místa setkávání kultur, ale i regionální centra té doby. Dále se objevila věnování vždy po jednom nápisu Diovi {\em Aithriovi}, Athéně a Neikonemesis {\em Sóteiře} a Hérakleovi {\em Sótérovi}.

Osobní jména dedikantů poukazují na jejich řecký původ, nicméně římský prvek začíná hrát důležitou roli i na dedikačních nápisech.\footnote{Celkem 26 jmen mělo řecký původ, 11 římský, dva thrácký a sedm nebylo možné s jistotou určit. Nejvíce jmen se vyskytlo na nápise {\em IK Byzantion} 19 z Byzantia, kde je možné napočítat až 26 jmen různého původu: jedná se o věnování Diovi {\em Aithriovi} obyvateli neznámé vesnice, kde funkce kněžích zastávají muži se třemi římskými jmény ({\em tria nomina}). Zhruba polovina obyvatel na této dedikaci má taktéž římské jméno, které je v několika případech kombinované se jménem řeckým (Lajtar 2000,50-51).} Na míru zapojení thráckých elit do praxe věnování nápisů božstvům, tedy zvyklostí do této doby omezené převážně na řeckou komunitu. Nápis {\em I Aeg Thrace} 458 věnovaný původně řeckému božstvu Hérakleovi Sótérovi z lokality Didymoteichon věnoval thrácký král Kotys, syn Rháskúporida mezi lety 42 a 31 př. n. l. \footnote{Tento nápis je na pomezí soukromého a veřejného nápisu, vzhledem k tomu, že byl věnován jménem krále Kotya, tedy svrchované politické autority.}

