
\subsection[faktory-ovlivňující-rozmístění-nápisů]{Faktory ovlivňující rozmístění nápisů}

Rozmístění nálezových míst nápisů ovlivňují v první řadě geografické podmínky. Pokud se podíváme na mapu nálezových míst a jejich pozici v krajině, téměř tři čtvrtiny nápisů se nalézají v nížinách do 249 m. n. m. Na úpatí hor a v nižších horských polohách se nachází něco málo přes 20 \letterpercent{} nápisů. Zbylých 6 \letterpercent{} nápisů pochází z horských oblastí zejména v severovýchodní části Thrákie. Obecně je v horských oblastech méně nápisů než ve vnitrozemí, nicméně podél toků řek se nápisy vyskytují i v polohách nad 1000 m. n. m. Jak je dále patrné z mapy 7.01 v Apendixu 2, nápisy se mají tendence shlukovat jednak na mořském pobřeží a ve vnitrozemí v okolí velkých řek, které v antice sloužily jako hlavní komunikační tepny (Bouzek 1996, 221-222; Bravo a Chankowski 1999, 310-311; Archibald 2002).

Zeměpisné podmínky však nejsou jediným faktorem ovlivňujícím rozmístění nápisů. Pravděpodobně ještě důležitější roli hraje umístění lidských sídel, případně míst určených pro vybranou aktivitu, jako např. pohřebiště či svatyně. Z mapy 7.02 v Apendixu 2 je patrné, že velká část nápisů se nachází ve vzdálenosti 20 km od města, tedy ve vzdálenosti, kterou bylo možné ujít v jednom dni. V okruhu do 20 km od města se nachází 67 \letterpercent{} nápisů. Zbývajících 33 \letterpercent{} pochází z oblastí vzdálených od města více než 20 km. Pokud okruh okolo města rozšíříme na délku maximálního denního pochodu, do vzdálenosti 40 km od města spadá 83,5 \letterpercent{} nápisů. Zbývajících 16,5 \letterpercent{} nápisů se nachází ve větší vzdálenosti než 40 km od regionálních produkčních center.

Důležitou roli hraje nejen vzdálenost od nejbližšího města, ale i pozice vůči cestám, jak je dobře vidět na mapě 7.03 v Apendixu 2. Jako komunikace byly v antice využívány i velké řeky, které byly pravděpodobně splavné minimálně na některých úsecích. Cesty existovaly již v předřímských dobách, ale k jejich systematické stavbě, rozšiřování silniční sítě a údržbě docházelo až od 1., ale zejména ve 2. a 3. st. n. l., jak dosvědčují dochované milníky či související stavby (Madzharov 2009, 29-40). Z blízkosti několika kilometrů od cest pocházela převážná část nápisů: ve vzdálenosti do 20 km od cest se našlo 95 \letterpercent{} nápisů, nad 20 km pak zbývajících 5 \letterpercent{} nápisů. Ve vzdálenosti do 10 km od cesty se našlo 83 \letterpercent{} nápisů, nad 10 km pak 17 \letterpercent{} nápisů. Ve vzdálenosti do 5 km od cest bylo lokalizováno 75 \letterpercent{} nápisů, 25 \letterpercent{} pak ve vzdálenosti větší než 5 km.\footnote{Se zmenšující se vzdáleností od cesty se zmenšovala i počet nápisů. Jinými slovy, oblast do 5 km okolo cest obsahovala méně nápisů než oblast pokrývající území v okruhu 10 či 20 km okolo cesty.} Z toho vyplývá, že více jak dvě třetiny nápisů byly nalezeny v bezprostředním okolí cest. Tato existující infrastruktura v podobě silnic do velké míry usnadňovala pohyby nejen vojsk, ale i běžného obyvatelstva, které tak mohlo např. snadněji navštěvovat svatyně ve vzdálenějších oblastech. V oblasti se schůdným terénem jako např. v okolí Filippopole se vzdálenost od města, kterou bylo možné ujít v jednom dni, prodlužuje. V případě pohybu po {\em Via Diagonalis} a nápisů nalezených v okolí města Filippopolis to může být až 60 km po západo-východní ose.

Zásadní roli na rozmístění nápisů v krajině tedy hrály jednak příznivé přírodní podmínky a přístupný terén v kombinaci s blízkostí lidských sídel a rozmístění sítě komunikací, ať už v podobě řek či pozemních cest. Hustota nápisů byla největší ve městech, případně ve vybraných svatyních a s narůstající vzdáleností od města počet nápisů klesal. Tento trend částečně narušovaly komunikace, v jejichž bezprostřední blízkosti se nápisy taktéž nacházely, a to pravděpodobně díky zvýšenému pohybu obyvatelstva a usazování v menších sídlech a stanicích, které se staraly o údržbu a bezpečnost cest (Madzharov 2009, 43-57).

