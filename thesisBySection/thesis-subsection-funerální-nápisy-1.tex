
\subsection[funerální-nápisy-1]{Funerální nápisy}

Nápisy označené jako funerální je možné rozdělit do dvou kategorií dle jejich původní funkce a vztahu ke známému archeologickému kontextu: primární funerální nápisy, které byly vytvořeny pro účel pohřebního ritu, tedy např. vnitřní vybavení hrobky, architektonické součásti hrobky, a dále sekundární funerální nápisy, jejichž funkce byla původně jiná, ale pro svou sentimentální a společenskou hodnotu předměty nesoucí nápis tvořily součást pohřební výbavy (Janouchová a Weissová 2015).\footnote{Dosud nepublikovaný příspěvek na konferenci Symposium on Mediterranean Archaeology, 12. - 14. listopadu 2015, Kemer-Antalya, Turecko.}

\subsubsection[primární-funerální-nápisy]{Primární funerální nápisy}

Primární funerální nápisy jsou typicky kamenné funerální stély či jiné předměty, které sloužily k označení místa pohřbu a připomínání zemřelého. Dochovalo se jich celkem 37 a pocházejí výhradně z území řeckých měst v pobřežních oblastech.

Přes 91 \letterpercent{} primárních funerálních nápisů nese pouze řecká jména, s výjimkou jednoho nápisu z Apollónie, kde spolu figurují řecká a thrácká jména. Na černomořském pobřeží bylo možno u dvou nápisů určit použití dórského dialektu, a to u nápisů z Mesámbrie, která byla založená jako dórská kolonie, a dále i použití dialektu iónsko-attického u dvou nápisů z původně iónských kolonií Perinthu a Apollónie. Tento fakt souvisí s dialektem užívaným v rámci řeckých obcí, které oblasti osídlily a s nimiž je pojilo silné kulturně-historické pouto. Vyjádření identity na funerálních nápisech se objevuje celkem na čtyřech nápisech: vždy se jedná o vyjádření příslušnosti k řeckému městskému státu a většinou se nachází v kombinaci s řeckým osobním jménem.\footnote{Vyskytující se termíny: Aigínétés, Athénaios, Kyzikénos a Paroités.} Geografické jméno se na funerálních nápisech vyskytuje pouze jednou a jedná se o město Perinthos, tedy o řecké osídlení z území Thrákie samotné. Co se týče hledaných termínů a vyjádření identity na funerálních nápisech se vyskytuje pouze jediný administrativní termín, popisující identitu ženy nesoucí jméno řeckého původu jako propuštěnou otrokyni.\footnote{Nápis {\em IG Bulg} 1,2 334octies z~Mesámbrie.}

Z dochovaných funerálních nápisů je patrné, že místní řecké komunity byly i v 5. st. př. n. l. poměrně uzavřené a ke kontaktu mezi thráckým obyvatelstvem docházelo v minimální míře v okolí Apollónie Pontské. Naopak ke kontaktu s dalšími částmi řecky mluvícího světa docházelo na pobřeží Egejského moře, které bylo v této době místem zvýšeného zájmu řeckých obcí. Nedostatek interakcí na epigrafickém materiále však nutně nemusí znamenat neexistenci kontaktů, ale spíše poukazuje na charakter nápisů jako na velmi selektivní médium, zachycující pouze malou část tehdejší společnosti. Archeologické výzkumy naopak dokazují, že mezi řeckými a thráckými komunitami docházelo v 5. st. př. n. l. k vzájemné interakci na každodenní bázi, která se ale bohužel neprojevila na nápisech (Kostoglou 2010, 180-185; Ilieva 2007, 212-221).

\subsubsection[sekundární-funerální-nápisy]{Sekundární funerální nápisy}

Sekundární funerální nápisy se nacházejí na kovových či keramických předmětech, které primárně nebyly vyrobeny pro pohřební ritus, ale do dnešní doby se dochovaly právě jako součást pohřební výbavy. Celkem se jedná o tři nápisy na kovových nádobách a tři na keramice, které pocházejí převážně z thráckého vnitrozemí z kontextu aristokratických pohřbů.

Nápisy na předmětech z drahých kovů, případně na importované keramice kvalitního provedení\footnote{Tři nápisy na keramice byly taktéž nalezeny na thráckém území v monumentálních hrobkách: dva z nápisů pocházejí taktéž z nekropole u vesnice Duvanlij ({\em SEG} 47:1061,4 a {\em SEG} 47:1061,5) a jeden nápis pochází z nekropole města Apollónia Pontská na černomořském pobřeží ({\em SEG} 54:630). Ve všech případech se jednalo o keramické nádoby řecké provenience, jako je attická hydria, dále střep keramického talíře nesoucí mužské jméno řeckého původu, a nakonec skyfos nesoucí řecké jméno a věnování Afrodíté. Předměty mohly původně sloužit jako obchodní artikl, či dar, a text nápisu nemusí mít žádnou souvislost se sekundární depozicí v hrobce, ani s majitelem hrobky. I přesto, že tyto nápisy nemají přímou výpovědní hodnotu k průběhu funerálního ritu samotného, jedná se o důkaz cirkulace předmětů mezi thráckým vnitrozemím a černomořským pobřežím.}, a tedy i vysoké hodnoty, byly nalezeny ve funerálním kontextu v thráckém vnitrozemí: předměty pocházejí z monumentálních hrobek, které se v 5. st. př. n. l. nacházely na území kmene Odrysů, nalezených v blízkosti moderní vesnice Duvanlij (Filov {\em et al.} 1934). Monumentalita hrobek, kterou je možné spatřit ještě dnes, a nákladnost nalezené pohřební výbavy dává soudit, že se jednalo o významné jedince, pravděpodobně elitní členy kmene Odrysů (Archibald 1998, 154-171). Krátké nápisy na kovových nádobách jsou psány řeckou alfabétou a mají charakter soukromého nápisu. Protože předměty byly zhotoveny ze stříbra a ze zlata, pravděpodobně si je mohli dovolit jen nejbohatší členové tehdejší společnosti. Dochovaná jména jsou thráckého původu a bývají interpretována jako jména majitele hrobky a pohřební výbavy ({\em SEG} 46:871; Filov {\em et al.} 1934).\footnote{Konkrétně se jedná se o dva zlaté pečetní prsteny, stříbrné nádoby, součásti luxusního picího servisu.} Charakter dochovaných nápisů na kovových předmětech napovídá, že se jednalo předměty sloužící již za života majitele, které byly po jeho smrti přeneseny do hrobky jako hodnotný předmět, ukazatel společenského postavení majitele a důkaz prestiže v rámci komunity (Sahlins 1963; Whitley 1991, 354-361; Bliege Bird and Smith 2005, 221-222, 233-234).

Využití písma ve funerálním ritu se thráckém kulturním prostředí v 5. st. př. n. l. se poměrně zásadně odlišovalo od funkcí, které písmo zastávalo v řeckých komunitách na thráckém pobřeží. Docházelo-li ke kontaktu za účelem obchodní výměny zboží, nedocházelo ještě v této době k prolínání kulturních zvyklostí a společenského uspořádání, jak by se dalo očekávat. Pokud k nim přesto docházelo, výrazně se ale neprojevily na výsledné podobě a využití funerálních nápisů.

