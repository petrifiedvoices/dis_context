
\subsection[funerální-nápisy-10]{Funerální nápisy}

Dochovaných 35 funerálních nápisů pochází výhradně z kontextu řeckých komunit na pobřeží, nicméně onomastické záznamy nasvědčují na proměňující se zvyklosti a pravděpodobně i složení tamější populace. Tři čtvrtiny nápisů pocházejí z Byzantia, kde lze pozorovat narůstající politickou moc Říma, která se projevila i na charakteru nápisů.

O narůstající roli římského kulturního vlivu svědčí i nápis {\em I Aeg Thrace} 72 je psán výhradně latinsky. Navíc se v této době u funerálních nápisů začíná zvyk uvádět roky, jichž se zemřelý dožil, většinou zaokrouhlené na pět let, což je zvyk typický pro římské nápisy (MacMullen 1982, 238). Mezi jmény vyskytujících se na nápisech nicméně i nadále převládají řecká jména, kterých je přibližně čtyřikrát více než jmen římských a pětkrát více než jmen thráckých. Jako měst svého původu označuje Byzantion na nápise {\em IK Byzantion} 352 muž nesoucí čistě římská jména Gaios Ioulios, což svědčí o jeho dvojí loajalitě směrem k římské tradici, ale i politické příslušnosti k Byzantiu, tedy původně řecké kolonii. Z toho lze soudit, že jména v této době přestávají být jednoznačným ukazatelem původu, ale spíše se jedná o uvědomělou volbu identity, vědomým prohlášením přináležitosti k určité komunitě.

