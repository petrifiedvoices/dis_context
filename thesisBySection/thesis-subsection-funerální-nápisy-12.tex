
\subsection[funerální-nápisy-12]{Funerální nápisy}

Celkem 59 nápisů je interpretováno jako funerální\footnote{57 nápisů je soukromého charakteru a dva jsou veřejného charakteru.}, což představuje zhruba dvě třetiny celého souboru, přibližně stejně jako v 1. st. př. n. l. Většina nápisů je zhotovena na stéle, ale spadá sem i skupina 16 sarkofágů, které nesly nápis, a zároveň sloužily jako úložiště tělesných pozůstatků zemřelého. Hlavní produkční centra funerálních nápisů jsou Perinthos s 42 nápisy, Byzantion a Maróneia s pěti nápisy.

Sarkofágy z Perinthu užívají téměř totožné formule a termíny, a na většině z nich se vyskytuje formule o ochraně sarkofágu a pozůstatku zemřelého politickou autoritou města, jako ve skupině nápisů z 1. st. př. n. l. Osobní jména na sarkofázích nesou asi v polovině případů římská jména, a ve zbývající polovině jména řecká. Thrácká jména se na sarkofázích vůbec nevyskytují, což naznačuje římský původ této zvyklosti.

Převaha nápisů stále pochází z řeckých komunit, které si i nadále udržují relativně konzervativní charakter. Nejčastěji se na nápisech objevují řecká jména, zhruba v polovině příkladů. Dále ve 40 \letterpercent{} jména římská a zbylých 10 \letterpercent{} připadá jménům thráckým a jménům nejistého původu. Thrácká jména se objevují pouze na dvou nápisech, a to jak samotná, tak v kombinaci s řeckým jménem, výhradně v souvislosti s funerálním nápisem zemřelého zastávající funkci stratéga.\footnote{Nápis Manov 2008 199 a nápis {\em I Aeg Thrace} 87.} Geografický původ uvádí pouze jeden nápis, odkazující na bíthýnskou Nikomédii.\footnote{{\em Perinthos-Herakleia} 144.}

Obsah nápisů si udržoval tradiční formu, nicméně se vyskytují termíny popisují nově zavedené součásti funerálního ritu, jako sarkofág, ale objevují se i termíny pro nově vzniklá povolání a funkce.\footnote{Invokační formule oslovující okolo jdoucího čtenáře {\em (chaire}) se objevily celkem 26krát, ({\em parodeita}) 21krát. Funerální nápisy sloužily ve většině případů pro rodinné pohřby: celkem 24krát se objevuje text zhotovený jedním z partnerů pro sebe a pro manžela či manželku, či jiného člena nejbližší rodiny. Pro popis hrobu samotného sloužily termíny {\em mnémeion} jednou, {\em latomeion} v devíti případech a {\em soros} v 11 případech, popisující sarkofág, dále {\em stéla} se sedmi výskyty jako označení hrobu a {\em bómos} se dvěma výskyty a po jednom výskytu {\em tymbos} a séma označují hrob samotný.} Text nápisů nesl více detailů se života zemřelého: setkáváme se s šesti vojáky různých hodností, jako například legionář, jezdec, {\em stratégos} či {\em pragmatikos}. Dále se setkáváme se stavitelem domů ({\em domotektón}), ale i notářem ({\em notários}). V 17 případech se dozvídáme věk, kterého se zemřelý dožil, typicky zaokrouhlený na pět let. Jedná se o typický římský kulturní zvyk, který se na území Thrákie vyskytl již v 1. st. n. l. ve velmi omezeném počtu, nyní však u celé třetiny nápisů.

