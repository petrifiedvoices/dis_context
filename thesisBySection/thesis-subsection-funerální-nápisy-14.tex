
\subsection[funerální-nápisy-14]{Funerální nápisy}

Funerálních nápisů se dochovalo celkem 54, z čehož 52 má povahu soukromého nápisu a dva byly zhotoveny na náklady města.\footnote{{\em SEG} 48:894 a {\em I Aeg Thrace} 217.} Nápisy pocházejí z celého území Thrákie, tedy jak z pobřeží, tak z vnitrozemí. Hlavními produkčními centry je údolí středního toku Strýmónu, odkud pochází 14 nápisů, dále Perinthos s devíti nápisy a Maróneia se šesti nápisy.

Text nápisů pozvolna opouští od tradičně používaných formulí a stejně jako u skupiny nápisů z 2. st. n. l. se objevují nové termíny, které popisují nově vzniklé skutečnosti a předměty, jako termíny pro sarkofág a formule zajišťující právní ochranu jejich obsahu.\footnote{Invokační formule {\em chaire} se objevuje pouze čtyřikrát, oslovení okolojdoucího ({\em parodeita}) pouze čtyřikrát. termín označující hrob ({\em tymbos}) se objevuje jednou, termín {\em stélé} čtyřikrát, {\em mnémeion} jednou, termín pro sarkofág celkem pětkrát ({\em soros, latomeion, diathéké}). Třikrát objevuje invokační formule vzývající podsvětní bohy v latině ({\em Dis Manibus}) a ani jednou v řečtině ({\em Theoi Katachthonioi}). Celkem čtyři nápisy uvádí věk zemřelého, což je typický zvyk pro římské nápisy. Podobně jako u nápisů datovaných do 2. st. n. l. pochází skupina devíti sarkofágů zejména z lokalit v okolí Propontidy (Perinthos, Byzantion), z nichž čtyři nesou ochrannou formuli, která zakazuje nové použití sarkofágu pod peněžní pokutou. Osobní jména na těchto sarkofázích jsou převážně řeckého původu.} Obsah textů poukazuje na nadále se proměňující složení společnosti, či alespoň narůstající manifestaci nejrůznějších životních drah a profesí na nápisech. Stejně tak se i nadále objevují zvyklosti typické pro nápisy z římské doby, stejně jako u nápisů z 1. a zejména z 2. st. n. l.\footnote{Texty jsou zhotovovány členy nejbližší rodiny a hrobky slouží k pohřbům více členů rodiny. Celkem 26 nápisů zmiňuje společný hrob s partnerem zemřelého či zmiňuje členy rodiny. Vojáci se na nápisech objevují celkem třikrát, konkrétně jde o legionáře, jezdce a {\em carceraria}, hlídače vojenského vězení a jednou jako veterán. Jiná, než vojenská povolání zmiňují funkce vždy po jednom výskytu kněžího ({\em archiereus}, {\em hiereus}), člena městské rady ({\em búleutés}), námořního obchodníka ({\em naukléros}), či patrona. Geografický původ zemřelého nápisy neudávají.}

Vyskytující se osobní jména i jsou nadále převážně řecká, nicméně je možné pozorovat nárůst jak jmen římských, tak především i jmen thráckých. Řecká jména představují 38 \letterpercent{}, což představuje zhruba 12 \letterpercent{} pokles oproti nápisům z 1. až 2. st. n. l. Římská jména představují 30 \letterpercent{}, což značí 10 \letterpercent{} pokles. Thrácká jména naopak zaznamenala nárůst ze zhruba 3 \letterpercent{} až na 20 \letterpercent{}. Největší koncentrace devíti nápisů nesoucích thrácká jména pochází v údolí středního toku Strýmónu a dále z okolí Augusty Traiany a Filippopole po dvou nápisech. Osoby nesoucí thrácká jména většinou odkazují na svůj původ i pomocí thráckého jména rodiče či dalších členů rodiny, případně i prohlášením o původu.\footnote{Příkladem je nápis {\em IG Bulg} 3,2 1794, který patří Apollónidovi, synovi Aulozénida ze Sapaiké, který je však pochován v thráckém vnitrozemí v lokalitě známé jako Dodoparon nedaleko středního toku řeky Tonzos (Janouchová 2017).} K mísení thráckých a římských onomastických tradic dochází na funerálních nápisech zcela minimálně a důležitou roli začíná v tomto období hrát spíše thrácká identita a příslušnost k místní komunitě.

