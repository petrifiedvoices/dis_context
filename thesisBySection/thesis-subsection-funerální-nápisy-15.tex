
\subsection[funerální-nápisy-15]{Funerální nápisy}

Funerálních nápisů ze 3. st. n. l. se dochovalo celkem 76, což představuje zhruba nárůst o 40 \letterpercent{} oproti předcházejícímu století. Podobně jako v předcházejícím období si funerální nápisy uchovávají standardní formule, typické pro tento druh nápisů, nicméně jejich celkový počet se snižuje, pravděpodobně jako reakce na proměňující se funerální ritus.\footnote{Např. typická formule na památku ({\em mnémé charin} či {\em mnéiás charin)} ve dochovala ve 21 případech. Místo pohřbu je nejčastěji nazýváno {\em mnémeion} ve třech, {\em larnax} v jednom případě pro sarkofág, {\em latomeion} ve třech případech taktéž pro sarkofág, {\em soros} v sedmi případech pro urnu, {\em chamosorion} pro plochou hrobku umístěnou na zemi. Celkem tři nápisy zmiňují vztyčení stély a ve dvou případech i zhotovení textu nápisu. Celkem čtyři nápisy oslovují okolo jdoucího poutníka ({\em chaire/chairete parodeita}), což je znatelně méně než v předcházejícím období.}

Typický text nápisu nese jméno nebožtíka, jeho zařazení v rámci komunity, jeho původ a jeho dosažené postavení či vykonané skutky. Dále je zde uveden zhotovitel nápisu a jeho vztah k zemřelému. Většinou se jedná o člena rodiny, přítele či kolegu z armády. Důvody pro uvádění pozůstalých jsou pravděpodobně spojeny s dědickými nároky a povinnostmi (MacMullen 1982; Meyer 1990). Geografická jména poukazují na původ nebožtíků či jejich blízkých, jednak z Filippopole a oblasti Haimu, ale i z měst Malé Asie jako je Níkaia či Smyrna, avšak výskyt na pouhých čtyřech nápisech svědčí o tom, že geografický původ nebyl nejdůležitějším faktorem identifikace jednotlivce.\footnote{V jednom případě se setkáváme s označením barbar, které popisuje skupinu lupičů neznámého původu, jimž se podařilo uniknout knězi Aureliovi Flaviovi Markovi na nápise {\em IG Bulg} 1,2 1 z lokality Tvardica v regionu Caron Limen. Termín {\em barbaros} tak nelze spojovat výlučně s thráckým etnikem, a pokud ano, jedná se o etický emotivně zabarvený popis nespecifické etnické skupiny. Etnická příslušnost tak ve 3. st. n. l. nehraje téměř žádnou roli, a pokud ano, pak pouze v samosprávě a rozdělení obyvatel v rámci provincie.} Povolání zemřelého se dochovala na pětině nápisů, což poukazuje na narůstající váhu prezentace povolání a dosaženého společenského postavení.\footnote{Setkáváme se s šesti vojáky různých hodností, dále s čtyřmi gladiátory nejrůznějších specializací, jedním strážným, jedním knězem, jedním propuštěncem, jedním členem {\em búlé} a jedním členem {\em gerúsie}.}

Dle výskytu osobních jmen v průběhu ve 3. st. n. l. dochází k většímu zapojení osob nesoucích thrácká jména do praxe vztyčování funerálních nápisů a jejich podíl je nyní již takřka pětinový.\footnote{Celkem se na nápisech dochovalo 182 jmen, z nichž 43 \letterpercent{} je řeckých, 32 \letterpercent{} římských, 18,5 \letterpercent{} thráckých a 6,5 \letterpercent{} je nejistého původu. Řecká jména, ať už samotná, či v kombinaci, se vyskytovala zejména v okolí řeckých měst na pobřeží Egejského moře, Perinthu a Byzantia. Ve vnitrozemské Thrákii se řecká jména vyskytovala zejména v údolí střední toku Strýmónu. Až polovina osob nesoucí řecké jméno nesla i jméno římské, což značí přijetí římského onomastického systému. Římská jména bez kombinace s thráckým, řeckým či jiným jménem se dochovala pouze na deseti nápisech rovnoměrně rozmístěných na území Thrákie.} Thrácká jména pocházela především z nápisů nalezených v údolí středního toku Strýmónu a částečně v okolí Maróneie a Topeiru. Polovina thráckých jmen se vyskytovala v kombinaci se jménem římským, což taktéž naznačovalo proměnu onomastické tradice směrem ke způsobům praktikovaným v rámci římské říše. Přijetí římského jména a jeho uvedení na nápisech mohlo signalizovat jak společenské postavení dané osoby, tak i jeho nejbližší rodiny, vzhledem k tomu, že až do roku 212 n. l. se udílelo především za individuální zásluhy. V roce 212 n. l. však bylo uděleno římské občanství a právo nosit jméno římského původu všem obyvatelům římské říše, a právo nosit římské jméno tak postupně ztrácelo váhu a společenskou prestiž, kterou mělo před rokem 212 n. l. (Blanco-Perez 2016, 271).

V této době je možné pozorovat větší nárůst geografických jmen, často označujících geografický původ osoby, odkazující zejména na místa v Malé Asii, z toho čtyři v sousední Bíthýnii. Zvýšená přítomnost osob z Malé Asie nasvědčuje migračním proudům z této oblasti, které začaly již koncem 2. st. n. l., ale na nápisech se projevily zejména až ve 3. st. n. l. (Delchev 2013, 18-19; Sharankov 2011, 141, 143). Na nápisech je možné pozorovat i zvýšenou přítomnost osob ze západních římských provincií, která ale nedosahuje úrovně přítomnosti osob z maloasijských regionů.

Převaha textů byla i nadále v řecké jazyce a latinské nápisy se objevovaly především v souvislosti s vojáky či veterány a pocházely z míst s permanentně umístěnou vojenskou posádkou.\footnote{Latinský text se objevil u funerálních nápisů pouze pětkrát, a to u zemřelých vojáků sloužících v římské armádě ve čtyřech případech a u propuštěnce v jednom případě. Čistě latinský text nápisu se objevil třikrát v Byzantiu, a to zejména u vojáků nesoucích čistě jména římského či nethráckého původu. Jeden bilingvní překlad identického řeckého textu se objevil ve Filippopoli, a to u vojáka nesoucí římské a thrácké jméno, který byl thráckého původu, ale kariéru si vybudoval v rámci římské armády. Latinský text byl uvedený na prvním místě, řecký text až jako druhý. Text věnovaný propuštěncem svému pánovi, římskému centurionovi thráckého původu, obsahuje invokační formuli v řečtině a text nápisu je v latině a pochází z Perinthu.} Latina sloužila jako oficiální jazyk římské armády a do jisté míry ovlivnila i podobu funerálních nápisů zemřelých vojáků. Volba jazyka byla dána spíše funkcí nápisu a publikem, jemuž byla určena.\footnote{Pokud byl text určen pouze vojákům nethráckého původu, případně pokud se mělo poukazovat na postavení zemřelého v rámci armády, byla volena spíše latina jako je tomu u nápisů z Byzantia. Pokud však měl být nápis určen jak pro vojenskou, tak nevojenskou komunitu, tak docházelo k prolínání latiny a řečtiny či ke kompletním překladům textů, jako je tomu i nápisů z Perinthu a Filippopole.}

Ve 3. st. n. l. dále narůstá míra standardizace funerálních nápisů a souvisejících procedur. Ve 18 případech se setkáváme s frází na konci nápisu postihujícím případné nové použití hrobky, ale zejména sarkofágu a urny, stejně jako v 1. a 2. st. n. l.\footnote{Nejvíce takovýchto nápisů bylo nalezeno v Byzantiu, celkem šest, dále pak v Maróneii čtyři nápisy, v Perinthu tři nápisy, dva v Topeiru a jeden v Abdéře, Sélymbrii a Filippopoli. Převážně se jednalo o osoby nesoucí původní řecká jména a nově přijaté jméno Aurelios či Aurelia. V jednom případě se jedná o člena {\em gerúsie} z Filippopole, člena {\em búlé} z Perinthu a dceru veterána z Maróneie. Sarkofágy většinou sloužily pro dva a více členů rodiny a byly věnovány partnery či potomky zemřelých.} Z poměrně vysoké frekvence a rozšíření tohoto zvyku můžeme soudit, že se jednalo o poměrně častou praxi, kdy byla místa pohřbu využívána sekundárně a bylo nutné se chránit. Politická autorita města, pod jehož ochranu nápis spadal, tak určitou ochranu pravděpodobně zajišťovala, jinak by nedošlo k tak hojnému rozšíření tohoto zvyku, a tím pádem i rozšíření epigrafické formule do několika produkčních center.

