
\subsection[funerální-nápisy-17]{Funerální nápisy}

Zvyk zhotovovat funerální nápisy se tak ve 4. st. n. l. uchoval v Thrákii pouze v rámci křesťanské komunity, převážně z okolo Hérakleie. Funerálních nápisů se celkem dochovalo 10 a přestavují tak nejčetnější skupinu nápisů ze 4. st. n. l. Devět nápisů nese jasné znaky křesťanské víry nebožtíků či pozůstalých, ať už je to slovní vyjádření, zobrazení christogramu či vyobrazení křížů.\footnote{Např. na nápise {\em Perinthos-Herakleia} 219 se setkáváme s frází „Χρειστιανοὶ δὲ πάντες ἔνεσμεν”, tedy vědomé přihlášení se všech zmíněných osob ke křesťanské víře.} Fakt, že většina nápisů nese textové či vizuální konotace na křesťanskou víru, není nijak překvapivý, vzhledem k tomu, že hlavních produkční centrum Hérakleia, odkud pochází šest nápisů, byla zároveň jedním ze hlavních sídel křesťanské komunity pro region jihovýchodní Thrákie a křesťanství se v průběhu 4. st. n. l. stalo uznávaným náboženstvím a dokonce oficiální vírou římské říše (Dumanov 2015, 92-96). Osoby na nápisech i nadále nesou jména vzniklá kombinací římského a řeckého jména, a i nadále se v mírně pozměněné formě udržují kulturní zvyklosti předcházejících období, jako je např. fráze o ochraně hrobu či texty nápisů promlouvající ke kolemjdoucím.\footnote{Na nápise {\em Perinthos-Herakleia} 177 vystupuje Aurelios Afrodeiseios spolu s manželkou Aurelií Deionoisií. V nápisu je také uvedeno, že kdokoliv se odváží či pokusí neoprávněně použít hrobku, musí zaplatit pokutu právoplatným dědicům. Celkem se tato fráze objevuje na čtyřech sarkofázích z Hérakleie. Na závěr nápisu je taktéž ve čtyřech případech uvedena tradiční formule {\em chaire parodeita}, avšak s mírně pozměněnou ortografií, která pravděpodobně reflektovala tehdejší výslovnost.}

