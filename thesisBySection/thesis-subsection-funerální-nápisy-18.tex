
\subsection[funerální-nápisy-18]{Funerální nápisy}

Funerálních nápisů se celkem dochovalo 28 a i nadále pocházejí převážně z křesťanské komunity, jak je patrné z obsahu i dekorace náhrobků samotných.\footnote{Místo spočinutí označují slova jako {\em mnémeion} či {\em mnéma} označující jak hrob, tak stélu samotnou. Dále se začíná objevovat termín {\em thesis}, typický právě pro křesťanské nápisy, označující místo posledního spočinutí. Termín {\em latomeion} se vyskytuje celkem pětkrát a označuje sarkofág, který nese nápis a zároveň slouží jako místo posledního odpočinku. Sarkofágy pocházejí převážně z křesťanské komunity z Hérakleie a sloužily pro rodinné pohřby, v jednom případě až pro šest lidí. Texty na sarkofágu typicky ztotožňovaly nebožtíka jako řádného obyvatele Hérakleie. V šesti případech nápisy obsahovaly formuli, které zakazovala jejich další používání pod pokutou, vymahatelnou v rámci samosprávy či církve právě v Hérakleii. Podobné nápisy existovaly v několika městech již od 1. st. n. l. Podle množství dochovaných nápisů s identickým, či velmi podobným textem se dá usuzovat, že se jednalo o poměrně častý problém, který ve 4. st. n. l. přetrval zejména v Hérakleii, např. na nápise {\em Perinthos-Herakleia} 180.} I nadále si uchovaly poměrně informativní a interaktivní charakter: v osmi případech nápisy promlouvají k náhodně procházejícímu poutníkovi ({\em chaire parodeita}) a sdělují mu životní osudy zemřelého. Zemřelý je identifikován pomocí osobního jména a jména rodiče, případně jeho přináležitost ke křesťanské komunitě. Poměrně dlouhý rozsah nápisů poskytuje řadu informací nejen o nebožtíkovi, ale i o jeho rodině.\footnote{Ve dvou případech se jednalo o vojáky z povolání, konkrétně o legionáře a {\em centenaria}, v jednom případě o lékaře, v jednom případě o námezdního dělníka, dále o architekta, stříbrotepce, mincovního mistra a výběrčího daní. Dozvídáme se o manželkách zesnulých, jejich potomcích a někdy i o věku, jehož se dožili. Výjimku tvoří nápis {\em SEG} 49:871 nalezený na nástěnné malbě uvnitř hrobky číslo 252 v regionu města Augusta Traiana s typickou formulí přející štěstí ({\em agathé týché}).}

Identita osob byla jasně dána křesťanskou vírou, a nebylo tedy nutné uvádět geografický původ. K prolínání onomastických tradic téměř již nedocházelo, převaha osobních jmen byla řeckého původu, s malým podílem jen římského původu. Thrácký prvek se až na jednu výjimku z dochovaného epigrafického materiálu zcela vytratil.\footnote{Nápis {\em Perinthos-Herakleia} 183 patřil stříbrotepci s původně thráckým jménem, Mókiánem z Hérakleie, který přijal křesťanskou víru, a text nápisu obsahoval stejnou právní formulku poskytující ochranu před novým použitím sarkofágu, která se vyskytovala v oblasti již od 1. st. n. l.} Tímto zásadním způsobem proměnilo křesťanství podobu a obsah epigrafické produkce v Thrákii, ale i mimo ni. Přestala být oceňována identita politická či vojenská kariéra a společenská prestiž v rámci státního aparátu, ale namísto nich na důležité místo ve společnosti nastoupila sounáležitost s křesťanskou obcí.

