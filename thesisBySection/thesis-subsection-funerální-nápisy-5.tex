
\subsection[funerální-nápisy-5]{Funerální nápisy}

Funerální nápisy tvoří nejpočetnější skupinu soukromých nápisů, podobně jako v předcházejících stoletích. Celkem se dochovalo 71 primárních funerálních nápisů datovaných do 3. st. př. n. l. Do skupiny sekundárních funerálních nápisů patří jeden nápis vyrytý do fresky uvnitř mohylové hrobky.

\subsubsection[primární-funerální-nápisy-2]{Primární funerální nápisy}

Do této skupiny již tradičně patří funerální stély či jiné předměty, jejichž hlavní funkcí bylo označovat místo pohřbu a připomínat zemřelého. Celkem se jich dochovalo 71 a pocházely především z řeckých obcí na mořském pobřeží. Největší koncentrace nápisů pochází z černomořské Mesámbrie s 23 nápisy, z Byzantia a z egejské Maróneie se 17 nápisy. Mimo pobřežní oblasti byl nalezen jeden náhrobních nápis z okolí moderní vesnice Červinite Skali v oblasti středního toku Strýmónu.

Skupina nápisů pocházejících z pobřežních oblastí naznačuje, že nápisy z řeckých měst a jejich bezprostředního okolí nadále udržují funerální ritus v podobě, v jaké se projevoval v předcházejících staletích.\footnote{Texty funerálních nápisů byly převážně jednoduché, v rozsahu dvou až tří řádků. Nejdelší text měl 11 řádků, ale to se jednalo spíše o výjimku. Typický nápis obsahoval jméno zemřelého a jméno jeho rodiče či partnera. Pouze výjimečně nápis obsahoval informace z nebožtíkova života či ve čtyřech případech vyjádření zármutku pozůstalých zhotovené metricky. Texty tedy spíše následovaly tradici jednoduchých textů poukazujících pouze na osobu zemřelého, jak bylo obvyklé v předcházejících stoletích.} Analýza osobních jmen potvrdila, že 84 \letterpercent{} osobních jmen je řeckého původu, 13 \letterpercent{} není možné určit a pouze 3 \letterpercent{} jmen jsou pravděpodobně thráckého původu. Thrácká jména se vyskytovala celkem na třech nápisech, z nichž dva pravděpodobně pocházely z Odéssu a jeden z Byzantia. V jednom případě se jednalo o ženu, v jednom případě o manželský pár-sourozence a v jednom případě o muže. Z toho plyne, že funerální stély tedy byly využívány výhradně obyvateli se jmény řeckého původu. Další vyjádření identity jako kolektivní pojmenování odkazují na řecký původ zemřelých či jejich rodinných příslušníků s kořeny v okolí thráckého regionu.\footnote{Patří sem například termíny jako Filippeus, Lýsimacheus, Hérakleótés, a pak dále Krés a Rhodios. Geografické pojmy se vyskytují jen v jednom případě a poukazují na původ jistého Apollónia z Babylónu. Celkem šest lidí udávalo svůj geografický původ, který byl z více než poloviny mimo oblast Thrákie, nicméně jejich jména byla řeckého původu.} Ač nečetné, jedná se o zdokumentované případy migrace části obyvatel. Nelze však tvrdit, že by ve 3. st. př. n. l. migrace obyvatelstva byla častější než ve 4. st. př. n. l. pouze na základě většího výskytu geografických termínů. Kladení důrazu na geografický původ může souviset s narůstající propojeností hellénistického světa, která je jedním z průvodních znaků hellénismu a se snahou udržet si svou identitu i v rámci nové komunity, a proto se tato vyjádření začala objevovat častěji.

Nejen osobní jména a vyjádření identity, ale i hledané termíny poukazují na udržování tradic řeckého kulturního prostředí i v rámci funerálního ritu.\footnote{V osmi případech vyskytují se zde typické invokační formule ({\em chaire}). Pro popis hrobky se ve dvou případech používá taktéž zcela typické vyjádření ({\em tymbos}), z čehož v jednom případě na nápise {\em IK Byzantion} 305 byl termín upřesněn jako navršený hrob čili mohyla ({\em chóstos tymbos}). Dále se zde vyskytuje jednou termín pro hrob samotný ({\em tafos}).} Všechny dochované termíny jsou použity v jejich původním významu a v kontextu v jakém bychom je mohli najít i ve zbytku řecky mluvícího světa té doby. Veškeré důkazy tedy poukazují na přetrvávání tradičních funerálních zvyklostí a jejich projevů na nápisech bylo ve 3. st. př. n. l. omezené pouze na řecké komunity.

\subsubsection[sekundární-funerální-nápisy-2]{Sekundární funerální nápisy}

Nápis pochází z thrácké aristokratické hrobky z Kazanlackého údolí na řece Tonzos, na území tradičně ovládaném kmenem Odrysů. Dle bohaté pohřební výbavy se usuzuje, že se mohlo jednat o členy thrácké aristokracie, podobně jako v podobných případech v předcházejících stoletích (Sharankov 2005, 29-35). {\em Dipinti} {\em SEG} 58:703 bylo nalezeno na fresce v hrobové komoře a má dvě části. První části nese jméno Seutha, syna Rhoigova a souvisí s ním kresba mladého muže, která je umístěna v bezprostřední blízkosti. Druhá část je tzv. autorský podpis malíře jménem Kozimazés, který je znám jako zhotovitel další fresky {\em SEG} 58:674 v hrobce v Alexandrovu, která je datovaná do 4. až 3. st. př. n. l. a o níž jsem hovořila dříve. Charakterem tento nápis odpovídá podobným skupinám funerálních nápisů z 5. a 4. st. př. n. l., kdy thrácká aristokracie využívala písmo zejména utilitárně pro svou soukromou potřebu vně komunity, tedy odlišným způsobem, než můžeme vidět v řeckých komunitách.

