
\subsection[funerální-nápisy-6]{Funerální nápisy}

Celkem se dochovalo 30 funerálních nápisů datovaných do 3. až 2. st. př. n. l., které pocházejí převážně z řecké komunity na pobřeží a z řecké či makedonské komunity v okolí Hérakleie Sintské, jak dokazuje přítomnost zejména řeckých jmen zemřelých osob.\footnote{Jediné jméno, které je možné průkazně spojovat s thráckým původem, je jméno Amatokos z {\em IK Byzantion} 325, které je použito jako jméno rodiče Hermia.} Charakteristika nápisů je totožná s funerálními nápisy pocházející z řeckých komunit 5. až 3. st. př. n. l. Za zvláštní pozornost nicméně stojí skupina tří funerálních nápisů z Hérakleie Sintské, na nichž se dochovala jména celkem pěti osob, z čehož byly čtyři ženy.\footnote{Jednotlivé osoby byly identifikovány nejen pomocí osobního jména, ale i pomocí údajů o rodičích a partnerech a veškerá dochovaná jména jsou jména řeckého či makedonského původu. Jiná vyjádření identity, či podoby jazyka, která by pomohla komunitu lépe zařadit, se nedochovala.} Nosiče nápisů byly ve dvou případech vyrobeny z místně dostupného materiálu jako je tuf, vápenec a varovik, ale uchovávaly si tradiční vzhled jednoduchých funerálních stél s akroteriem, figurální funerální scénou a v jednom případě písmeny malovanými červenou barvou. Zdroj materiálu byl sice místní, nicméně provedení odkazuje na techniky a motivy tradičně používané v rámci řecké či makedonské komunity. Z archeologických zdrojů však víme, že v době hellénismu byla v Hérakleii Sintské založena pravděpodobně makedonská vojensko-obchodní stanice, která se později rozrostla na město (Nankov 2015, 7-10, 22-27). Není zcela jasné, zda se jednalo o osídlení čistě makedonské, či bylo obývané jak Makedonci, tak Thráky, jak bývalo obvyklé u měst zakládaných Filippem II. (Adams 2007, 9-11). V současné době zde neustále probíhají archeologické výzkumy, a tak je možné, že se do budoucna objeví ještě více důkazů. Zatím je však zřejmé, že mimo pobřežní oblasti byl zvyk stavění náhrobních kamenů ve 3. až 2. st. př. n. l. rozšířen pouze v oblasti obývané řeckými či makedonskými osadníky, a nevyskytoval se v čistě thrácké komunitě.

Podobně jako v 5. a 4. st. př. n. l. je písmo v kontextu thrácké aristokracie využíváno pro velmi specifický účel a v okruhu velmi omezeného počtu lidí. Hlavním účelem je ztotožnit majitele, který patřil do okruhu thrácké aristokracie, či zhotovitele, který mohl být jak thráckého, případně řeckého původu. V případě nápisu na kovovém předmětu {\em SEG} 59:759 se jedná o jméno tvůrce na zlatém diadému, který se nalezl uvnitř hrobky patřící pravděpodobně ženě. Jméno zhotovitele předmětu je řeckého původu a používá typicky řeckou formuli {\em epoi{[}é{]}sen}, tedy zhotovil Démétrios (Manov 2009, 27-30).\footnote{Na témže diadému se nachází ještě pravděpodobně thrácké jméno Kortozous v genitivu singuláru, u nějž není jisté, zda patřilo muži či ženě. Manov usuzuje, že je to jméno majitele diadému, a pokud jím byla žena pohřbená v hrobce, kde byl předmět nalezen, pak diadém mohl patřit právě jí. Nejedná se tedy o primárně funerální nápis, ale o předmět osobní potřeby, který byl uložen do hrobu po smrti majitele.} Pořizování předmětů s nápisy nicméně stále nepatřilo k běžnému standardu ani mezi thráckými aristokraty, natož mezi běžnou thráckou populací a přístup k písmu byl odlišný v rámci řecké a thrácké komunity, což dokazuje i nadále pokračující absence pohřební stél z thráckého kontextu.

