
\subsection[funerální-nápisy-8]{Funerální nápisy}

Celkem 98 funerálních nápisů pochází z 2. až 1. st. př. n. l. Nejvíce nápisů pochází z Byzantia, celkem 78, což představuje téměř 80 \letterpercent{} všech funerálních nápisů z daného období. Z vnitrozemí pocházejí pouze tři nápisy, a to z údolí středního toku Strýmónu z regionu Hérakleie Sintské, podobně jako v předcházejícím období. Celkově dochází k poklesu počtů funerálních nápisů napříč celou Thrákií, s výjimkou řeckého Byzantia, kde dochází k poměrně markantnímu nárůstu. Podobně jako ve 2. st. př. n. l. zcela chybí sekundární funerální nápisy, které by bylo možné spojovat s thráckou aristokracií, což může svědčit o oslabení politické a ekonomické moci thráckých aristokratů či o proměnách přístupu Thráků k užití písma a k funerálním ritu obecně.

Skupina tří nápisů z thráckého vnitrozemí z regionu Hérakleie Sintské zaznamenává jména řeckého původu. Jedná se o funerální stély tří žen, jejichž otcové nesou taktéž jména řeckého (či makedonského) původu. Jak je již patrné v předcházejícím období, tato komunita si uchovává tradiční hodnoty, alespoň co se týče epigrafického projevu.

