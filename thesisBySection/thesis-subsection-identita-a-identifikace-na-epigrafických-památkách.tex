
\subsection[identita-a-identifikace-na-epigrafických-památkách]{Identita a identifikace na epigrafických památkách}

Způsob, jakým se jedinec rozhodne na nápisech vystupovat či jakým způsobem je reprezentován, vypovídá mnoho o vztahu, jaký zaujímá k svému nejbližšímu okolí a jak vnímá sám sebe v kontextu hierarchii dané společnosti. Díky tomu, že jsou nápisy „přímým pramenem”, lze se domnívat, že identifikace na epigrafických památkách patří mezi nejautentičtější vyjádření identity jednotlivců a skupin, jaké se nám z~antiky dochovala.

V pojetí identity a sebe-identifikace vycházím z díla sociologa Richarda Jenkinse v nichž je identita chápána jako základní prostředek lidí zasadit vlastní existenci do širšího rámce společnosti a je výsledkem přirozené lidské potřeby orientovat se v mezilidských vztazích (2008, 16-18). V Jenkinsově pojetí identita a reprezentace jednotlivce patří mezi nejzákladnější pojítko mezi člověkem a okolním světem. Kontakty s dalšími členy komunity lidé formují svou identitu. Identifikaci je pak možno chápat jako konkrétní projev identity, což je neustále probíhající obousměrný proces mezi jedincem a komunitou, který v sobě zahrnuje vědomou či nevědomou reflexi vzájemné pozice (Jenkins 2008, 13, 36-). Jinými slovy vědomá identifikace respektuje jedinečnost každého člověka, a zároveň ho napomáhá utvářet celospolečenské vztahy. Identita každého člověka je proměnlivá a vyvíjí se v závislosti na životní situaci. Tato interakce mezi jednotlivcem a komunitou je nikdy nekončící a stále se měnící proces, spolu s~tím, jak je jedinec konfrontován se změnami v~životním stylu a proměnami společnosti okolo něj (Jenkins 2008, 17, 36-48). Tím, že jedinec v rámci komunikace s dalšími lidmi zdůrazní určitou součást identity, dělá to proto, že je to pro něj v danou chvíli určitým způsobem výhodné, či se to ztotožňuje s jeho aktuálním světonázorem. Tím, že se jedinec situuje do určité komunity podobně se identifikujících lidí, si může zajistit bezpečí, ekonomickou soběstačnost, prestiž, nebo jen legitimizuje svou pozici ve společnosti.

Identita každého člověka se skládá z několika částí, z nichž každá složka může hrát důležitou roli v jiné životní situaci a více tak vystupuje na povrch. Každý člověk má potřebu se zároveň vůči společnosti vymezit, definovat svou vlastní identitu, a tím se zároveň zařadit do existujících komunitních struktur.\footnote{Ve své přelomové práci Fredrik Barth (1969) popisuje dynamiku fungování etnických skupin, a to jak vnitřní uspořádání, tak především interkomunitní společenské vztahy na principech stejnosti a odlišnosti. Barth tvrdí, že tyto principy fungují nejen mezi jednotlivými komunitami, ale utvářejí taktéž vnitřní dynamiku skupiny a v konečném důsledku napomáhají i identifikačním procesům jedince. Barthův model je možné použít nejen na popis fungování etnických skupin, ale jakéhokoliv kolektivu a formování identity obecně (Jenkins 2008, 119).} Jedinec se může vůči svému okolí vymezovat několika způsoby, jako například volbou použitého jazyka, dále jedinečným poznávacím znamením, jako je osobní jméno, nebo vymezením svého původu a vztahu k okolním komunitám za použití kolektivní identifikace.

