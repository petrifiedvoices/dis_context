
\subsection[kontextualizace-mezikulturních-kontaktů]{Kontextualizace mezikulturních kontaktů}

Směrem, který má velký ohlas v současných pracích zabývajícími se mezikulturními kontakty, je přístup navrhovaný Michaelem Dietlerem (1997; 1998; 2005). Zatímco Dietler odmítá monolitické modely jako hellénizace či teorie světových systémů pro svou přílišnou obecnost, Dietler se namísto velkých dualistních modelů zaměřuje na formování lokální identity a různé reakce obyvatelstva na kontaktní situace. Dietlerovou inovací je zdůraznění nutnosti sledovat daný fenomén v původním kontextu, s přihlédnutím k roli, jakou daný prvek měl v původní kultuře, a jaké místo zaujímá v kultuře nově formované. Dietler tak navazuje na antropologický směr přisuzující materiální kultuře aktivní roli při formování společenských struktur, jinak též znám jako tzv. {\em social life of things} (Appadurai 1986).

Dietlera zajímají zejména důsledky prvotních kontaktů a jejich projevy na materiální kulturu (1998, 218-219).\footnote{Dietler 1998, 218: „{\em Examination of the initial phase of the colonial encounter in Iron Age France is important precisely because it holds the promise of revealing the specific historical processes that resulted in the entanglement of indigenous and colonial societies and how the early experience of interaction established the cultural and social conditions from which other, often unanticipated, kinds of colonial relationship developed.}” Michael Dietler na příkladu archeologického materiálu rané doby železné z oblasti jižní Galie a Germánie podél toku řeky Rhôny sleduje projevy mezikulturní výměny na materiální kulturu. Na tomto území docházelo k setkávání velkého množství skupin, zahrnující např. Etrusky, Féničany, Řeky, Kelty a v neposlední řadě Římany.} Dietlerova interpretace materiálu může být charakterizována jako přístup „zdola”, tedy přistupující přímo k interpretaci materiálních pramenů, oproti obecně pojatým modelům, které k materiálu přistupují „shora”, tedy čtením dochovaných literárních pramenů. Hlavním zkoumaným materiálem je pro Deitlera keramika, zejména keramika používaná ke konzumaci alkoholu v aristokratických kontextech.\footnote{Téma konzumace se ostatně prolíná i Dietlerovou metodologií: Dietler sám používá termín {\em consumption} pro popis kontextu v jakém materiální kultura splňuje své poslání a je využívána ke svému účelu čili jakou roli zaujímá v rámci zkoumané společnosti, případně jak se vyvíjí s měnícími se společensko-politickými poměry (1998, 219-221).} Volba konkrétního stylu, materiálu, symbolického systému, a tedy i jejich případná změna, je pojímána jako vědomý akt, volba na úrovni jednotlivce, či místní komunity, které však může být motivována celospolečenskými jevy. Změna je podmíněna různými motivy, jejichž objasnění není vždy snadné, či dokonce možné. Příkladem může být nádoba považovaná v jedné kultuře za téměř bezcennou, zatímco v kultuře druhé za exotický předmět, jemuž je přisuzována zvláštní hodnota (Appadurai 1986, 6-16). Role, jakou předmět v kultuře hraje, je do velké míry jeho společenskými a symbolickými konotacemi. Aby mohl být předmět vnímán jako hodnotný, a tedy i vhodný ke „konzumaci” v rámci dané komunity, musí splňovat požadavky dané konkrétním společensko-politickým uspořádáním a systémem hodnot. Tento systém hodnot bývá často interpretován jako vkus, či materializovaný projev životního stylu ({\em habitus} v bourdieovské teorii; Bourdieu 1977).

{\em Habitus}, tak jak ho definuje Bourdieu (1977), je souhrnem všech dispozic a předpokladů, které utvářejí životní styl a světonázor každého daného jedince. {\em Habitus} se nevědomě formuje v závislosti na společenských konvencích a struktuře společnosti, které člověka obklopuje, či obklopovala v minulosti. Nevyhnutelným projevem {\em habitu} každého člověka je materiální svět, který si sám vědomě okolo sebe vytváří v závislosti na vkusu a osobních preferencích (Bourdieu 1984, 173-5). Lidé pocházející z podobného prostředí, mají zpravidla podobné zvyklosti a podobá se i materiální kultura, kterou se obklopují. Jakmile dojde ke změně jedné či více z okolností (kontextů), které formují {\em habitus}, musí časem dojít i k proměně materiální stopy, kterou po sobě člověk zanechává (Sapiro 2015, 487). Může se jednat o změnu ekonomických podmínek, změnu společenského statutu, či se jedná o reakci na kontakt s cizí materiální kulturou, či zavedení nové technologie apod. Jinými slovy, plošné změny ve společenském uspořádání by se měly taktéž projevit na materiální kultuře a její produkci.

Kontextualizace, jak ji navrhuje Dietler, představuje poměrně komplexní analytický model, který vyžaduje systematické zhodnocení jak lokálních kontextů, definice role importovaných předmětů, tak i kvantitativní zhodnocení materiální kultury a její a časoprostorové rozmístění (Dietler 1998, 221). Stejný přístup může být velice dobře použit i na studium nápisů a z nich plynoucích společenských změn. Nápisy, v daleko větší míře než keramické nádoby, poskytují informace o struktuře společnosti, místních identitách a proměnách vkusu dané komunity.

