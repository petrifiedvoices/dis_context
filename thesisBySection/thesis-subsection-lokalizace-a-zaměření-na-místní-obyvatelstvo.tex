
\subsection[lokalizace-a-zaměření-na-místní-obyvatelstvo]{Lokalizace a zaměření na místní obyvatelstvo}

Postkoloniální teoretické přístupy podtrhují fluiditu mezikulturní výměny, hybnou sílu procesů a aktivní roli místní komunity v reakci na mezikulturní kontakty. Nově popsaným fenoménem je vznik tzv. smíšených společností, kdy nově vzniklá kultura v sobě nese prvky obou původních společností, která je neustále obnovována na základě vzájemných interakcí všech zúčastněných stran (White 1991; 2011). Nový pohled na mísení kultur tak nabízí zcela nové interpretace na interagující společnosti - obě kultury jsou nahlíženy jako sobě rovné, chybí prvek dominance jedné z nich a zcela zásadní je zde kreativní prvek, tedy vytváření nové kultury namísto přejímaní kultury dominantní společnosti.

Badatelé zabývající se mezikulturními vztahy hledali nové přístupy k popsání mnohdy velmi komplikovaných situací. Jedním z nejvlivnější přístupů je tzv. {\em middle-ground theory} amerického historika Richarda Whitea (1991; 2011). White, jakožto historik zabývající se kolonizací západní části amerického kontinentu a vzájemnými vztahy mezi Indiány a kolonizujícími bělochy, rozhodně netušil, jak dalekosáhlé důsledky a jak velké uplatnění jeho dílo bude mít i na poli středomořské archeologie. Základní myšlenkou je teorie o místě setkávání dvou kultur, které však nepatří ani do jedné z nich (tzv. {\em middle-ground}). White pojímal {\em middle-ground} jako reálné místo kontaktů, ale i přeneseně jako {\em ad hoc} vzniklý symbolický prostor s prvky z obou zúčastněných kultur, který byl ale plný vzájemných nedorozumění a nových významů (White 2011, xii).\footnote{V původním whiteovském použití {\em middle ground} vzniká jako reakce na obchodní kontakty mezi původními kmeny a příchozími francouzskými kolonizátory, které byly mnohdy plné násilí, vzájemného neporozumění, ale zároveň absence převahy jedné ze zúčastněných stran. Nicméně dokud se obě strany vzájemně potřebovaly, ať už z čistě obchodního hlediska, snažily se dosáhnout určité shody. Tím vznikl poměrně křehký stav, kde se obě strany snažily balancovat vzájemné vztahy a zároveň udržet života schopnou komunitu, např. zajištěním půdy, obživy atp.} {\em Middle-ground} je chápáno jako dočasný fenomén, který zaniká, pokud se jedna ze zúčastněných stran přestane podílet na společenské interakci, či získá dominanci nad druhou stranou (Bayman 2010, 132). White tedy formování {\em middle-ground} chápal jako neustálý proces vytváření symbolického systému porozumění, typický pro daný čas a dané místo, který nelze dost dobře aplikovat na jiné situace (White 2011, xiii). Ač je tento teoretický přístup možné aplikovat pouze na velmi malé množství situací, otevírá zcela nový prostor k interpretacím a poukazuje na aktivní roli obou zúčastněných stran a na specifika mezikulturní výměny.

„{\em Middle-ground theory}” se stala velice oblíbeným interpretačním rámcem i pro prostředí středomořské archeologie a historie (Malkin 1998; 2011; Woolf 2009; Antonaccio 2013). V kontextu Středomoří badatelé interpretují {\em middle-ground} jako novou kulturu, která sice pochází z obou původních kultur, a uchovává si dlouhodobý, avšak proměnlivý, charakter (Malkin 1998; 2011).\footnote{Malkin aplikuje {\em middle-ground} v kombinaci s tzv. {\em network theory}, tedy teorii o decentralizované řecké společnosti, kde k hlavním interakcím dochází v místech setkávání - přístavech a nadregionálních svatyních (2011, 45-48). Pro Malkina je každé takovéto místo setkávání (angl. {\em node}, {\em cluster}) zároveň místem, kde konstantně dochází k formování nové kultury, tedy {\em middle-ground}. Malkinův koncept má spíše popisný než interpretační charakter, avšak i tak výrazně ovlivnil současnou akademickou debatu. Důraz se tak začal klást nejen na propojenost Středozemního prostoru (Malkin {\em et al.} 2009; Constantakopoulou 2007; Archibald 2013, 96-97), ale i na jeho decentralizaci a roli lokálních komunit.} Irad Malkin interpretuje {\em middle-ground} jako nově vznikající kulturu v místech kontaktu, tj. například v koloniích a jejich bezprostředním okolí, či v emporiích, kde docházelo k setkávání velkého množství skupin z odlišných kultur. Malkin, a po jeho vzoru i další badatelé, se uchýlili k tomuto modelu, protože stírá vzájemné odlišnosti a protiklady typu Řek vs. barbar, ale zároveň nechává dostatek prostoru pro různé druhy interakcí a vysvětluje vznik nových smíšených kultur (Antonaccio 2013, 239).\footnote{Alternativní pohled představuje Greg Woolf (2009, 224), který chápe {\em middle-ground} jako svět prezentovaný antickými geografy a etnografy, v čele s Hérodotem. Je to svět, který není ani jedním ze dvou světů a často vzniká ze vzájemného neporozumění, vytržení z původního kontextu. Jedná se o zcela odlišnou interpretaci termínu, než jak ho navrhoval White, nicméně Woolfovo pojetí vrhá nový úhel pohledu na naše vnímání antických etnografických textů a jejich relevanci pro popis dané kultury.}

Použitelnost konceptu {\em middle-ground} je v kontextu řecké kolonizace omezená jen na určitý druh situací a nedá se obecně aplikovat na veškeré mezikulturní kontakty. Zásadní přínos {\em middle-ground theory} je odklon o dřívějších binárních modelů, které předpokládaly jednosměrnou výměnu s jasnými výsledky v přijímající kultuře. Poukázání na fakt, že reakce místního obyvatelstva nebyla vždy jasně daná, a na jeho aktivní účast, zcela pozměnila paradigma současného přístupu k mezikulturním kontaktům.

