
\subsection[makedonská-kolonizace-a-období-hellénismu]{Makedonská kolonizace a období hellénismu}

Již před polovinou 4. st. př. n. l. Filip II. Makedonský podnikl několik výprav na pobřeží egejské Thrákie, kde ovládl řecká města, zmocnil se zdrojů nerostných surovin a umístil zde vojenské posádky a o několik let později se vypravil do vnitrozemí Thrákie, aby si podrobil nebezpečné a výbojné thrácké kmeny a zabezpečil oblast bezprostředně sousedící s Makedonií (Worthington 2015, 76). Filip tak využil ve svůj prospěch nejednotnosti thráckých kmenů, k níž došlo po smrti Kotya I. a nakonec se v roce 340 př. n. l. zmocnil větší části území vnitrozemské a pobřežní Thrákie.\footnote{Makedonci podnikli výpravy nejen proti Odrysům, ale i proti Bessům či Maidům v údolí Strýmónu a nějakou dobu neúspěšně obléhali Perinthos a Byzantion a válčili s kmenem Triballů.} Thrákie se tak stala nedílnou součástí pozdější makedonské říše a odryští králové byli dočasně poraženi. Bohužel nemáme přesné informace o rozsahu makedonské moci v Thrákii, ale předpokládá se, že zhruba od roku 340 př. n. l. většina území plně spadala pod makedonskou administrativu, systém výběru daní a docházelo i k verbování thrácké populace do makedonské armády. Na území dohlížel ustanovený makedonský vojevůdce, kterým se v roce 323 př. n. l. stal Lýsimachos ({\em stratégos}; D. S. 17.62.5; Arr. {\em Anab.} 1.25.2; Archibald 1998, 231-239; Delev 2015, 49-53; Worthington 2015, 76).

Během pobytu v thráckém vnitrozemí Filip II. založil několik vojensky zaměřených osídlení na místě existujících thráckých sídlišť (Dem. 8.44, 10.15). Tato místa se postupem času rozrostla na sídla městského charakteru, jak je patrné na příkladě Kabylé nebo původní thráckého sídla {\em Pulpudeva}, která je od této doby známá pod jménem Filippopolis. Podobným způsobem vznikla i Hérakleia Sintská na řece Strýmón, kdy se původně obchodní osada s vojenskou posádkou rozrostla až do podoby hellénistického města (Nankov 2015, 26-27). V těchto prvních sídlech městského charakteru v thráckém vnitrozemí spolu pravděpodobně žili jak nově příchozí Makedonci, tak původní Thrákové. Jedinečný charakter těchto míst umožnil setkávání několika kultur na každodenní bázi, což mělo za následek i rozšíření epigrafické produkce do thráckého vnitrozemí.

Z literárních zdrojů víme o několika pokusech o osamostatnění se z řad makedonských vojevůdců, které však byly vesměs potlačeny (tzv. Memnónova revolta v roce 331/0 př. n. l., D. S. 17.62-63). Nicméně ani thráčtí aristokraté se nehodlali smířit s makedonskou nadvládou a docházelo ke konfliktům mezi Makedonci a Odrysy.\footnote{Příkladem thráckého odboje je konflikt mezi makedonským Lýsimachem a thráckým Seuthem III., který pravděpodobně vyústil v určitou nezávislost Seuthova postavení (D. S. 18.14.2-4; Tacheva 2000, 12-15; Delev 2015, 53-55). Lýsimachos nicméně i přes Seuthovo nezávislé postavení ovládal velkou část Thrákie a na Thráckém Chersonésu založil město Lýsimacheia (Jones 1971, 5).}

Odryský panovník Seuthés III., který žil na přelomu 4. a 3. st. př. n. l., je známý především díky objevu výstavné rezidence na řece Tonzos, která nesla jeho jméno, Seuthopolis, a díky objevu monumentálních mohylových hrobek v Kazanlackém údolí, patřícím Seuthovi a jeho rodině (Dimitrov, Čičikova a Alexieva 1978; Dimitrova 2015; Delev 2015, 53-54). Dle nalezeného archeologického a epigrafického materiálu se zdá, že Seuthopolis byla osídlena jak thráckým, tak řeckým či makedonským obyvatelstvem, které se po dobu existence rezidence podílelo na vzniku unikátní kombinace kultur a zvyklostí. Seuthopolis, ač se nachází uprostřed thráckého vnitrozemí, nesla všechny charakteristiky typické pro hellénistická sídla tehdejší doby, počínaje užitou architekturou s domy typu {\em pastas} a {\em prostas}, nalezenou keramikou řeckého a místního původu, precizně zhotovenou toreutikou, sochařskou a dekorativní výzdobou a zejména unikátními epigrafickými nálezy (Tacheva 2000, 25-35). Krátký časový horizont existence Seuthopole nicméně potvrdil, že tamní unikátní společnost plně závisela na osobě panovníka Seutha III. a po jeho smrti na počátku 3. st. př. n. l. došlo k poměrně rychlému úpadku aktivit, vymizení jak materiální, tak epigrafické produkce a zániku tohoto jedinečného příkladu hellénistické kultury v samém středu thráckého území (Nankov 2012, 120; Janouchová 2017, v tisku).

Ve 3. st. př. n. l. byla Thrákie svědkem válečných konfliktů diadochů, kteří se snažili ovládnout tuto významnou spojnici mezi Evropou a Asií. Asi nejvýznamnějším byl konflikt mezi makedonským Lýsimachem, původně místodržícím v Thrákii, který se stal vládcem evropské části makedonské říše a samozvaným králem Thrákie, a Seleukem I., jemuž v roce 281 př. n. l. nakonec Lýsimachos podlehl a Thrákie se stala součástí seleukovské říše (Samsaris 1980, 33-34). V následujících letech pokračovaly konflikty následovnických rodů, které z větší či menší části zahrnovaly i území Thrákie. Z této doby pocházejí mince seleukovských panovníků ražené na území Thrákie, např. stříbrné mince z Kabylé, ale i drobné seleukovské mince nesoucí kontramarky Kabylé, určené pro lokální trh (Draganov 1991, 198-208; Draganov 1993, 87-99). Řecká města na pobřeží se v polovině 3. st. př. n. l. dostala do vlivu Ptolemaiovců, což vysvětluje výskyt původně egyptských božstev a motivů ve figurálním umění a na nápisech (Delev 2015, 61).\footnote{Jones 1971, 6: Pod vliv Ptolemaiovců se dostal Ainos, Maróneia a část Chersonésu s Lýsimacheiou.}

V průběhu 3. st. př. n. l. do Thrákie v několika vlnách vpadly keltské kmeny, což s sebou neslo významné změny ve fungování mnoha významných osídlení či dokonce jejich zánik, jako v případě Seuthopole, či {\em emporia} Pistiros. Keltové se na území Thrákie na určitou dobu usadili, a dokonce si postavili nové hlavní město Tylis, které existovalo až do roku 213 př. n. l. (Theodossiev 2011, 15). Přítomnost Keltů měla vliv na thrácké umění, zejména na toreutiku, kde je možné sledovat výskyt nových motivů, ale i technologií. Keltové dokonce razili vlastní mince a využívali k tomu již existující infrastrukturu, čehož jsou důkazem např. mince keltského panovníka z Kabylé, Mesámbrie či Odéssu (Draganov 1993, 107).

Nedostatek pramenů neumožňuje podrobně rekonstruovat následující vývoj, a ve výkladu zůstává mnoho bílých míst. Thrákie zůstala ve sféře vlivu Makedonie až do poloviny 2. st. př. n. l. I nadále docházelo k politickým konfliktům na území Thrákie a ve větší míře zde operovaly makedonské armády, nicméně makedonská administrativa fungovala bez větších změn. Během 2. st. př. n. l. se začal také pomalu stupňovat politický vliv Říma na dění v Thrákii, který se zintenzivnil po roce 148-146 př. n. l., kdy došlo k ovládnutí Makedonie Římem (Eckstein 2010, 248). Thráčtí Odrysové využili situace a stali se spojenci a podporovatelé Říma (T. Liv. 45.42.6-12; Delev 2015, 63-68). Řecká města na pobřeží zčásti spadala pod římskou provincii {\em Macedonia}, stejně tak jako vnitrozemská Hérakleia Sintská, a z části si udržela autonomní pozici, jako např. Abdéra, Maróneia a Ainos (T. Liv. 45.29.5-7).\footnote{Jones 1971, 7: města Ainos a Maróneia spolu s pobřežními oblastmi se opět stala součástí Makedonie. V roce 168 př. n. l. se obě města stala autonomními politickými autoritami.} Byzantion v roce 148 př. n. l. uzavřel spojeneckou smlouvu, což vedlo k nárůstu politické moci Říma v regionu a zároveň i rozvoji Byzantia (Tac. {\em Ann.} 12.62-63; Jones 1971, 7). Římská přítomnost v regionu byla patrná zejména podél cesty {\em Via Egnatia}, která spojovala oblast západního a východního balkánského poloostrova. Oblast Thráckého Chersonésu v té době spadala pod pergamské království a byla spravována vojenskými veliteli ({\em stratégy}), jak dokazují dochované nápisy (Delev 2015, 68). Se zánikem pergamského království se v r. 133 př. n. l. Thrácký Chersonésos stal součástí římské provincie {\em Asia}. V první polovině 1. st. př. n. l. se část Thráků a řeckých měst přidala na stranu pontského krále Mithridata VI., což mělo za následek římské vojenské akce na území Thrákie ve snaze zajistit stabilitu v oblasti a eliminovat případný odpor místních kmenů (Lozanov 2015, 76-77). V této době se tak vytvořil prostor pro iniciativní jedince z řad thrácké aristokracie, kteří za svou podporu Říma získali výsadní postavení.

