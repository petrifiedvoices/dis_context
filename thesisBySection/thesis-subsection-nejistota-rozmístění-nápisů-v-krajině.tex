
\subsection[nejistota-rozmístění-nápisů-v-krajině]{Nejistota rozmístění nápisů v krajině}

Dalším problémem, s nímž bylo nutné se vypořádat, je nejistota spojená s určením místa nálezu nápisu. Ve velkém množství případů epigrafické korpusy udávají pouze přibližnou polohu, kde byl nápis nalezen, a není ho tak možné spojovat s konkrétní archeologickou lokalitou. Pro provedení analýzy rozmístění nápisů, která bere v potaz vzájemné prostorové vztahy a zasazení epigrafické produkce do kontextu zeměpisných podmínek, bylo nutné se vypořádat s mírou nejistoty při určování nejpravděpodobnějšího místa nálezu nápisu.

Za tímto účelem jsem vytvořila tzv. {\em position certainty index}, tedy koeficient přesnosti určení míst nálezu, který určuje velikost území, na němž byl nápis nejpravděpodobněji nalezen.\footnote{Tabulka 4.04 v Apendixu 1 přehledně shrnuje hodnoty koeficientu a celkového počtu nápisů jednotlivých skupin.} Při stanovování velikosti území jsem vycházela z informací poskytovaných editory jednotlivých autorů, zejména pak {\em IG Bulg}. Pokud bylo možné místo spojit s konkrétní archeologickou lokalitou, přesnost míry určení místa jsem stanovila ve vzdálenosti do 1 km, nesoucí koeficient 1.\footnote{Příkladem udávání místa nálezu, které bylo možno spojit s konkrétní archeologickou lokalitou je např. {\em IG Bulg} 118: „{\em Odessi reperta in via Prespa}.”; {\em I Aeg Thrace} 195: „Προέρχεται άπο τήν θέση Μάρμαρα της αρχαίας Μαρώνειας.”} Pokud autor korpusu udal místo nálezu v okolí moderního sídla s udáním konkrétních vzdáleností a směru, kde byl nápis nalezen, místo nálezu se s největší pravděpodobností nacházelo v okruhu 5 km od tohoto moderního sídla.\footnote{Příkladem relativně přesného místa nálezu je např. {\em IG Bulg} 3,2 1843: „{\em Repertus in agro quodam ad vicum nunc Dobrinovo, olim Hasbeglij dicto}.”; {\em IG Bulg} 4 2014: „{\em Reperta 1 km orientem versus a vico Gurmazovo, conservabatur penes vicanum eiusdem vici Pane Gjorev}.”} Pokud editor korpusu udal pouze všeobecnou informaci o nálezovém místě a jeho poloze vůči moderním sídlům, oblast nejpravděpodobnějšího místa nálezu byla stanovena do vzdálenosti 20 km od uvedeného moderního sídla.\footnote{Příkladem obecného udávání místa nálezu je např. {\em IG Bulg 4} 2034: „{\em Reperta ad vicum Dragoman}.”; {\em IK Byzantion} 159: „{\em Gefunden in Istanbul.}”} Skupina nápisů, jejichž místo nálezu je známé jen velmi obecně, či vůbec, ale editoři nápisu udávají, že pochází z Thrákie, nese koeficient 4, což značí velmi malou míru pravděpodobnosti konkrétního zeměpisného určení.\footnote{Příkladem nápisu s neznámou lokalitou je {\em IG Bulg} 5 5927: „{\em Thrace, région indéterminée}.”} Nápisy z této poslední skupiny vynechávám ze všech analýz rozmístění v kapitole 7, vzhledem k velmi malé výpovědní hodnotě o místě nálezu a prostorovém uspořádání těchto nápisů v krajině.

