
\subsection[období-vazalských-králů]{Období vazalských králů}

V 1. st. př. n. l. se prosadilo několik thráckých aristokratických dynastií, které získaly své postavení zejména spoluprací s Římem a došlo k vytvoření tzv. vazalských království, podobně jako např. v oblasti Británie či Judeje. Thrákové poskytovali Římu zejména vojenskou a materiální pomoc a výměnou se jim dostalo relativní autonomie a osobních výhod.\footnote{David Braund (1984, 23-29) uvádí, že vazalští králové často obdrželi jak římské občanství a jiné posty, za které byli ochotni zaplatit nemalé finanční částky. Spolu s poctami často dostávali i luxusní dary, jako oděvy a znaky moci, které zvyšovaly jejich společenskou prestiž a upevňovaly postavení krále.} Mezi nejvýznamnější kmeny této doby patří Odrysové, Sapaiové a Astové (Lozanov 2015, 78).

Odryští králové byli u moci přibližně do poloviny 1. st. př. n. l. (Manov 2002, 627-631) a z historických pramenů víme, že po roce 42 př. n. l. thrácký Rhéskúporis I. založil sapajskou dynastii se sídlem v Bizyi v jihovýchodní Thrákii a vládl v letech 48 - 41 př. n. l. (Strabo 7, frg. 47; 12.3.29; Jones 1971, 9-10). Bezprostředně po něm není následnická linie zcela jasná, nicméně se sapajští králové udrželi u moci až do r. 46 n. l.\footnote{Lozanov 2015, (78-80); Manov 2002 (627-631): přesné roky vlády jednotlivých králů jsou i nadále předmětem akademické debaty. Z literárních pramenů známe jména a přibližnou dobu vlády následujících panovníků: Rhéskúporis I (48-42 př. n. l.), Kotys (42-18? př. n. l.), Rhéskúporis II. (18-13? př. n. l.), Rhoimetalkás I. (13 př. n. l. - 11 n. l.), Kotys (12-19 n. l.), Rhoimetalkás II. (18-38 n. l.) a Rhoimetalkás III (38-45 či 46 n. l.).} Posledním thráckým králem se stal Rhoimetalkás III., který vládl v letech 38 až 46 n. l. Ač byla dynastie Sapaiů teoreticky nezávislá, v praxi se jednalo o panovníky dosazené k moci Římem, který sledoval své vlastní mocenské zájmy. Z epigrafických a literárních zdrojů víme, že území Thrákie bylo rozděleno na samosprávní jednotky, tzv. stratégie, které spravovali stratégové, původem thráčtí aristokraté, či loajální Řekové. Rozdělení do stratégií mělo primárně usnadňovat administrativu a sloužilo i pro zvýšení vojenské kontrola území či verbování jednotek (Lozanov 2015, 78-79). Za vlády sapajské dynastie mnoho thráckých mužů vstoupilo do římské armády jako příslušníci pomocných jednotek (Tac. {\em Ann.} 4.47). Nejen, že Thrákové v této době sloužili v římské armádě, ale římská armáda operovala na území Thrákie.\footnote{Z literárních pramenů se dozvídáme, že místodržící v Makedonii Marcus Lucullus v roce 72/1 př. n. l. porazil kmen Bessů a dobyl nejen Kabylé, ale zmocnil se i měst na pobřeží Černého moře a založil zde vojenské posádky (Eutrop. {\em Breviarium} 6.10).}

Postupný nárůst moci Říma měl za následek transformaci vazalských království do podoby římské provincie. Sever území okolo řeky Istros se stal součástí římské říše pravděpodobně okolo r. 15. n. l. jako provincie {\em Moesia Inferior}. O 30 let později po smrti Rhoimetalka III. v roce 45 či 46 n. l. Řím využil příležitosti a chaosem zmítané území vnitrozemské Thrákie přeměnil na provincii {\em Thracia}, která tak plně spadala pod autoritu římského císaře (Lozanov 2015, 76-80).

