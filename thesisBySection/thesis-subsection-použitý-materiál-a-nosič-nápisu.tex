
\subsection[použitý-materiál-a-nosič-nápisu]{Použitý materiál a nosič nápisu}

Převážná většina nápisů byla zhotovena z kamene (4518 nápisů, což představuje 97 \letterpercent{}), a to především z mramoru (3292 nápisů), vápence (652 nápisů), syenitu (53 nápisů), pískovce (45 nápisů) a jiných, vesměs lokálních variant použitého kamene.\footnote{Agniezka Tomas (2016, 37-40) poukazuje na využití místního kamene na nápisech z okolí římského tábora v Novae. Převážně jsou využívány místní zdroje vápence a pískovce, pocházející jednak z bezprostředního okolí Novae a jednak z lokalit dostupných proti proudu řeky Jantra směrem na jih. Místní zdroj kamene byl tak využíván nejen pro epigrafické aktivity, ale zejména pro stavbu budov a jiných staveb nejen od druhé poloviny 1. st. n. l., jak naznačují dochované nápisy, ale pravděpodobněji již dříve. Tomas zaznamenává šest míst v regionu Novae, kde byly nápisy vyráběny ve větším měřítku, kde umísťuje i produkční dílny: Butovo, Novae, Dimum, Nicopolis ad Istrum, Karaisen a Pejčinovo - vše lokality zhruba v okruhu 40 km s výjimkou Nicopolis ad Istrum.} Z materiálů jiných než kámen se jednalo o 24 nápisů na kovu,\footnote{Jako je zlato, stříbro, olovo a editorem blíže neurčený kov.} dále 15 nápisů bylo na keramice, jeden nápis na dřevě, a šest na jiném druhu materiálu, jako jsou například mozaiky, či jako součást nástěnné malby. Téměř dvě třetiny všech objektů nesoucích nápis byly dekorované (2904 nápisů, 65,04 \letterpercent{}). Převládající typ dekorace byla reliéfní výzdoba (64,32 \letterpercent{} všech objektů nesoucích nápis, tedy 98,89 \letterpercent{} objektů s dekorací), nápisy s dochovanou malbou či jiným typem dekorace tvořily dohromady pouze 0,72 \letterpercent{} všech nápisů. Mezi nejčastější typy reliéfní výzdoby patřila výzdoba figurální (39,14 \letterpercent{} ze všech objektů nesoucích nápis), dále pak architektonická výzdoba či výzdoba ornamentální (23,88 \letterpercent{} ze všech objektů nesoucích nápis).\footnote{Vysvětlení jednotlivých pojmů a jejich obrazová galerie je součástí Apendixu 1. Jednotlivým druhům dekorace a jejich výskytu se věnuji podrobněji v následujících chronologicky řazených sekcích v této kapitole.} Nápisy tesané do kamene a jejich dekorace prochází v průběhu staletí vývojem směrem od prostých stél s jednoduchou florální či malovanou dekorací z klasické a hellénistické doby, až po složité reliéfní dekorace a nejrůznější motivy, nacházející se jak na stélách, tak na architektonických prvcích jako jsou sloupy, oltáře či dokonce sarkofágy v době římské.\footnote{Paradoxně k největšímu rozšíření reliéfních vyobrazení řeckých božstev a scén z řecké mytologie dochází až v době římské, nikoliv v době hellénismu, jak by se dalo očekávat.} Stupeň a způsob dochování objektů nesoucí nápis, jak je uvádějí autoři korpusu, či jak bylo patrné z přiložené vizuální dokumentace, jako jsou fotografie, kresby, oklepky apod. podrobně dokumentuje tabulka 5.01 v Apendixu 1. Je zřejmé, že výpovědní hodnota a přesnost interpretace je vyšší u nápisů, které se dochovaly v co možná nejkompletnější podobě. Bohužel u čtvrtiny nápisů nebylo možné určit stupeň dochování, a to zejména kvůli chybějící vizuální dokumentaci, a nejasnému charakteru textu. I přesto se však téměř 70 \letterpercent{} objektů nesoucí nápis dochovalo v takové podobě, že je možné z nich získat data potřebná k další analýze obsahu textu.

