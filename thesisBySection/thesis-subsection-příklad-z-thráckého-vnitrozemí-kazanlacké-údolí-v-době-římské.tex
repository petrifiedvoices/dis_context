
\subsection[příklad-z-thráckého-vnitrozemí-kazanlacké-údolí-v-době-římské]{Příklad z thráckého vnitrozemí: Kazanlacké údolí v době římské}

Jiný druh srovnání archeologických a epigrafických dat nabízí následující příklad z vybraného regionu vnitrozemské Thrákie, Kazanlackého údolí. Srovnání na mikro-regionální úrovni umožňuje porovnat poměr všech známých archeologických lokalit s epigrafickými lokalitami, což je jen velmi těžko dosažitelné na úrovni makro-regionální jako v případě srovnání všech známých lokalit ze 7. až 4. st. př. n. l. Proto jsem jako příklad zvolila dobře ohraničené území ve vnitrozemské Thrákii a zaznamenala všechny známé archeologické lokality a porovnala je se známými místy nálezů nápisů.\footnote{Pro tyto účely jsem zvolila oblast Kazanlackého údolí ve střední části Bulharska. Oblast okolo Kazanlaku je známá jako kulturní a historické centrum thráckých panovníků, kteří udržovali čilé kontakty s řeckými obcemi. V letech 2009-2011 zde probíhaly povrchové sběry projektu {\em The Tundzha Regional Archaeological Project} (TRAP; Sobotkova {\em et al.} 2010; Ross {\em et al.} v přípravě, vyjde 2017), během nichž se podařilo zmapovat osídlení ve větší části Kazanlackého údolí. Výsledky těchto sběrů tak představují výborný výchozí soubor dat, pokrývající větší část vybraného regionu v jeho komplexnosti a zaznamenávají jak viditelné monumenty, tak i koncentrace keramiky a architektonických prvků na povrchu. Pro úplnost porovnávám data i se soupisy archeologických lokalit, které byly pořízeny pro Kazanlak v roce 1991 (Domaradzki 1991; Tabakova-Tsanova 1991) a výstupy archeologických vykopávek na zkoumaném území (Chichikova, Dimitrov a Alexieva 1978; Tabakova 1959; Tabakova-Tsanova 1961, 1980; Dinchev 1997; Nekhrizov {\em et al.} 2013). Toto srovnání vychází ze studie, která vyjde v roce 2017 rámci sborníku z projektu {\em The Tundzha Regional Archaeological Project} (Janouchová v přípravě, vyjde 2017).}

Projekt {\em The Tundzha Regional Archaeological Project}, jehož závěry používám jako výchozí data, zaznamenal v průběhu let 2009 až 2011 celkem 82 archeologických lokalit a 773 mohyl na území o rozloze 85 km\high{2} (Sobotková v přípravě, vyjde 2017). Pro dobu pozdně železnou (500 - 0 př. n. l.) bylo nalezeno 38 lokalit a pro dobu římskou (1 - 400 n. l.) 23 lokalit, což v průměru znamená výskyt jedné lokality na 2,23 km\high{2} v době pozdně železné a na 3,69 km\high{2} v době římské. Jak je patrné z tabulky 7.03 v Apendixu 1, z vybraného území se dochovalo celkem 43 nápisů, z nichž osm spadalo do doby pozdně železné a bylo nalezeno ve čtyřech lokalitách s centrem v Seuthopoli, hellénistické rezidenci odryského panovníka Seutha (Dimitrov, Chichikova a Alexieva 1978, 3-5). Tyto nápisy poukazují na prominentní roli Seuthopole a zcela ojedinělý přístup k publikaci nápisů a řecké kultuře obecně, který panovník Seuthés zaujímal (Janouchová v přípravě, vyjde 2017). Nápisy z Kazanlackého údolí v hellénistické době pocházely pouze z kontextů spojených s panovníkem Seuthem a zvyk veřejně vystavovat nápisy tesané do kamene, jak je obvyklé v řeckých komunitách, po jeho smrti postupně vymizel z thráckého prostředí na několik dalších století. Nicméně povědomí o užívání písma přetrvalo alespoň v prostředí thrácké aristokracie v mírně změněné formě, která předměty nesoucí nápisy využívala pro soukromé účely, jakožto prestižní předmět, zdůrazňující jejich společenský status (Sahlins 1963; Whitley 1991, 349-350).

Z doby římské pocházelo 35 nápisů z pěti lokalit, z čehož většina nápisů pocházela ze svatyní Apollóna, nesoucího místní přízviska {\em Teradéenos} a {\em Zerdénos}, umístěných v okolí moderní vesnice Kran (Tabakova 1959, 97-104; Tabakova-Tsanova 1980, 173-194). Z toho plyne, že ze známých lokalit doby pozdně železné 10,5 \letterpercent{} obsahovalo minimálně jeden nápis. V době římské se tento poměr zvýšil na dvojnásobek na 21,7 \letterpercent{}. Pokud vezmeme v potaz charakter jednotlivých epigrafických lokalit, v římské době většina nápisů z oblasti pozemních sběrů v rámci projektu TRAP pochází ze svatyní Apollóna nesoucího místní přízvisko. Většina nápisů má soukromý charakter a jejich zhotoviteli jsou osoby nesoucí převážně thrácká či kombinovaná římská a thrácká jména, což poukazuje na jejich zapojení v římské armádě a samosprávě měst (Janouchová v přípravě, vyjde 2017). V římské době tedy nárůst počtu epigrafických lokalit odpovídá i většímu zapojení místního obyvatelstva do epigrafické produkce, která i přesto zůstávala nadále poměrně nízká.

