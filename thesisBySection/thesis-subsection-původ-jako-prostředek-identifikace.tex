
\subsection[původ-jako-prostředek-identifikace]{Původ jako prostředek identifikace}

Označení původu je po osobním jméně druhý nejčastější způsob identifikace jedince a slouží jako bezprostřední zasazení do nejbližší komunity, či naopak jeho vymezení se vůči ní. Původ je vlastní každému člověku, bez ohledu na jeho volbu, a provází ho od narození po celý život. Původ člověka do značné míry utváří jeho pohled na svět a ovlivňuje i projevy v rámci epigrafické produkce. Nejtypičtějším vyjádřením původu je odkaz na biologický či geografický původ dané osoby.

Nejjednodušší označení biologického původu je označení rodičů, případně prarodičů. K identifikaci osoby se v řeckém světě mohlo se však užívat i jméno jiného rodinného příslušníka, např. matky, manžela, bratra (Fraser 2000, 150). Ženy většinou uváděly jméno svého otce a poté jméno manžela, případně jen jednoho z nich (McLean 2002, 94).\footnote{Areté, dcera Helléna, manželka Agathénóra, syna Artemidóra, {\em IG Bulg} 1,2 143, nedatovaný nápis z Odéssu.} Pokud to bylo důležité, mohla se přidávat ještě výjimečně informace o předchozí generaci pro zdůraznění rodové linie.\footnote{Makou, dcera Amyntóra, syna Hierónyma, {\em IG Bulg} 12 127, 2. - 3. st. n. l., z Odéssu. Dalším, poněkud méně rozšířeným, způsobem uvádění biologického původu je vypisování genealogických informací o předcházejících generacích, či o tzv. mýtických předcích a zakladatelích obcí, čím člověk legitimizoval své postavení v komunitě (Hall 2002, 25-29; Malkin 2005, 64-66).} Specifikace biologického původu zároveň ale hraje roli při legitimizaci majetkoprávních nároků při případných dědických sporech, a proto se v epigrafických pramenech velmi často setkáváme s její veřejnou prezentací na nápisech v době římské.\footnote{Podrobně se biologickým původem na nápisech zabývám v kapitola 5.}

Geografický původ je taktéž nedílnou součástí určení identity jednotlivce, která se však projevuje pouze v určitých situacích. Vyjádření geografického původu se nemusí vztahovat jen k místu, odkud člověk pochází, kde se narodil, ale může naznačovat i místo, k němuž se člověk hlásí a přijal ho za své. Člověk tak může vyzdvihovat své rodné město, ale stejně tak město, ve kterém nyní žije a je pro něj důležité zdůraznit přináležitost ke svému současnému bydlišti.\footnote{Jedinec se může vztahovat ke svému rodnému místu, současnému či minulému bydlišti, místu, odkud pochází jeho předci, jako např. Métrodotos, syn Artémóna, původem z Mandry, {\em I Aeg Thrace} 164, nedatovaný nápis z lokality Mitriko v severním Řecku.} K udávání geografického původu většinou docházelo v kontextu kontaktu s jinou kulturou či společností, kde jedinec potřeboval zdůraznit svůj původ, jako např. cizinec žijící na území jiného státu.

Vyjádření původu na nápisech je jedním z důležitých měřítek o míře konzervativismu dané společnosti. V uzavřených společnostech se setkáváme s malou mírou udávání geografického původu, či lidí cizího původu je pouze minimum, ať už z obavy možných persekucí, či z velmi malé míry fluktuace obyvatelstva. Naopak u společností na vyšším stupni společenské komplexity se setkáváme s vyšším počtem vyjádření geografického původu, což poukazuje na větší pohyb obyvatelstva a mnohonárodnostní složení těchto uskupení.

