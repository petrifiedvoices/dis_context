
\subsection[shrnutí-13]{Shrnutí}

Počty dochovaných nápisů z 1. st. př. n. l. poukazují na celkové snížení produkce nápisů a jejich vymizení z vnitrozemí, pravděpodobně související s celkovým úpadkem ekonomické prosperity thráckého vnitrozemí (Lozanov 2015, 84). Hlavním ekonomickým a produkčním centrem je i nadále Byzantion, ale nápisy se v menší míře vydávají i v dalších městech na pobřeží. Řecký prvek hraje i nadále důležitou roli, ale je doplněn jak thráckým, tak římským i dalšími prvky. Vzhledem k narůstající moci Říma se začíná proměňovat i složení epigraficky aktivní populace a její zvyky, ať už se jedná o projevy náboženství či nové onomastické trendy, případně o detailnější obsah funerálních nápisů.

Dochází k proměně na politické scéně, kdy se v 1. st. př. n. l. se opět objevuje thrácká aristokracie, tentokrát v podobě dynastie Sapaiů, Odrysů a Getů. Její přítomnost a projevy v epigrafice se však nepodobají nápisům 5. až 3. st. př. n. l., nicméně mají formu obvyklou spíše u politické autority typu řecké {\em polis}. Je tedy možné říci, že tato nově reformovaná thrácká aristokracie přistoupila na společenské normy a zvyklosti svých nejbližších sousedů a partnerů a jako komunikační strategii zvolila formu veřejných dekretů a usnesení.

