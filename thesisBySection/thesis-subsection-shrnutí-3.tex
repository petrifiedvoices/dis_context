
\subsection[shrnutí-3]{Shrnutí}

Dochovaný vzorek nápisů ze 6. st. př. n. l. je bohužel velmi omezený a na jeho základě není možné hodnotit případné kontakty řecké a thrácké kultury. První nápisy se objevily v řecké komunitě v prvních desetiletích po založení kolonií, a jejich charakter se velmi podobal nápisům z jiných částí řecky mluvícího světa jak formou, tak obsahem. Malý počet dochovaných nápisů může poukazovat na a) nedostatečně a nerovnoměrně archeologicky prozkoumané kulturní vrstvy 6. st. př. n. l., b) nejasný charakter nápisů, který neumožňuje nápisy datovat právě do 6. st. př. n. l., c) na fakt, že společnost řeckých měst z území Thrákie v 6. st. př. n. l. neprodukovala velké množství nápisů, protože se potýkala jednak s interními, tak s externími problémy souvisejícími s osídlováním již obydleného území.

