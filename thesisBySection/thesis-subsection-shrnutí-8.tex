
\subsection[shrnutí-8]{Shrnutí}

Nápisy datované do 4. až 3. st. př. n. l. poukazují na narůstající pronikání epigrafické produkce do thráckého vnitrozemí, které je patrné již u skupiny nápisů datovaných do 4. st. př. n. l. Ve vnitrozemí se nápisy objevují zejména v okolí řeckých a makedonských osídlení, v nichž mohli žít i místní obyvatelé či minimálně se s nimi museli stýkat na každodenní bázi. Nicméně i v okolí těchto sídel si epigrafická produkce udržuje tradiční řecký charakter a dle osobních jmen je do publikační činnosti zapojena pouze populace nesoucí řecká jména a dodržující řecké zvyklosti.

