
\subsection[skupiny-produkčních-center-nápisů]{Skupiny produkčních center nápisů}

Produkční centra je možné charakterizovat jako místa se zvýšenou koncentrací nálezů nápisů. Mapa 7.04 v Apendixu 2 dokumentuje hustotu nalezených nápisů na území Thrákie v podobě tzv. teplotní mapy. Místa s tmavší barvou představují místa s vyšší koncentrací nápisů v okruhu 20 km. Nejtmavší místa jsou velká města jak na pobřeží, tak ve vnitrozemí na důležitých cestách či křižovatkách cest. V horských oblastech je minimum míst s nejvyšší koncentrací nápisů, ale vyskytují se zde lokality se středními hodnotami počtu nápisů. Většina míst s největší koncentrací nápisů je v nejbližším okolí měst, ale v několika případech se setkáváme i s vysokou koncentrací nápisů v lokalitách neměstského charakteru.

\subsubsection[produkční-centra-městského-charakteru]{Produkční centra městského charakteru}

Jak plyne z výše řečeného, největším producentem nápisů jsou města a jejich nejbližší regiony ve vzdálenosti do 20 km. Města je možné rozdělit podle počtu nalezených nápisů na nadregionální producenty s počtem okolo 300 a více nápisů, velké regionální producenty s počty od 150 do 250 nápisů, menší regionální centra s 50 až 149 nápisy a malá produkční centra s 49 a méně nápisy. Jejich polohu v Thrákii a kategorii, do níž dané produkční místo spadalo ilustruje mapa 7.05 v Apendixu 2.

Do kategorie producentů s nadregionálním významem patří města, která svým významem překračovala území Thrákie či zastávala významnou pozici v rámci nadregionální samosprávy. Do této kategorie patří města Byzantion, Odéssos a Filippopolis. Ač nejvíce nápisů bylo nalezeno v černomořském Odéssu, a to celkem 359, toto číslo je ve skutečnosti o pár desítek nápisů nižší, protože do 20 km okruhu okolo Odéssu zasahuje území Marcianopole a Dionýsopole a program QGIS započítal tyto nápisy na pomezí ke všem městům stejně. Nicméně i přesto z regionu Odéssu pochází zhruba 300 nápisů a řadí se tak k jedněm z největších producentů nápisů v Thrákii. Z Byzantia a jeho regionu pochází 352 nápisů, což ho tak řadí na první místo epigrafické produkce.\footnote{Byzantion zastával významnou pozici v době vlády Říma, a to zejména od 4. st. n. l., kdy se stal pod jménem Konstantinopol hlavním městem římské říše (Jones 1971, 23).} Filippopolis s 296 nápisy spadá do kategorie nadregionálních produkčních center, jakožto administrativní a kulturní centrum provincie {\em Thracia}.

Do skupiny velkých regionálních center spadají čtyři města na pobřeží Černého, Marmarského a Egejského moře, která hrála roli administrativní a ekonomického centra daného přímořského regionu. Největším producentem z této skupiny je Perinthos s 248 nápisy, dále Maróneia s 234 nápisy, Apollónia Pontská s 218 nápisy a Mesámbria se 172 nápisy. Tato města zaujímala pozici regionálních center zejména ve stoletích př. n. l., nicméně si jistou míru autonomie a politické moci udržela i v římské době.

Skupina středních regionálních center zahrnuje města jak na mořském pobřeží, tak města vnitrozemská, která se nacházela většinou na křižovatkách cest či významných komunikacích a sloužila jako administrativní a ekonomické centrum pro nejbližší region. Do této kategorie patří Parthicopolis se 147 nápisy, Nicopolis ad Istrum se 136 nápisy, Hérakleia Sintská se 128 nápisy, Augusta Traiana se 124 nápisy, Pautália se 122 nápisy, Serdica se 112 nápisy a Marcianopolis se 122 nápisy.

Skupina malých regionálních center s méně než 30 nápisy zaujímala pouze marginální roli v epigrafické produkci, což však nereflektuje její postavení ve společnosti jako ekonomické či politické centrum okolní oblasti, jak je tomu např. u Ainu či Nicopolis ad Nestum či Bizyé. Tento stav spíše reflektuje nedostatečný stav prozkoumání regionu těchto měst, zejména z důvodu moderní zástavby. V budoucnu se dá tak v těchto městech dají očekávat nálezy nápisů, které by tato města posunula do kategorie středních regionálních producentů.

\subsubsection[produkční-centra-neměstského-charakteru]{Produkční centra neměstského charakteru}

Z již zmiňované teplotní mapy plyne, že místa s velkou hustotou nápisů nebyla pouze městského charakteru, ale dochovalo se i několik lokalit mimo region města s vysokou koncentrací nápisů. V těchto případech se jedná o svatyně umístěné ve volné přírodě, kde bylo nalezené velké množství dedikací. Společnou charakteristikou těchto svatyní byla dobrá dostupnost, a tedy i blízkost cest v případě horských lokalit či nenáročnost terénu v případě svatyň v nížinách a v podhůří, jak je též patrné z mapy 7.06 v Apendixu 2.

Mezi svatyně s největším počtem nápisů patří svatyně v Batkunu s 194 nápisy nalezenými ve vzdálenosti 5 km od místa svatyně Asklépia.\footnote{Tsonchev (1941).} Dále sem patří svatyně Asklépia z lokality Slivnica se 72 nápisy, Glava Panega se 77 nápisy, Daskalovo se 76 nápisy.\footnote{Boteva (1985); Gocheva (1992), Oppermann (2006, 147-154).} Svatyně Apollóna s místními epitety pochází z lokalit Kiril Metodievo s 33 nápisy a Kran taktéž s 33 nápisy.\footnote{Tabakova (1959, 1961), Tabakova-Tsanova (1980).} Svatyně Nymf a Asklépia je známá z lokality Búrdapa, kde bylo nalezeno 47 nápisů.\footnote{Janouchová (2013, 14).} V lokalitě Mezdra bylo nalezeno 17 nápisů věnovaných různým božstvům, mimo jiné Démétér či Diovi. S lokalitě Skaptopara se pak jedná převážně o funerální nápisy náležející k blízkému osídlené - thrácké vesnici Skaptopara.

Většina těchto svatyní pochází ze severozápadní Thrákie z horských oblastí. Nejníže položená lokalita se nachází ve výšce 209 m. n. m a nejvýše položená ve výšce 754 m. n. m. Průměrná nadmořská výška všech devíti lokalit je 404 m. n. m. (aritmetický průměr; medián je 345 m. n. m.). Lokality jsou v těsné blízkosti cest a řek, což usnadňovalo věřícím přístup a pravděpodobně tento fakt hrál roli v jejich oblíbenosti mezi věřícími. Svatyně s nejvíce nápisy se nachází ve vzdálenosti 28, respektive 50 km od města Filippopolis, které po většinu římské doby zastávalo roli hlavního města provincie a jehož obyvatelé s největší pravděpodobností navštěvovali tuto svatyni Nymf v Búrdapě a Asklépia v Batkunu.

Ať už se jedná o nápisy nalezené v regionu měst či v lokalitách neměstského charakteru, zásadní roli na rozmístění lokalit měla existence místní infrastruktury a systému komunikací. Nápisy se objevují zejména ve vzdálenosti dosažitelné v rámci jednoho dne od hlavních administrativních center a podél cest. Čím bylo dané sídlo důležitější a koncentrovaly se v něm politické a administrativní instituce, tím více se v jeho okolí objevilo i nápisů. Ekonomická aktivita jednotlivých měst na celkovou výši epigrafické produkce zásadní vliv neměla, příkladem může být město Ainos, jeden z hlavních říčních a přímořských přístavů s až překvapivě nízkým počtem nápisů. Obecný trendem zůstává, že s přibývající vzdáleností od lidských sídel, narůstající nadmořskou výškou a vzdáleností od cest počet nápisů klesá.

