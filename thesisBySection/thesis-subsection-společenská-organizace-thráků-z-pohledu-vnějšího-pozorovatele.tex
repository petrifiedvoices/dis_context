
\subsection[společenská-organizace-thráků-z-pohledu-vnějšího-pozorovatele]{Společenská organizace Thráků z pohledu vnějšího pozorovatele}

Dle Strabóna (7.7.4) bylo vnitrozemí směrem na východ od řeky Strýmónu obýváno thráckými kmeny, zatímco pobřeží bylo osídleno řecky mluvícími obyvateli, žijícími v městech. Hlavním zdrojem obživy bylo pro Thráky zemědělství, chov koní a dobytka, válečné výpravy a do omezené míry zde fungoval i obchod a nepeněžní výměna se Středomořím (Strabo 7.3.7).

Hérodotos (5.3) uvádí, že Thrákové jsou hned po Indech druhý nejpočetnější národ v tehdy známém světě.\footnote{Ač Hérodotos používá slovo národ či kmen (ἔθνος) a Thrákové byli v řeckém prostředí vnímáni a popisováni jako jedno etnikum, ve skutečnosti se jednalo o mnoho kmenů, které spadaly pod jednotné označení „thrácké”.} Sám Hérodotos uznává, že Thrákové jsou nejednotní, a tím přichází tak o značnou strategickou výhodu, jakou jim velké množství obyvatel přináší. Spojuje je však geografická blízkost obývaných území, a až na výjimky, podobné zvyklosti. Pozdější zdroj Plinius Starší uvádí jména 36 kmenů s relativní polohou oblasti, kterou obývají (Plin. {\em H. N.} 4.11), ale zároveň mluví o existenci 50 stratégií, tedy administrativních jednotek, které pravděpodobně vycházely z kmenového principu (Theodossiev 2011, 8).\footnote{Plin. H. N. 4.11. 40: {\em Thracia sequitur, inter validissimas Europae gentes, in strategias L divisa}.} Strabón uvádí pouze 22 kmenů pro celou Thrákii (7, frg. 48; 7.3.2), avšak dá se předpokládat, že jich celkem bylo až dvakrát tolik. Fol a Spiridonov (1983, 21-61) celkem sesbírali na 50 jmen kmenů, které je možné označit jako thrácké a které se vyskytovaly v písemných pramenech do poloviny 3. st. př. n. l. Některé z kmenů prameny zmiňují jen jednou či ojediněle, o jiných víme poměrně hodně detailů, jako např. o Odrysech, Getech, Sapaích atp.\footnote{Sears (2013, 9-13) uvádí nejdůležitější thrácké kmeny vyskytující se v řeckých literárních pramenech: Apsinthiové, Bessové, Bisaltové, Bíthýnové, Diové, Dolonkové, Édónové, Mygdóni, Thýnové, Odomanti, Odrysové a Satrové.}

Jednotlivým kmenům vládnou kmenoví vůdci či dle řeckých pramenů králové, kteří ve svých rukou soustředili jak politickou, tak ekonomickou moc Archibald (2015, 912).\footnote{Thúkýdidés a Hérodotos používají termín οἱ βασιλέες, králové (Thuc. 2.97; Hdt. 5.7; 6.34); Diodóros pak při popisu odryského panovníka Sitalka používá termín ὁ τῶν Θρᾳκῶν βασιλεὺς, král Thráků (D. S. 12.50).} S řecky mluvícími městy udržovali kontakty právě tito thráčtí aristokraté, kteří pocházeli z kmenů žijících v blízkosti řeckých sídlišť na pobřeží. Zásadní roli mezi thráckou aristokracií hrálo společenské postavení a status, který si aristokraté pečlivě budovali. Hérodotos nás informuje o rozšířené praxi mnohoženství, které bylo známkou společenské prestiže, protože několik žen si mohl dovolit jen bohatý člověk (Hdt. 5.5; Strabo 7.3.4). Další známkou společenského postavení byl zvyk přijímat, a nikoliv dávat bohaté dary, který popisuje Xenofón (Xen. {\em Anab.} 7.3.18-20). Čím vyšší bylo společenské postavení obdarovaného, tím větší a cennější dar se očekával. Úspěšný aristokrat si takto mohl opatřit poměrně velké bohatství, jako to Thúkýdidés popisuje u odryského panovníka Sitalka (2.97.3; D. S. 12.50). Bohatství pak Thrákové vystavovali na odiv v rámci společenských setkání, ale i zhotovováním nákladných hrobek a konáním pohřebních rituálů, které jsou u nich v oblibě (Hdt. 5.8).

Hérodotos dělí thrácké kmeny dle vztahu k Řekům a řecké kultuře jednak na civilizované kmeny, které jsou v kontaktu s řeckou komunitou a jsou do velké míry provázány společnými zájmy a žijí v dosahu pobřeží a řeckých kolonií,\footnote{Hdt. 7.110 a 7.115; Hdt. 6.34 např. kmen Dolonků.} a dále na kmeny žijící v horách, které jsou na Řecích nezávislé či jsou vůči nim nepřátelsky naladěné, mají bojovný charakter a udržují si své tradiční zvyky.\footnote{Hdt. 7.111 a 7.116, např. kmen Satrů či Hdt. 6.36 a 9.119 kmen Apsinthiů, Hdt. 5.124-126 kmen Édónů, Hdt. 6.45 kmen Brygů. V lidové tradici a mýtech je Thrákie je taktéž vnímána jako domov mýtického pěvce Orfea, a místo, kde se v odlehlých horách odehrávají divoké bakchické rituály (Pindar frg. 126.9; Paus. 6.20.18; Archibald 1999, 460).} Postoj thráckých aristokratů vůči řecké komunitě se různí, od nepřátelských kontaktů až po snahu některých jedinců o adopci řeckého stylu života a vzdělání.\footnote{Hdt. 4.78-80: Hérodotos popisuje situaci u Skýthů, přímých sousedů Thráků, u nichž mohla situace vypadat analogicky. Autor mluví o skýthském princi Skýlovi, který pocházel ze smíšeného svazku a uměl mluvit a psát řecky a byl nakloněn řeckému způsobu života, např. nosil řecký oděv, vyznával řecká božstva, a nakonec se usídlil v řeckém Borysthénu.} Ač jsou tito Thrákové řeckými prameny popisováni jako „hellénizovaní”, i nadále si udržují své typické zvyky a náboženství (Hdt. 5. 3-8).\footnote{Známá pasáž z Hérodota (Hdt 5. 7) je zářným příkladem: Hérodotos tvrdí, že Thrákové uctívají Area, Dionýsa, Artemidu a aristokraté ještě navíc Herma. Zůstává nadále otázkou, nakolik řecké prameny přizpůsobovaly reálie řeckému posluchači ({\em interpretatio Graeca}) a nakolik skutečně reflektovaly realitu. Nicméně srovnání s dostupnými archeologickými prameny poukazuje na uchování tradičního charakteru náboženství za užití řecké nomenklatury (Janouchová 2013a; Janouchová 2013b). Více k tomuto tématu v kapitole 6.} Ke kontaktům a prolínání kulturních zvyklostí mezi Thráky a Řeky docházelo i dle Helláníka (frg. 71a), který nazývá původně thrácké obyvatele polovičními Řeky, avšak tento popis se týká pouze thráckých kmenů žijících na pobřeží v těsné blízkosti řeckých měst.\footnote{Termín μιξέλληνες Helláníkos používá k popisu obyvatel thráckého pobřeží známého jako Kolpos Melas, mezi Thráckým Chersonésem a pevninou.}

Politické vztahy a vzájemné postavení řeckých měst a thráckých kmenů literární prameny přímo nezmiňují, nicméně Thúkýdidés nepřímo informuje o jistém druhu finanční nadřazenosti Thráků nad Řeky žijícími v Thrákii, když poukazuje na nemalé finanční částky, které kmen Odrysů vybíral právě od Řeků žijících na území Thrákie (Thuc. 2. 97). Kmen Odrysů si získal výsadní postavení v řecké historiografii, zejména pro svou důležitost v rámci athénské zahraniční politiky v 5. a ve 4. st. př. n. l., a tudíž nelze vztahovat vyjádření týkající se Odrysů obecně na všechny thrácké kmeny. Je však jisté, že Thrákové po většinu své historie netvořili jednotný stát, který by zahrnoval celou oblast Thrákie, ale spíše se jednalo o kmeny s podobnými zvyky a příbuznými dialekty (Theodossiev 2011, 2), které se dostávaly do kontaktu s řeckou komunitou.

Thrákie a thráčtí kmenoví vůdci byli pověstní svou nejednotností a odporem vůči jakékoliv autorit, tedy i vůči vlastním kmenovým vůdcům. Pokud však vysoce postavení Thrákové vycítili politickou příležitost, dokázali využít situace za účelem naplnění vlastních mocenských ambicí, a to i za cenu ztráty autonomie v dlouhodobém měřítku (Haynes 2011, 7). To je případ thráckých panovníků z 1. st. př. n. l. a 1. st. n. l., kteří se vzdali politické autonomie a stali se vazaly Říma výměnou za vlastní prospěch (Tac. {\em Ann}. 4.46-47; Lozanov 2015, 75).

