
\subsection[thrákie-v-literárních-pramenech]{Thrákie v literárních pramenech}

Thrákie patří k relativně dobře popsaným regionům antického světa, alespoň co se týká pobřežních oblastí, které byly v archaické době osídleny řecky mluvícími obyvateli.\footnote{Velmi užitečné informace o thráckém pobřeží, řeckých koloniích a obyvatelích poskytuje autor 4. st. př. n. l. Pseudo-Skylax ve svém Periplu, kap. 67, 1-10 (Shipley 2011, 69-71). Pseudo-Skylax uvádí, že po moři je možné obeplout thrácké pobřeží od řeky Strýmónu až k Séstu za dva dny a dvě noci, ze Séstu do Pontu pak za další dva dny a dvě noci. Odtud pak k řece Istru za tři dny a tři noci. Celkem je tedy dle něj možné obeplout Thrákii po moři za osm dní (67.10), {[}ač prostý součet je dní sedm, poznámka P. J.{]}. Dalším autorem popisujícím především řecká města na pobřeží je Pseudo-Skymnos (646-746), jehož dílo bývá datováno do 1. st. př. n. l. (Kazarow 1949, 143). Důležitým zdrojem z 1. st. n. l. je Klaudios Ptolemaios, který udává rozměry a uspořádání provincií {\em Thracia} a {\em Moesia Inferior} ({\em Geogr}. 3.10-11). Obecně avšak autoři římské doby udávají více detailů jako například vzdálenosti mezi městy (Arrián {\em Peripl. Ponti Euxini} 21-24).} Konkrétní hranice Thrákie se měnily v závislosti na politické situaci a na vnímání daného autora, nicméně pohoří a vodní plochy stanovily poměrně stabilní přírodní hranice. Tradičně dle antických pramenů území Thrákie začínalo již za makedonskou Piérií, ze západu bylo pak ohraničena řekami Strýmónem a Néstem, z jihu Egejským mořem, Marmarským mořem a Helléspontem, z východu pak Černým mořem. Severní hranice Thrákie jsou méně jasně geograficky ohraničené, ale většinou se za severní hranici považuje řeka {\em Istros} (dn. Dunaj) a pohoří {\em Haimos} (dn. Stara Planina, téže Balkán). Horské masivy Rodopy, Pirin a Rila tvořily relativně neprostupnou bariéru, oddělující pobřežní od vnitrozemské Thrákie, a tedy Řeky od Thráků. Spojnicí vnitrozemí se Středozemím tvořily zejména řeky {\em Hebros} (dn. Marica-Evros), {\em Tonzos} (dn. Tundža-Tonzos), {\em Strýmón} (dn. Struma-Strymónas) a {\em Néstos} (dn. Mesta-Néstos) a několik cest skrze horské průsmyky (Sears 2013, 7-8; Bouzek a Graninger 2015, 12-19; Theodossiev 2011, 2-4; Theodossiev 2014, 157).\footnote{Někteří autoři zařazují do území obývané Thráky oblasti Bíthýnie v Malé Asii, ostrovy Thasos a Samothráké, a území na severu sahající až k řece Morava (Theodossiev 2014, 157).}

Thrákie je v literárních pramenech z archaické a klasické doby známá pro své nerostné bohatství, úrodnou zemědělskou půdu, nevlídné a kruté zimy, vodnaté řeky, vysoké hory a široké planiny, na nichž se dařilo chovu koní, které zmiňuje již Homér (Hom. {\em Il}. 13.1-16; 10.484; Sears 2013, 31; Tsiafakis 2000). Již od dob Archilocha je Thrákie proslulá svými silným vínem a existencí bohatých zlatými a stříbrných dolů (Diehl frg. 2 a 51), k nimž se jako první z Řeků pokoušeli dostat kolonisté z ostrova Paros.\footnote{V pozdějších dobách se o zdroj nerostných surovin začali zajímat i Athéňané (např. Peisistratos, Hdt. 1.64.1; Isaac 1986; 14-15; Lavelle 1992, 14-22).} Hérodotos také zmiňuje i nadbytek dřeva vhodného ke stavbě lodí, jednu z hlavních komodit antického starověku (Hdt. 5.23.2). Celkově je možné říci, že Thrákie byla známá jako plodná, ale poměrně drsná krajina, a podobně řecké prameny líčily i její obyvatele.

