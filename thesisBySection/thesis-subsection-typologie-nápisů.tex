
\subsection[typologie-nápisů]{Typologie nápisů}

Dochované nápisy z 6. st. př. n. l. je možné typologicky i obsahově určit jako soukromé funerální texty, vytvořené za účelem označení místa pohřbu a jako připomínka zesnulého pro okruh nejbližších lidí. Krátký rozsah textů poukazuje na účelnost sdělení: typický text obsahuje jméno zesnulého a určení jeho biologického původu udáváním jména rodiče, případně zhotovitele nápisu. Nápisy mohou promlouvat ke čtenáři: personifikovaný náhrobní kámen oznamuje komu přesně patří a jehož život připomíná.\footnote{Např. {\em Perinthos-Herakleia} 69: Ἡγησιπόλης εἰμὶ τῆς Ἡγεκράτεος, „Náležím Hégésipole, dceři Hégékratea”.} Obsah těchto sdělení měl význam převážně v nejbližší komunitě, kde každý znal zesnulého či jeho rodinu.

Ze 6. st př. n. l. tak nemáme žádné důkazy o prolínání řeckého a thráckého obyvatelstva. Nápisy svým kontextem i obsahem pocházejí z řeckých komunit a jejich charakter poukazuje na udržování tradičních společenských norem i v rámci nově vzniklých kolonií. Osobní jména, která se na nápisech vyskytovala, byla výhradně řeckého původu, dokonce i ve sledu dvou generací. Na nápisech nenalézáme žádné další vyjádření identity, ani se zde nevyskytují hledané společensko-kulturní termíny, až na jednu výjimku výskytu formulí typických pro náhrobní nápisy v řeckém světě.\footnote{Termín {\em mnéma} pro označení hrobu či náhrobního kamene samotného.} Udržení tradiční formy i obsahu poukazuje na relativní uzavřenost tehdejších komunit a konzervativnost projevů epigraficky aktivní společnosti, tedy lidí, kteří se podíleli na publikování nápisů.

