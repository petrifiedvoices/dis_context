
\subsection[veřejné-nápisy-11]{Veřejné nápisy}

Veřejných nápisů se dochovalo celkem 28, z čehož šest pochází z Perinthu a Byzantia a tři z Filippopole. Perinthos se po polovině 1. st. n. l. stal sídlem místodržícího provincie {\em Thracia}, a tak není výskyt celkem 10 nápisů překvapivý (Lozanov 2015, 76).\footnote{Hlavním městem provincie {\em Thracia} se stává Perinthos, kde vznikla i většina institucí a město se stalo jak sídlem místodržícího, tak zde sídlila vojenská posádka, a město se tak stalo přirozeným produkčním centrem veřejných nápisů (Sharankov 2005, 521; Sayar 1998, 74).} V 10 případech se jedná o honorifikační dekrety, v pěti případech o texty náboženského charakteru a další kategorie jsou zastoupeny v řádech kusů nápisů. Politickou autoritu představuje římský císař, případě vysocí provinciální úředníci, kteří ho zastupují. Instituce jednotlivých měst mohou vydávat nařízení pod patronátem římské říše či thráckého krále, avšak převážně se veřejné nápisy vydané městskými institucemi omezují na honorifikační dekrety.\footnote{Např. nápis {\em SEG} 55:752 z Filippopole či {\em I Aeg Thrace} 83 z Abdéry. Politická autorita se proměňuje v závislosti na změně politické situace. Do poloviny 1. st. n. l. jako autority vystupuje jednak lid, instituce řeckých měst a thráčtí panovníci, kteří však byli smluvně zavázání Římu. Tito panovníci z kmene Odrysů, Astů a Sapaiů jsou známí jako vazalští králové Říma, kteří si částečně udržují autonomii, ale ve velké míře podléhají vlivu a politickým rozhodnutím Říma (Lozanov 2015, 78-80). Pravděpodobně na začátku 1. st. n. l. dochází k rozdělení Thrákie na tzv. stratégie, tedy administrativní a vojenské regiony, které byly oficiálně řízeny thráckými stratégy z řad thráckých aristokratů. Gabriella Parissaki navrhuje, že se jednalo o mezistupeň mezi tradičním kmenovým uspořádáním a centralizovanou samosprávou římské provincie, kde thráčtí aristokraté sehráli významnou roli (Parissaki 2009, 320-328).} V této době se taktéž objevuje první milník, informující o době vzniku daného úseku cesty, a první nápisy informující o císařském provinciálním stavebním programu.\footnote{{\em I Aeg Thrace} 453 z lokality Ferai, datovaný do doby vlády císaře Nerona a {\em IG Bulg} 5 5691 z města Serdica, což představuje první důkaz o císařem organizované stavbě komunikací v Thrákii na pobřeží Egejského moře i ve vnitrozemí. Budování cest pravděpodobně probíhalo již v dřívějších stoletích, ale z té doby se nám nedochovaly milníky či jiné epigrafické záznamy (Lozanov 2015, 76; Madzhahov 2009, 63-65).; {\em IG Bulg} 1,2 57 z Odéssu a {\em IK Sestos} 29 ze Séstu slouží jako doložení stavebních aktivit.} Zmínky o thrácké aristokracii jako takové a thráckých králích z epigrafických záznamů zcela mizí po polovině 1. st. n. l., ač systém stratégií se ještě udržuje několik desetiletí (Lozanov 2015, 78-81). Thrácká aristokracie se tak pravděpodobně adaptovala na nové podmínky a zaujala roli v provinciální samosprávě, právě v roli stratégů.

Dochází také k pozvolné proměně použitého jazyka veřejných nápisů a instituce spojené s chodem římské provincie se začínají objevovat ve zvýšené míře.\footnote{Celkem 24 administrativních termínů se vyskytuje ve 37 případech. Většina termínů se již objevila v minulých stoletích, mezi nové termíny však patří {\em hegémón}, {\em kaisar} jako termíny označující římského císaře, dále {\em búleutés} jako člen {\em búlé}, dále {\em agoranomos}, {\em nauarchos}, a nakonec termín pro hlavní město {\em métropolis}. Mezi nejčastěji se opakující termín patří i nadále {\em démos} s 6 výskyty, a {\em gymnasiarchés} se čtyřmi výskyty.} Osobní jména poukazují na proměňující se trendy při výběru jmen a přijímání římských onomastických tradic zejména mezi Thráky sloužícími v římské armádě a vykonávajícími vysoké úřednické funkce. Lidé zastávající funkci stratéga běžně nesou tři jména, kombinující římská jména, jako Titos, Flavios, Tiberios, Gaios, Ioulios nebo Klaudios, spolu s osobními jmény thráckého původu.\footnote{Např. na nápise {\em I Aeg Thrace} 84 z Topeiru.}

