
\subsection[veřejné-nápisy-16]{Veřejné nápisy}

Nápisů datovaných do 3. až 4. st. n. l. se dochovalo dohromady 12, což představuje téměř 50 \letterpercent{} všech nápisů z dané skupiny. Celkem se jedná o čtyři honorifikační dekrety, čtyři milníky, tři hraniční kameny a jeden nápis s věnováním císařům. Všechny nápisy pochází z období tetrarchie a jsou úzce spojeny s osobnostmi císařů.

Všechny honorifikační dekrety pocházejí z Hérakleie (původního Perinthu), která se v této době hrála důležitou roli regionálního centrum, než se jím v roce 330 n. l. stala Konstantinopol, bývalé Byzantion. Všechny čtyři nápisy jsou věnované obyvateli Hérakleie tehdejším císařům Diokletiánovi, Maximiánovi, Konstantinu Chloru a Galeriovi. Stejně tak i dochované tři hraniční kameny jsou věnovány tetrarchům, kteří tak vyměření hranic území daného sídla propůjčují patřičnou legitimitu. \footnote{Ve všech případech se jedná o hraniční kámen místní thrácké vesnice, neznámé z dalších historických zdrojů. {\em I Aeg Thrace} 398 jedná se o vyznačení hranic osídlení ({\em chórióma}) Barilos (?). {\em I Aeg Thrace} 382 vyznačení hranic vesnice Eresén{[}-{]}. {\em I Aeg Thrace} 383 vyznačení hranic neznámé vesnice.} Dochované čtyři milníky pocházejí z okolí {\em Via Diagonalis} (Serdica, Augusta Traiana a Perinthos, tehdy již jako Hérakleia) a jsou datované na přelom 2. a 3. st. n. l., kdy docházelo k stavebním aktivitám a úpravám této významné silnice organizovaným tehdejšími císaři na začátku 4. st. n. l.

