
\subsection[veřejné-nápisy-6]{Veřejné nápisy}

Celkem se dochovalo 20 nápisů spadající do kategorie veřejných nápisů: převážná většina z nich byly dekrety vydávané v rámci řeckých městských států na pobřeží, přičemž šest dekretů pochází z Odéssu a pět z Maróneie. Politickou autoritu jednotlivých měst zastupovaly orgány jako {\em démos} a {\em búlé} a nejčastější druhem dokumentu jsou honorifikační dekrety vystavené pro význačné osoby či osoby, které se výjimečným způsobem zasloužili o udělení poct. Dle dochovaných osobních jmen se většinou jednalo o muže řeckého původu. V několika případech známe jejich mateřskou obci: Athény, Kallatis, Chersonésos a Antiocheia.\footnote{{\em I Aeg Thrace} 172, {\em IG Bulg} 1,2 13ter, {\em IG Bulg} 1,2 39, {\em IG Bulg} 1,2 41. V případě seznamu osob {\em Perinthos-Herakleia} 62 z Perinthu se dochovala jména šesti mužů řeckého původu. Svou identitu udávají pomocí osobního jména a jména otce, a dále specifikují svůj původ: tři muži pocházejí z fýly {\em Théseis}, dva z fýly {\em Basileis} a jeden z města Byzantion.}

Texty honorifikačních nápisů vycházely ze stejného základu, nicméně každý nápis byl přizpůsoben konkrétní situaci. Konkrétní znění dekretů se lišilo město od města, což poukazuje na jejich politickou samostatnost a nezávislý vývoj epigrafických formulí, které nicméně vycházejí ze společného základu. Součástí textu byly občas i podmínky zveřejnění nápisu, což pro řecké komunity zpravidla bývalo umístění veřejného nápisu do svatyně patrona daného města, jak je typické i pro jiné řecké komunity.\footnote{Text nápisů {\em IG Bulg} 1,2 41 a 42 z Odéssu nařizuje umístit honorifikační nápis v podobě sochy do svatyně samothráckých božstev a do svatyně anonymního božstva.}

