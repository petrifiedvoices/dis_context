
\subsection[veřejné-nápisy-8]{Veřejné nápisy}

Celkem se dochovalo 14 veřejných nápisů, což značí propad oproti předcházejícím obdobím. Krom jednoho nápisu z thráckého Kabylé všechny nápisy pocházejí z okolí řeckých měst na pobřeží. Klesající počet nápisů může naznačovat na pokles moci publikujících politických autorit, či poukazuje na nedostatečnou míru prozkoumání kulturních vrstev 2. a 1. st. př. n. l. Nejčastější termíny jsou stále {\em démos}, {\em búlé}, {\em politai} a {\em pséfisma}, a i nadále fungují dříve ustanovené procedury spojené se zhotovováním a vystavováním nápisů, nicméně klesající počet termínů označujících instituce může naznačovat jejich postupný úpadek či pozbytí významu v rámci fungování obce. Forma honorifikačních dekretů z Odéssu si udržela stejnou formu jako u nápisů datovaných do 2. st. př. n. l. a pravděpodobně vycházela ze stejných předpisů a pravidel. Honorifikační nápisy z Mesámbrie a Apollónie mají zcela jinou formu a používají jiné formule, což nasvědčuje o stále trvající regionální autonomii řeckých měst.

Nápis {\em I Aeg Thrace} 212 představuje seznam věřících kultu Sarápida a Ísidy z Maróneie a objevuje se na něm až 75 jmen převážně řeckého původu, avšak i se sedmi jmény římskými a dvěma thráckými. Z přítomnosti osobních jmen je patrné, že kult byl oblíben převážně u mužů nesoucí řecká jména, nicméně byl přístupný i Thrákům a Římanům. Funkce kněžích zastávali muži nesoucí řecká jména a z velké části nedocházelo k mísení onomastických tradic, tj. římská jména tvořila osobní jméno výhradně s dalším jedním či dvěma římskými jmény a nedocházelo k jejich prolínání s řeckými či thráckými jmény. Relativní izolovanost římských jmen svědčí o tom, že zvyk přijímat římská jména se v tomto období ještě neprosadil v podobě, jaká bude běžná v následujících stoletích.

