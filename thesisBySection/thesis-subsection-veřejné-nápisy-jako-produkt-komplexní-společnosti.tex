
\subsection[veřejné-nápisy-jako-produkt-komplexní-společnosti]{Veřejné nápisy jako produkt komplexní společnosti}

Rozmístění veřejných nápisů odpovídá funkci, kterou hrály v rámci organizace společnosti, jako např. uchovávání a rozšiřování informací důležitých pro chod společnosti a udržení pořádku. Veřejné nápisy sloužily nejen jako zdroj informací, ale také samy byly projevem politické moci daného subjektu, a proto se vyskytují v hojné míře v administrativních a politických centrech, ať už regionálního či nadregionálního charakteru (Tainter 1986, 99-106).

Mapa 7.07 v Apendixu 2 velmi dobře ilustruje rozmístění veřejných nápisů a jejich vzdálenosti od městských center, případně cest. Z celkem 695 veřejných nápisů pochází 79 \letterpercent{} ze vzdálenosti do 20 km, a zbývajících 21 \letterpercent{} nápisů ze vzdálenosti nad 20 km. Skupina nápisů nalezených na území ve vzdálenosti maximálního denního pochodu, tedy do 40 km od měst, představuje 92 \letterpercent{} všech nápisů. Zbylých 8 \letterpercent{} nápisů pochází ze vzdálenosti větší než 40 km od měst a nachází se většinou v přímé blízkosti silnic. Obsah této menší skupiny milníků a stavebních nápisů nejčastěji souvisí právě s údržbou římských silnic a vytyčováním vzdáleností.\crlf
Jedním z příkladů projevů politické autority a existující infrastruktury v krajině je rozmístění milníků, které označovaly vzdálenost k významnému městu v římské době (Hollenstein 1975, 23-45; Madzharov 2009, 57-59). Zpravidla milník udával vzdálenost v mílích k městu, do jehož regionu spadala správa cesty. Nápis byl umístěn viditelně vedle cesty tak, aby si každý cestující zjistil jednak jak velkou vzdálenost mu zbývá ujít či ujet do daného města, ale nápis také nesl informace o tom, kdo cestu spravuje, případně se postaral o její postavení či opravy. Na mapě 7.08 v Apendixu 2 je velice dobře vidět, že nálezová místa milníků kopírují trasu známých římských cest.\footnote{Madzharov (2009, 57-59) udává celkový počet řeckých i latinských milníků z území dnešního Bulharska na 180 exemplářů. Mapa 7.08 v Apendixu 2 zobrazuje pouze datované milníky, z nichž většina je psána řecky. Latinsky psané milníky nejsou z větší části zahrnuty do HAT databáze.} Nejvíce milníků pochází z tzv. {\em Via Diagonalis}, což byla jedna z nejvýznamnějších cest Balkánu, protože spojovala východní provincie se západem a sloužila k častému přesunu vojsk mezi lokalitou Singidunum, dnešním Bělehradem, a Byzantiem, dnešním Istanbulem (Jireček 1877; Madzharov 2009, 70-131). Desítky milníků se dochovaly z okolí města Serdica, Filippopolis a Augusta Traiana, pod jejichž správu vybrané úseky {\em Via Diagonalis} spadaly. První milník se objevil již v 1. st. n. l. v oblasti {\em Via Egnatia} na egejském pobřeží, nicméně stavební aktivity ve vnitrozemské Thrákii jsou dobře dokumentované až z 2. st. n. l. Nejvíce milníků pochází z 3. st. n. l., kdy docházelo k úpravám {\em Via Diagonalis}, vzhledem k jejímu častému využití římskými vojsky (Hollenstein 1975, 27-41). Na přelomu 3. a 4. st. n. l. docházelo k omezení stavebních aktivit a odpovídá tomu i menší počet dochovaných milníků. Poslední milník se dochoval ze 4. st. n. l. z {\em Via Egnatia} v okolí Perinthu.

Z některých úseků pochází velké množství milníků, např. v okolí města Serdica, ale z velké části cest se nedochoval ani jeden milník. Tento fakt je možné přisuzovat stavu archeologických výzkumů, které se zaměřují na zkoumání osídlení, což vede k relativně málo známému systému římských cest v Thrákii. Výzkumy posledních let se začínají zaměřovat i na síť silnic, jako na důležitou součást provinciálního uspořádání, a v budoucnosti se tak dají očekávat nové objevy, které mohou přinést i nové milníky a související stavební nápisy.

