
\subsection[římské-provincie-thracia-a-moesia-inferior]{Římské provincie Thracia a Moesia Inferior}

Bezprostředně po vzniku nových provincií {\em Moesia Inferior} a {\em Thracia} nedošlo k zásadním proměnám společenského uspořádání a administrativy, ale spíše došlo k navázání na již existující infrastrukturu vytvořenou vazalskými panovníky. Systém stratégií přetrval pravděpodobně až do začátku 2. st. n. l. a výsadní pozici si udržovala i nadále thrácká aristokracie, která se dokázala adaptovat na nové podmínky.\footnote{Role stratégií zůstala pravděpodobně stejná, avšak s postupem času se snižoval jejich počet z původních 50 na 14 (Plin. {\em H. N.} 4.11.40; Ptolem. {\em Geogr.} 3.11.6; Jones 1971, 10-15).} Odměnou za jejich loajální služby bylo udělení římského občanství a vysoká pozice v provinciálním aparátu (Lozanov 2015, 80-82). Některé z řeckých měst si udržela nezávislost na systému stratégií a pravděpodobně si uchovala politickou a ekonomickou autonomii. Plinius zmiňuje specificky Abdéru, Maróneiu, Ainos a Byzantion (Pliny {\em H. N.} 4.42-43). Naopak Anchialos a Perinthos se stala hlavními městy stratégií, ale byla jim ponechána určitá autonomie (Jones 1971, 15).

Provincie {\em Thracia} byla územím bez trvale usídlené legie ({\em provincia inermis}) a existovalo zde pouze několik pravidelných jednotek, skládajících se mimo jiné i z místního obyvatelstva, jako např. tábor v Kabylé či Germaneii. Oproti tomu {\em Moesia Inferior} byla provincií se silnou vojenskou přítomností, zejména podél řeky Dunaje, kde byla vytvořena soustava vojenských táborů a opevnění, což mělo vliv i na civilní obyvatelstvo (Haynes 2011, 7-9; Lozanov 2015, 80-82).

Předpokládá se také, že Thrákové se aktivně účastnili služby v římské armádě již v průběhu 1. st. př. n. l. a jejich nábor měl mít na starosti stratégos, který jednal znal nejlépe místní poměry, ale zároveň prokázal svou věrnost Římu (Lozanov 2015, 79-81). Z literárních a epigrafických pramenů pocházejících z celé římské říše víme, že již v 1. st. n. l. existovaly pomocné vojenské jednotky ({\em auxilia}), které nesly ve jméně thrácký původ. Tyto vojenské jednotky se skládaly jak z jezdců ({\em alae}), tak z pěšího vojska ({\em cohortes}) a sloužily po celém území římské říše (Jarrett 1969, 215). Jména pomocných jednotek nesla typicky odkaz na thrácký původ, např. {\em alae Thracorum}, či {\em alae Bessorum}.\footnote{Kmen Bessů byl používán nikoliv proto, že by všichni členové jednotky pocházeli z kmene Bessů, ale jako zástupné obecné pojmenování původu vojáků. Tento jev byl součástí verbovací strategie římských pomocných jednotek a je pozorovatelný i na jiných místech římského impéria, jako např. v Batávii (Derks 2009, 239-270). Thrákové naverbovaní do římské armády byli zařazeni do jednotky, která primárně nemusela odpovídat jejich kmenové příslušnosti, ale v očích římské administrativy byl zvolen jeden kmen, podle nějž se jednotky jmenovaly. Thrákové byli nuceni se vstupem do římského vojska adoptovat novou identitu, která se zcela nemusela shodovat s jejich původem. Thrákové však na svůj původ nezanevřeli zcela a na vojenských diplomech často uvádí jednak svůj etnický původ, ale i konkrétní obci či vesnici, z níž pocházejí, což spolu s uchováním thráckých osobních jmen poukazuje na silný tradicionalismus (Dana 2013, 246).} Thrákové se stali jedním z nejpočetnějších etnik římské armády a po odsloužení 25 let vojenské služby se často vraceli jako veteráni zpět do vlasti, zakládali nová sídliště a podíleli se na správě provincie. Ke konci 1. st. n. l. byla zakládána nová města, kde byli umísťováni veteráni římské armády, jako např. {\em Colonia Flavia Pacis Deultensium}, známá jako Deultum, či {\em Colonia Claudia Aprensis}, známá jako Apros (Lozanov 2015, 85). Veteráni tak měli zajišťovat bezpečí v nejbližším okolí strategicky umístěných měst výměnou za pozemky. Tento trend byl ještě patrnější severně od pohoří Haimos v {\em Moesii Inferior}, kde v okolí vojenských táborů vznikala civilní osídlení, která zásobovala početné římské vojsko. Ve stejné době se zde rozvinul systém {\em vill}, tedy venkovských usedlostí zaměřených primárně na zemědělskou produkci, které se nacházely v blízkosti vojenských táborů, avšak dostatečně daleko od hranic.

