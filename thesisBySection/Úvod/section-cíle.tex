
\environment ../env_dis
\startcomponent section-cíle
\section[cíle]{Cíle}

Tato práce si klade za cíl zhodnotit na základě studia dochovaných epigrafických pramenů, zda je na počátku 21. století koncept {\em hellénizace} vhodným interpretačním rámcem vysvětlujícím vznik a rozvoj epigrafické produkce na území Thrákie, či je možné ho nahradit novými metodologickými přístupy. Součástí je i kritické zhodnocení, nakolik je možné i nadále používat mnohdy jednostranně zaměřený přístup hellénizace ke studiu epigrafické produkce z území Thrákie, a nakolik se jedná v dnešní době již o překonaný teoretický koncept.

Dalším cílem této práce je charakterizovat epigrafickou produkci z území Thrákie v co možná nejucelenější podobě a zhodnotit, jaký význam měla nápisná kultura v tehdejší společnosti. Jak je známo, přístup společnosti k epigrafické produkci se vyvíjel nejen v závislosti na čase s ohledem na tehdejší politické události, ale významnou roli hrály i zeměpisné a demografické faktory, což se odráží i na uspořádání této práce. Jedním z cílů této studie je vytvořit chronologický průřez společností antické Thrákie od 6. st. př. n. l. do 5. st. n. l. a zhodnotit, jakou roli v ní zaujímal zvyk publikovat nápisy, v jakých komunitách se úspěšně prosadil a v jakých nedošlo k jeho rozvoji a proč. Epigrafické prameny slouží jako primární zdroj informací k diskuzi o proměňujících se demografických trendech, kulturních zvyklostech a složení společnosti antické Thrákie. Dochované nápisy umožňují mapovat kulturní změny a projevy těchto změn v epigrafické kultuře, s cílem postihnout důvody, proč k těmto změnám docházelo, případně jaký byl jejich dopad na tehdejší společnost. V neposlední řadě mě zajímá, zda je možné spojovat epigrafickou produkci s projevy vlivu řeckých obcí bez ohledu na časové období, či zda je nárůst epigrafické produkce spojen spíše s měnící se úrovní společenské komplexity a přístupem tehdejší politické autority.

\stopcomponent