
\environment ../env_dis
\startcomponent section-hypotézy-a-zvolená-metoda
\section[hypotézy-a-zvolená-metoda]{Hypotézy a zvolená metoda}

V disertační práci se snažím prokázat, že k rozšíření epigrafické produkce v Thrákii došlo v souvislosti s aktivitami politických autorit dané doby, spíše než jako projev kulturního vlivu řecké společnosti a jako důsledek setkávání thrácké a řecké kultury. Ve snaze o objektivní přístup vůči kontaktům řecko-římské a místní thrácké kultury přistupuji k mapování změn společnosti antické Thrákie na základě zasazení do konkrétního časového a místního kontextu jednotlivých komunit a zhodnocení funkcí, které nápisy v tehdejší společnosti zastávaly. Zajímají mě především důsledky dlouhodobých změn politického uspořádání a složení struktury populace. V neposlední řadě se zaměřuji na projevy kulturních zvyklostí v závislosti na probíhajících mezikulturních kontaktech a vzájemném ovlivňování thrácké a řecké, případně makedonské a římské společnosti.

Jako hlavní pramen používám dochované a publikované nápisy, pocházející z oblasti Thrákie. Časové pokrytí této práce se věnuje dlouhodobým procesům a změnám společnosti od archaické doby, kdy se v oblasti objevily první nápisy, až do doby pozdní antiky, kdy ustává epigrafická produkce v podobě, v jaké se vyskytuje v předcházejících obdobích a dochází ke znatelné proměně tehdejší společnosti. Nápisy jako primární zdroj informací postihují širokou škálu součástí tehdejší společnosti, počínaje pohřebními zvyklostmi, přes náboženství, etnické složení populace až po politické uspořádání komunit. Charakter nápisů umožňuje jednotlivé exempláře poměrně dobře zařadit do konkrétního historického rámce a určit jejich provenienci. S tím úzce souvisí fakt, že nápisy představují jeden z nejlépe zdokumentovaných historických pramenů z dané oblasti, pokrývající více než tisíc let vývoje společnosti. Díky relativně velkému počtu dochovaných nápisů je tak možné provádět relevantní kvantitativní a kvalitativní analýzy za účelem sledování proměn tehdejší společnosti.

Za tímto účelem jsem shromáždila přes 4600 nápisů nalezených na území Thrákie v databázi {\em Hellenization of Ancient Thrace}, která je výchozím pramenným souborem této práce. Struktura databáze respektuje tradiční principy epigrafické disciplíny doplněné o moderní metodologické přístupy a technologie, zefektivňující analýzu velkého množství dat. Analýzy nápisů provádím s pomocí moderní mapovací a statistické technologie a metod, dosud používaných zejména v archeologickém prostředí. Výsledky srovnávám s dochovanými literárními a archeologickými prameny ve snaze dotvořit celkový obraz proměn společnosti antické Thrákie a možných příčin těchto změn.

\stopcomponent